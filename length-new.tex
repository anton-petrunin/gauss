\chapter{Length}
\label{chap:length}

\section{Definitions}

A sequence 
\[a=t_0 < t_1 < \cdots < t_k=b.\]
is called a \index{partition}\emph{partition} of the interval $[a,b]$.

\begin{thm}{Definition}\label{def:length}
Let $\gamma\:[a,b]\to \spc{X}$ be a curve in a metric space.
The \index{length of curve}\emph{length} of $\gamma$ is defined as
\begin{align*}
\length \gamma
&= 
\sup
\set{\dist{\gamma(t_0)}{\gamma(t_1)}{\spc{X}}
+\dots+
\dist{\gamma(t_{k-1})}{\gamma(t_k)}{\spc{X}}}{},
\end{align*}\index{10length@$\length \gamma$}
where the least upper bound is taken over all partitions $t_0,\dots,t_k$ of $[a,b]$.

The length of $\gamma$ is a nonnegative real number or infinity;
the curve $\gamma$ is called \index{rectifiable curve}\emph{rectifiable} if its length is finite. 

The length of a closed curve is defined as the length of the corresponding loop.
If a curve is parametrized by an open or semi-open interval, then its length is defined as the least upper bound of the lengths of all its restrictions to closed intervals.
\end{thm}


\begin{thm}{Exercise}\label{ex:integral-length-0}
Suppose curve $\gamma_1\:[a_1,b_1] \to\mathbb{R}^3$ is a reparametrization of $\gamma_2\:[a_2,b_2] \to\mathbb{R}^3$. 
Show that
\[\length \gamma_1 = \length \gamma_2.\]

\end{thm}

\begin{wrapfigure}[4]{r}{33 mm}
\vskip-0mm
\centering
\includegraphics{mppics/pic-224}
\end{wrapfigure}

Suppose $\gamma\:[a,b]\to \mathbb{R}^3$ is a parametrized space curve.
For a partition $a=t_0 < t_1 < \z\cdots < t_k=b$, set $p_i=\gamma(t_i)$.
Then the polygonal line $p_0\dots p_k$ is said to be \index{inscribed polygonal line}\emph{inscribed} in~$\gamma$.
If $\gamma$ is closed, then $p_0=p_k$, so the inscribed polygonal line is also closed.

Note that the length of a space curve $\gamma$ can be defined as the least upper bound of the lengths of polygonal lines $p_0\dots p_k$ inscribed in~$\gamma$.

\begin{thm}{Exercise}\label{ex:length-chain}
Let $\gamma\:[0,1]\to\mathbb{R}^3$ be a path.
Suppose that $\beta_n$ is a sequence of polygonal lines inscribed in $\gamma$ with vertices $\gamma(\tfrac in)$ for $i\z\in\{0,\dots,n\}$.
Show that 
\[\length\beta_n\to\length\gamma
\quad\text{as}\quad
n\to \infty.
\]
\end{thm}



\begin{thm}{Exercise}\label{ex:length-image}
Let $\gamma\:[0,1]\to\mathbb{R}^3$ be a simple path.
Suppose that a path $\beta\:[0,1]\to\mathbb{R}^3$ has the same image as $\gamma$;
that is, $\beta([0,1])=\gamma([0,1])$.
Show that 
\[\length \beta\ge \length \gamma.\]
Try to prove the same assuming that only that $\beta([0,1])\supset\gamma([0,1])$.
\end{thm}


\begin{thm}{Exercise}\label{ex:integral-length}
Assume $\gamma\:[a,b]\to\mathbb{R}^3$ is a smooth curve.
Show that
\vskip1mm
\begin{minipage}{.49\textwidth}
\begin{subthm}{ex:integral-length>}
$\length \gamma
\ge
\int_a^b|\gamma'(t)|\cdot dt$;
\end{subthm}
\end{minipage}
\hfill
\begin{minipage}{.49\textwidth}
\begin{subthm}{ex:integral-length<}
$\length \gamma
\le
\int_a^b|\gamma'(t)|\cdot dt$.
\end{subthm}
\end{minipage}

\vskip1mm
Conclude that 
\[\length \gamma
=
\int_a^b|\gamma'(t)|\cdot dt.\eqlbl{eq:length}\]
\end{thm}

\begin{thm}{Advanced exercises}\label{adex:integral-length}

\begin{subthm}{adex:integral-length:a}Show that the formula \ref{eq:length} holds for any Lipschitz curve $\gamma\:[a,b]\z\to\mathbb{R}^3$.
\end{subthm}

\begin{subthm}{adex:integral-length:b}
Construct a non-constant curve $\gamma\:[a,b]\to\mathbb{R}^3$ such that $\gamma'(t)=0$ almost everywhere.
(In this case, the formula \ref{eq:length} does not hold for~$\gamma$ despite that both sides are defined.).
\end{subthm}

\end{thm}


\section{Nonrectifiable curves}

Let us describe the so-called \index{Koch snowflake}\emph{Koch snowflake} ---
a classical example of a nonrectifiable curve.

Start with an equilateral triangle.
For each side, divide it into three segments of equal length, and then add an equilateral triangle with the middle segment as its base.
Repeat this construction recursively with the obtained polygons.
\begin{figure}[ht!]
\centering
\includegraphics{mppics/pic-225}
\end{figure}
The Koch snowflake is the boundary of the union of all the polygons.
Three iterations and the resulting Koch snowflake are shown on the diagram.



\begin{thm}{Exercise}\label{ex:nonrectifiable-curve}

\begin{subthm}{ex:nonrectifiable-curve:a} Show that the Koch snowflake is a simple closed curve; that is, it can be parametrized by a circle.
\end{subthm}


\begin{subthm}{ex:nonrectifiable-curve:b} Show that the Koch snowflake is not rectifiable. 
\end{subthm}
\end{thm}
  
\section{Semicontinuity of length}

The lower limit of a sequence of real numbers $x_n$ will be denoted by
\[\liminf_{n\to\infty} x_n.\] 
It is defined as the lowest partial limit; that is, the lowest possible limit of a subsequence of $x_n$.
The lower limit is defined for any sequence of real numbers, and it lies in the {}\emph{extended real line} $[-\infty,\infty]$


\begin{thm}{Theorem}\label{thm:length-semicont}
Length is lower semicontinuous with respect to the pointwise convergence of curves. 

More precisely, assume a sequence
of curves $\gamma_n\:[a,b]\to \spc{X}$ in a metric space $\spc{X}$ converges pointwise 
to a curve $\gamma_\infty\:[a,b]\to \spc{X}$;
that is, for any fixed $t \in [a,b]$, we have $\gamma_n(t)\z\to\gamma_\infty(t)$ as $n\to\infty$. 
Then 
$$\liminf_{n\to\infty} \length\gamma_n \ge \length\gamma_\infty.\eqlbl{eq:semicont-length}$$
\end{thm}



\parit{Proof.}
Fix a partition $a=t_0<t_1<\dots<t_k=b$.
Set 
\begin{align*}\Sigma_n
&\df
\dist{\gamma_n(t_0)}{\gamma_n(t_1)}{}
+\dots+
\dist{\gamma_n(t_{k-1})}{\gamma_n(t_k)}{}.
\\
\Sigma_\infty
&\df
\dist{\gamma_\infty(t_0)}{\gamma_\infty(t_1)}{}
+\dots+
\dist{\gamma_\infty(t_{k-1})}{\gamma_\infty(t_k)}{}.
\end{align*}

Note that for each $i$ we have 
\[\dist{\gamma_n(t_{i-1})}{\gamma_n(t_i)}{}
\to
\dist{\gamma_\infty(t_{i-1})}{\gamma_\infty(t_i)}{},\]
and therefore
\[\Sigma_n\to \Sigma_\infty\] 
as $n\to\infty$.
Note that 
\[\Sigma_n\le\length\gamma_n\]
for each~$n$.
Hence,
$$\liminf_{n\to\infty} \length\gamma_n \ge \Sigma_\infty.$$

  
Since the partition was arbitrary, by the definition of length, the inequality  \ref{eq:semicont-length} is obtained.
\qeds


\begin{wrapfigure}{o}{20 mm}
\vskip3mm
\centering
\includegraphics{mppics/pic-6}
\end{wrapfigure}


The inequality \ref{eq:semicont-length} might be strict.
For example, the diagonal $\gamma_\infty$ of the unit square 
can be approximated by stairs-like polygonal lines $\gamma_n$
with sides parallel to the sides of the square ($\gamma_6$ is in the picture).
In this case,
\[\length\gamma_\infty=\sqrt{2}\quad
\text{and}\quad \length\gamma_n=2\]
for any~$n$.

\begin{thm}{Exercise}\label{ex:cont-length}
Let  $\gamma\:[a,b]\to \spc{X}$ be a rectifiable curve in a metric space.
Given $t\in [a,b]$, denote by $s(t)$ the length of arc $\gamma|_{[a,t]}$.
Show that the function $t\mapsto s(t)$ is continuous.
\end{thm}


  
\section{Arc-length parametrization}

We say that a curve $\gamma$ has an \index{arc-length parametrization}\emph{arc-length parametrization} (also called \index{natural parametrization}\emph{natural parametrization})
if 
\[t_2-t_1=\length \gamma|_{[t_1,t_2]}\]
for any two parameter values $t_1<t_2$;
that is, the arc of $\gamma$ from $t_1$ to $t_2$ has length $t_2-t_1$.

Note that a smooth space curve $\gamma(t)=(x(t),y(t),z(t))$ is an arc-length parametrization if and only if it has unit velocity vector at all times;
that is, 
\[|\gamma'(t)|=\sqrt{x'(t)^2+y'(t)^2+z'(t)^2}=1\]
for all $t$; by that reason smooth curves equipped with an arc-length parametrization are also called \index{unit-speed curve}\emph{unit-speed curves}.
Observe that smooth unit-speed curves are automatically regular (see \ref{sec:Smooth curves}).

Any rectifiable curve $\gamma\:[a,b]\to\spc{X}$ can be reparametrized by its arc-length.
Indeed, consider the function
\[s(t)=
\length\gamma|_{[a,t]}.
\] 
Evidently, $t\mapsto s(t)$ is monotonic and by \ref{ex:cont-length} it is continuous.
It follows that $t\mapsto s(t)$ is an arc-length parameter of the curve~$\gamma$.

\begin{thm}{Proposition}\label{prop:arc-length-smooth}
If $t\mapsto \gamma(t)$ is a smooth regular curve, 
then its arc-length parametrization is also smooth and regular.
Moreover, the arc-length parameter $s$ of $\gamma$ can be written as an integral
\[s(t)=\int_{t_0}^t |\gamma'(\tau)|\cdot d\tau.\eqlbl{s(t)}\]
\end{thm}

\parit{Proof.} The function $t\mapsto s(t)$ defined by \ref{s(t)} is a smooth increasing function.
Furthermore, by the fundamental theorem of calculus, $s'(t)=|\gamma'(t)|$.
Since $\gamma$ is regular, $s'(t)\ne0$ for any parameter value~$t$.

By the inverse function theorem (\ref{thm:inverse}), the inverse function $s^{-1}(t)$ is also smooth
and $|(\gamma\circ s^{-1})'|\equiv1$.
Therefore, $\gamma\circ s^{-1}$ is a unit-speed reparametrization  of $\gamma$ by~$s$.
By construction, $\gamma\circ s^{-1}$ is smooth, and, since $|(\gamma\circ s^{-1})'|\equiv1$, it is regular.
\qeds

Most of the time we use $s$ for an arc-length parameter of a curve.

\begin{thm}{Exercise}\label{ex:arc-length-helix}
Reparametrize the helix 
\[\gamma_{a,b}(t)=(a\cdot\cos t,a\cdot \sin t, b\cdot t)\]
by its arc-length.
\end{thm}

We will be interested in the properties of curves that are invariant under reparametrizations.
Therefore, we can always assume that any given smooth regular curve comes with an arc-length parametrization.
A nice property of arc-length parametrizations is that they are almost canonical --- these parametrizations differ only by a sign and an additive constant.
For that reason, it is easier to express parametrization-independent quantities.
This observation will be used in the definitions of curvature and torsion.

On the other hand, it is usually impossible to find an explicit arc-length parametrization, which makes it hard to perform calculations;
therefore it is often convenient to use the original parametrization.

\section{Convex curves}

A simple plane curve is called \index{convex!curve}\emph{convex} if it bounds a convex region.
Since the boundary of any region is closed, any convex curve is either closed or open (see Section \ref{sec:proper-curves}).

\begin{thm}{Proposition}\label{prop:convex-curve}
Assume that a convex closed curve $\alpha$ lies inside the domain bounded by a closed simple plane curve $\beta$.
Then
\[\length\alpha\le \length\beta.\]
\end{thm}

To prove Proposition \ref{prop:convex-curve} it is sufficient to show that the perimeter of any polygon inscribed in $\alpha$ is less or equal than the length of $\beta$.
Since any polygon inscribed in $\alpha$ is convex, it is sufficient to prove the following lemma.

\begin{thm}{Lemma}\label{lem:perimeter}
Assume that a convex polygon $P$ lies in a figure $F$ bounded by a simple closed curve.
Then 
\[\perim P\le \perim F,\]
where $\perim F$ denotes the perimeter of~$F$.
\end{thm}

\parit{Proof.} 
A \emph{chord} in $F$ is defined to be a line segment in $F$ with endpoints in its boundary.
Suppose $F'$ is a figure obtained from $F$ by cutting it along a chord and removing one side.
By the triangle inequality, we have
\[\perim F'\le \perim F.\]

\begin{wrapfigure}{o}{24 mm}
\vskip-10mm
\centering
\includegraphics{mppics/pic-7}
\vskip3mm
\end{wrapfigure}

Observe that there is a decreasing sequence of figures 
\[F=F_0\supset F_1\supset\dots\supset F_n=P\]
such that $F_{i+1}$ is obtained from $F_{i}$ by cutting along a chord.
Therefore, 
\begin{align*}
\perim F=\perim F_0&\ge\perim F_1\ge\dots\ge\perim F_n=\perim P.
\end{align*}
\qedsf

\parit{Comment.}
Two other proofs of \ref{lem:perimeter} can be obtained by applying Crofton's formulas (see \ref{ex:convex-croftons}) and the closest-point projection (see Lemma~\ref{lem:closest-point-projection}).  

\begin{thm}{Corollary}\label{cor:convex=>rectifiable}
Any convex closed plane curve is rectifiable.  
\end{thm}

\parit{Proof.}
Any closed curve is bounded.
Indeed, the curve can be described as an image of a loop $\alpha\:[0,1]\to\mathbb{R}^2$, $\alpha(t)=(x(t),y(t))$.
The coordinate functions $t\mapsto x(t)$ and $t\mapsto y(t)$ are continuous functions defined on $[0,1]$.
This implies that the absolute values of both functions are bounded by some constant~$C$.
Therefore, $\alpha$ lies in the square defined by the inequalities $|x|\le C$ and $|y|\le C$.


By Proposition~\ref{prop:convex-curve}, the length of the curve cannot exceed the perimeter of the square; hence the result.
\qeds

Recall that the convex hull of a set $X$ is the smallest convex set that contains $X$; equivalently, the convex hull of $X$ is the intersection of all convex sets containing~$X$.

\begin{thm}{Exercise}\label{ex:convex-hull}
Let $\alpha$ be a simple closed plane curve.
Denote by $K$ the convex hull of $\alpha$; let $\beta$ be the boundary curve of~$K$.
Show that 
\[\length \alpha\ge \length \beta.\]

Try to show that the statement holds for arbitrary closed plane curves $\alpha$, assuming only that $K$ has a nonempty interior.
\end{thm}


\section{Crofton's formulas}
\label{sec:crofton}
\index{Crofton's formula}

For a function $f\: \mathbb{S}^1 \to \mathbb{R}$, we will denote its average value as $\overline{f(\vec u)}$.
For a vector $\vec w$ and a unit vector $\vec u$, we will denote by $\vec w_{\vec u}$ the orthogonal projection of $\vec w$ to the line in the direction of  $\vec u$;
that is,
\[\vec w_{\vec u}=\langle\vec u,\vec w\rangle\cdot\vec u.\] 

\begin{thm}{Theorem}
For any plane curve $\gamma$ we have
\[
\length\gamma
=\tfrac\pi2\cdot \overline{\length\gamma_{\vec u}},
\]
where $\gamma_{\vec u}$ is the curve defined by $\gamma_{\vec u}(t) \df (\gamma (t))_{\vec u}$.
\end{thm}

\parit{Proof.}
Note that the magnitude of any plane vector ${\vec w}$ is proportional to the average magnitude of its projections; that is,
\[|{\vec w}|=k\cdot \overline{|{\vec w}_{\vec u}|}\]
for some $k \in \mathbb{R}$.
The exact value of $k$ can be found by integration\footnote{It is the average value of $|\cos|$.}, but we will find it differently. 
Note that for a smooth plane curve $\gamma\:[a,b]\to\mathbb{R}^2$,
\[\gamma_{\vec u}'(t)=(\gamma'(t))_{\vec u}
\quad\text{and}\quad
|\gamma'_{\vec u}(t)|=|\langle\vec u,\gamma'(t)\rangle|\]
for any $t \in [a,b]$. Then, according to Exercise~\ref{ex:integral-length},
\begin{align*}
\length\gamma
&=\int_a^b|\gamma'(t)|\cdot dt=
\\
&=\int_a^b  k\cdot \overline{|\gamma_{\vec u}'(t)|}\cdot dt=
\\
&=k\cdot \overline{\length\gamma_{\vec u}}.
\end{align*}

Since $k$ is a universal constant, we can compute it by taking $\gamma$ to be the unit circle.
In this case,
\[\length \gamma=2\cdot\pi.\]
Note that for any unit plane vector ${\vec u}$, the curve $\gamma_{\vec u}$ runs back and forth along an interval of length~2.
Hence $\length\gamma_{\vec u}=4$ for any $\vec u$, and 
\[\overline{\length\gamma_{\vec u}} =4.\]
It follows that $2\cdot \pi =k\cdot 4$.
Therefore, Crofton's formula holds for arbitrary smooth curves.

Applying the same argument together with \ref{adex:integral-length}, we get that Crofton's formula holds for arbitrary Lipschitz curve.
Further, since arc length parametrization of any rectifiable curves is Lipschitz, we get Crofton's formula for arbitrary rectifiable curve.

It remains to consider the nonrectifiable case;
we have to show that 
\[\length\gamma=\infty
\quad\Longrightarrow\quad
\overline{\length\gamma_{\vec u}}=\infty.
\]

Observe that from the definition of length, we get
\[\length\gamma_{\vec u}+\length\gamma_{\vec v}\ge \length\gamma\]
for any plane curve $\gamma$ and any pair $(\vec u , \vec v )$ of orthonormal vectors in $\mathbb{R}^2$.
Therefore, if $\gamma$ has infinite length, then the average of lengths of $\gamma_{\vec u}$ is infinite as well.
\qeds

\begin{thm}{Exercise}\label{ex:convex-croftons}
Suppose a closed simple plane curve $\gamma$ bounds a figure~$F$.
Let $s$ be the average length of the projections of $F$ to lines.
Show that $\length\gamma\ge \pi \cdot s$.
Moreover, equality holds if and only if $\gamma$ is convex.

Use this statement to give another solution to Exercise~\ref{ex:convex-hull}.
\end{thm}

The following exercise gives analogous formulas in the Euclidean space.

As before, we denote by $\vec w_{\vec u}$ the orthogonal projection of $\vec w$ to the line passing thru the origin with direction $\vec u$.
Further, let us denote by $\vec w_{\vec u}^\bot$ the projection of $\vec w$ to the plane orthogonal to $\vec u$;
that is,
\[\vec w_\vec u^\bot=\vec w - \vec w_{\vec u}.\]

We will use the notation 
$\overline{f(\vec u)}$ for the average value
of a function $f$ defined on $\mathbb{S}^2$.

\begin{thm}{Advanced exercise}\label{adex:more-croftons}
Show that the length of a space curve is proportional to 
\begin{subthm}{}
the average length of its projections to all lines; that is,
\[\length\gamma=k\cdot\overline{\length\gamma_{\vec u}}\]
for some $k \in \mathbb{R}$.
\end{subthm}
\begin{subthm}{}the average length of its projections to all planes; that is,
\[\length\gamma=k\cdot\overline{\length\gamma_{\vec u}^\bot}\]
for some $k \in \mathbb{R}$.
\end{subthm}
Find the value of $k$ in each case.
\end{thm}

\section{Length metric}\label{sec:Length metric}

Let $\spc{X}$ be a metric space.
Given two points $x,y$ in $\spc{X}$, denote by $\ell(x,y)$ the greatest lower bound of lengths of all paths connecting $x$ to $y$; if there is no such path, then $\ell(x,y)=\infty$.

It is straightforward to see that the function $\ell$ satisfies all the axioms of a metric except it might take infinite values.
Therefore, if any two points in $\spc{X}$ can be connected by a rectifiable curve, then $\ell$ defines a new metric on $\spc{X}$;
in this case, $\ell$ is called the \index{induced length-metric}\emph{induced length-metric}.

Evidently, $\ell(x,y)\ge \dist{x}{y}{}$ for any pair of points $x,y\in \spc{X}$.
If the equality holds for all pairs, then the metric $\dist{{*}}{{*}}{}$ is said to be a \index{length-metric}\emph{length-metric}, and the corresponding metric space is called a \index{}\emph{length-metric space}.

Most of the time we consider length-metric spaces.
In particular, the Euclidean space is a length-metric space.
A subspace $A$ of a length-metric space $\spc{X}$ is not necessarily a length-metric space;
the induced length distance between points $x$ and $y$ in the subspace $A$ will be denoted as $\dist{x}{y}A$;
that is, $\dist{x}{y}A$ is the greatest lower bound of the lengths of paths in $A$ from $x$ to~$y$.\index{10aaa@$\lvert x-y\rvert_A$}

\begin{thm}{Exercise}\label{ex:intrinsic-convex}
Let $A\subset \mathbb{R}^3$ be a closed subset.
Show that $A$ is convex if and only if
\[\dist{x}{y}A=\dist{x}{y}{\mathbb{R}^3}\]
for any $x,y\in A$
\end{thm}


\section{Spherical curves}

Let us denote by $\mathbb{S}^2$ the unit sphere in the space; that is,
\[\mathbb{S}^2=\set{(x,y,z)\in\mathbb{R}^3}{x^2+y^2+z^2=1}.\]
A space curve $\gamma$ is called \index{spherical!curve}\emph{spherical} if it runs in the unit sphere;
that is, $|\gamma(t)|=1$, or equivalently, $\gamma(t)\in\mathbb{S}^2$  for any~$t$.

Recall that $\measuredangle(u,v)$ denotes the angle between two vectors $u$ and~$v$.

\begin{thm}{Observation}\label{obs:S2-length}
For any $u,v\in \mathbb{S}^2$, we have
\[\dist{u}{v}{\mathbb{S}^2}=\measuredangle(u,v)\]

\end{thm}

\parit{Proof.}
The short arc $\gamma$ of a great circle  from $u$ to $v$ in $\mathbb{S}^2$ has length $\measuredangle(u,v)$.
Therefore,
\[\dist{u}{v}{\mathbb{S}^2}\le\measuredangle(u,v).\]

It remains to prove the opposite inequality.
In other words, we need to show that given a polygonal line $\beta=p_0\dots p_n$ inscribed in $\gamma$ there is a polygonal line
$\beta_1=q_0\dots q_n$ inscribed in any given spherical path $\gamma_1$ connecting $u$ to $v$ such that 
\[\length\beta_1\ge \length \beta.\eqlbl{eq:length beta=<length beta}\]

Define $q_i$ as the first point on $\gamma_1$ such that $|u-p_i|=|u-q_i|$, but set $q_n=v$.
Clearly, $\beta_1$ is inscribed in $\gamma_1$, and, according to the triangle inequality for angles (\ref{thm:spherical-triangle-inq}), we have that 
\[ \measuredangle(q_{i-1},q_i)\ge\measuredangle(p_{i-1},p_i).\]
By the angle monotonicity (\ref{lem:angle-monotonicity}),
\[|q_{i-1}-q_i|\ge|p_{i-1}-p_i|\]
and \ref{eq:length beta=<length beta} follows.
\qeds

\begin{thm}{Hemisphere lemma}\label{lem:hemisphere}
Any closed spherical curve of length less than $2\cdot \pi$ lies in an open hemisphere. 
\end{thm}

This lemma will play a key role in the proof of Fenchel's theorem (\ref{thm:fenchel}).
The following proof is due to Stephanie Alexander.
It is not as simple as one may think --- try to prove it  before reading further.

\parit{Proof.}
Let $\gamma$ be a closed curve in $\mathbb{S}^2$ of length $2\cdot\ell$.
Suppose $\ell<\pi$.


\begin{wrapfigure}[8]{o}{35mm}
\vskip-0mm
\centering
\includegraphics{mppics/pic-52}
\end{wrapfigure}

Let us subdivide $\gamma$ into two arcs $\gamma_1$ and $\gamma_2$ of length $\ell$;
denote their endpoints by $p$ and~$q$. 
Note that 
\begin{align*}
\measuredangle(p,q)&\le \length \gamma_1=
\\
&= \ell<
\\
&<\pi.
\end{align*}

Denote by $z$ be the midpoint between $p$ and $q$ in $\mathbb{S}^2$;
that is, $z$ is the midpoint of the short arc of a great circle from $p$ to $q$ in $\mathbb{S}^2$. 
We claim that $\gamma$ lies in the open northern hemisphere with the north pole at~$z$.
If not, $\gamma$ intersects the equator in a point~$r$.
Without loss of generality, we may assume that $r$ lies on~$\gamma_1$. 

Rotate the arc $\gamma_1$ by the angle $\pi$ around the line thru $z$ and the center of the sphere.
The obtained arc $\gamma_1^{*}$ together with $\gamma_1$ forms a closed curve of length $2\cdot \ell$ passing thru $r$ and its antipodal point $r^{*}$.
Therefore,
\[\tfrac12\cdot\length \gamma=\ell\ge \measuredangle(r,r^{*})=\pi,\] 
a contradiction.
\qeds

\begin{thm}{Exercise}\label{ex:antipodal}
Describe a simple closed spherical curve that does not pass thru a pair of antipodal points and does not lie in any open  hemisphere.
\end{thm}


\begin{thm}{Exercise}\label{ex:bisection-of-S2}
Suppose a closed simple spherical curve $\gamma$ divides $\mathbb{S}^2$ into two regions of equal area.
Show that 
\[\length\gamma\ge2\cdot\pi.\]
\end{thm}


\begin{thm}{Exercise}\label{ex:flaw}
Find a flaw in the solution of the following problem.
Come up with a correct argument.
\end{thm}

 
\parbf{Problem.}
Suppose a closed plane curve $\gamma$ has length at most 4.
Show that $\gamma$ lies in a unit disc.

\parit{Wrong solution.}
Note that it is sufficient to show that the diameter of $\gamma$ is at most 2;
that is, 
\[|p-q|\le 2\eqlbl{eq:|pq|=<2}\]
for any two points $p$ and $q$ on~$\gamma$.

The length of $\gamma$ cannot be smaller than the closed inscribed polygonal line which goes from $p$ to $q$ and back to~$p$.
Therefore, 
\[2\cdot |p-q|\le\length \gamma\le 4;\]
whence \ref{eq:|pq|=<2} follows.
\qedsf

\begin{thm}{Advanced exercises} \label{adex:crofton}
Given unit vectors ${\vec u},{\vec w}\in\mathbb{S}^2$, denote by ${\vec w}^*_{\vec u}$ the closest point to ${\vec w}$ on the equator with the pole at ${\vec u}$;
in other words, if ${\vec w}^\perp_{\vec u}$ is the projection of ${\vec w}$ to the plane perpendicular to ${\vec u}$, then ${\vec w}^*_{\vec u}$ is the unit vector in the direction of ${\vec w}^\perp$.
The vector ${\vec w}^*_{\vec u}$ is defined if ${\vec w}\ne\pm {\vec u}$.

\begin{subthm}{adex:crofton:crofton}
Show that for any spherical curve $\gamma$ we have
\[\length\gamma=\overline{\length\gamma^*_{\vec u}},\]
where $\overline{\length\gamma^*_{\vec u}}$ denotes the average length of $\gamma^*_{\vec u}$ with ${\vec u}$ varying in~$\mathbb{S}^2$.
(This is a spherical analog of Crofton's formula.)
\end{subthm}

\begin{subthm}{adex:crofton:hemisphere}
Use \ref{SHORT.adex:crofton:crofton} to give another proof of the hemisphere lemma (\ref{lem:hemisphere}). 
\end{subthm}
 
\end{thm}

Spherical Crofton's formula can be rewritten the following way:
\[\length\gamma=\overline n\cdot \pi,\]
where $\overline n$ denotes the average number of intersection points of $\gamma$ with equators.
The equivalence can be proved using Levi's monotone convergence theorem.

