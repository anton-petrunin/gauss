\chapter{Shortest paths}
\label{chap:shortest}

\section{Intrinsic geometry}

We begin to study the \index{intrinsic geometry}\emph{intrinsic geometry} of surfaces.
A property is called \index{intrinsic property}\emph{intrinsic} if it can be checked by measuring things inside the surface; for example, lengths of curves or angles between the curves that lie in the surface.
Otherwise, if its definition essentially uses the ambient space, then it is called \index{extrinsic property}\emph{extrinsic}.

For instance, the mean curvature, as well as the Gauss curvature, are defined via principal curvatures, which are extrinsic.
Later (\ref{thm:remarkable}) it will be shown that {}\emph{remarkably} the Gauss curvature is actually intrinsic --- it can be calculated via measurements inside the surface.
The mean curvature is not intrinsic; for example, the intrinsic geometry of the $(x,y)$-plane is not distinguishable from the intrinsic geometry of the graph $z\z=(x+y)^2$.
However, while the mean curvature of the former vanishes at all points, the mean curvature of the latter does not vanish, say at $p=(0,0,1)$.  

The following exercise should help you get in the right mood;
it might look like a tedious problem in calculus, but actually, it is an easy problem in geometry.
We learned this problem from Joel Fine, who attributed it to Frederic Bourgeois \cite{fine}.

\begin{wrapfigure}[6]{r}{33 mm}
\vskip-6mm
\centering
\includegraphics{mppics/pic-77}
\vskip-0mm
\end{wrapfigure}

\begin{thm}{Exercise}\label{ex:lasso}
А cowboy stands at the bottom of a frictionless ice-mountain formed by a cone with a circular base with  angle  of inclination~$\theta$.
He wants to climb the mountain;
he throws up his lasso which slips neatly over the tip of the cone, pulls it tight and starts to climb.

What is the critical angle $\theta$ at which the cowboy can no longer climb the ice-mountain?
\end{thm}

\section{Definition}

Let $p$ and $q$ be two points on a surface $\Sigma$.
Recall that $\dist{p}{q}\Sigma$ denotes the induced length distance from $p$ to $q$;
that is, the greatest lower bound of the lengths of paths in $\Sigma$ from $p$ to $q$.

Note that if $\Sigma$ is smooth, then any two points in $\Sigma$ can be joined by a piecewise smooth path.
Since any such path is rectifiable, the value $\dist{p}{q}\Sigma$ is finite for any pair $p,q\in\Sigma$.

A path $\gamma$ from $p$ to $q$ in $\Sigma$ that minimizes the length is called a \index{shortest path}\emph{shortest path} from $p$ to $q$.

The image of a shortest path between $p$ and $q$ in $\Sigma$ is usually denoted by $[p,q]$ or by $[p,q]_\Sigma$.\index{10aac@$[p,q]$, $[p,q]_\Sigma$}
In general, there might be no shortest path between two given points on a surface, and there might be many of them;
this is shown in the following two examples.

Usually, if we write $[p,q]_\Sigma$, we are assuming that such a shortest path exists, and we have chosen one of them.

{

\begin{wrapfigure}{r}{28 mm}
\vskip-6mm
\centering
\includegraphics{asy/sphere}
\end{wrapfigure}

\parbf{Nonuniqueness.} There are plenty of shortest paths between the poles of a round sphere --- each meridian is a shortest path.

\parbf{Nonexistence.} Let $\Sigma$ be the $(x,y)$-plane with the origin removed.
Consider two points $p=(1,0,0)$ and $q=(-1,0,0)$ in $\Sigma$.

}

\begin{thm}{Claim}
There is no shortest path from $p$ to $q$ in $\Sigma$.
\end{thm}

\begin{wrapfigure}{r}{28 mm}
\vskip-4mm
\centering
\includegraphics{mppics/pic-79}
\end{wrapfigure}

\parit{Proof.}
Note that $\dist{p}{q}\Sigma=2$. 
Indeed, given $\epsilon\z>0$, consider the point $s_\epsilon=(0,\epsilon,0)$.
Observe that the polygonal path $ps_\epsilon q$ lies in $\Sigma$, and its length $2\cdot\sqrt{1+\epsilon^2}$ approaches $2$ as $\epsilon\to0$.
It follows that $\dist{p}{q}\Sigma\le 2$.
Since $\dist{p}{q}\Sigma\z\ge \dist{p}{q}{\mathbb{R}^3}=2$, we get $\dist{p}{q}\Sigma=2$.


It follows that a shortest path from $p$ to $q$, if it exists, must have length 2.
By the triangle inequality, any curve of length 2 from $p$ to $q$ must run along the line segment $[p,q]$;
in particular, it must pass thru the origin.
Since the origin does not lie in $\Sigma$, there is no shortest from $p$ to $q$ in $\Sigma$ 
\qeds

\begin{thm}{Proposition}\label{prop:shortest-paths-exist}
Any two points in a proper smooth surface can be joined by a shortest path. 
\end{thm}

\parit{Proof.}
Choose two points $p$ and $q$ in a proper smooth surface $\Sigma$.
Set $\ell=\dist{p}{q}\Sigma$.

By the definition of induced length-metric (Section~\ref{sec:Length metric}),
there is a sequence of paths $\gamma_n$ from $p$ to $q$ in $\Sigma$ such that
\[\length\gamma_n\to \ell\quad\text{as}\quad n\to \infty.\]

Without loss of generality, we may assume that $\length\gamma_n<\ell+1$ for any $n$, and each $\gamma_n$ is parametrized proportional to its arc-length.
In particular, each path $\gamma_n\:[0,1]\to\Sigma$ is $(\ell+1)$-Lipschitz; 
that is,
\[|\gamma(t_0)-\gamma(t_1)|\le (\ell+1)\cdot|t_0-t_1|\]
for any $t_0,t_1\in[0,1]$.

Note that the image of $\gamma_n$ lies in the closed ball $\bar B[p,\ell+1]$ for any $n$.
It follows that the coordinate functions of $\gamma_n$ are uniformly equicontinuous and uniformly bounded.
By the Arzel\'{a}--Ascoli theorem (\ref{lem:equicontinuous})
 there is a convergent subsequence of $\gamma_n$, and its limit, say $\gamma_\infty\:[0,1]\to\mathbb{R}^3$, is continuous;
that is, $\gamma_\infty$ is a path.
Evidently, $\gamma_\infty$ runs from $p$ to $q$;
in particular
\[\length\gamma_\infty\ge \ell.\]
Since $\Sigma$ is a closed set, $\gamma_\infty$ lies in $\Sigma$.
Finally, since length is semicontinuous (\ref{thm:length-semicont}), we get that
\[\length\gamma_\infty\le \ell.\]
Therefore, $\length\gamma_\infty= \ell$ or, equivalently, $\gamma_\infty$ is a shortest path from $p$ to $q$.
\qeds

\section{Closest-point projection}\label{sec:closest-point-projection}

\begin{thm}{Lemma}\label{lem:closest-point-projection}
Let $R$ be a closed convex set in $\mathbb{R}^3$.
Then for every point $p\in\mathbb{R}^3$ there is a unique point $\bar p\in R$ that minimizes the distance to $R$;
that is, $|p-\bar p|\le |p-x|$ for any point $x\in R$.

Moreover, the map $p\mapsto \bar p$ is short;
that is,
\[|p-q|\ge|\bar p-\bar q| \eqlbl{eq:short-cpp}\]
for any pair of points $p,q\in \mathbb{R}^3$.
\end{thm}

The map $p\mapsto \bar p$ is called the \label{closest-point projection}\index{closest-point projection}\emph{closest-point projection};
it maps the Euclidean space to $R$.
Note that if $p\in R$, then $\bar p=p$.

\parit{Proof.}
Fix a point $p$, and set 
\[\ell=\inf\set{|p-x|}{x\in R}.\]
Choose a sequence $x_n\in R$ such that $|p-x_n|\to \ell$ as $n\to\infty$.

Without loss of generality, we can assume that all the points $x_n$ lie in a ball of radius $\ell+1$ centered at~$p$.
Therefore, we can pass to a \index{partial limit}\emph{partial limit} $\bar p$ of $x_n$; that is, $\bar p$ is a limit of a subsequence of $x_n$.
Since $R$ is closed, $\bar p\in R$.
By construction 
\begin{align*}
|p-\bar p|&=\lim_{n\to\infty}|p-x_n|=\ell.
\end{align*}
Hence, the existence follows.

{

\begin{wrapfigure}{r}{22 mm}
\vskip-0mm
\centering
\includegraphics{mppics/pic-40}
\vskip-0mm
\end{wrapfigure}

Assume there are two distinct points $\bar p, \bar p'\in R$ that minimize the distance to $p$.
Since $R$ is convex, their midpoint $m=\tfrac12\cdot (\bar p+\bar p')$ lies in~$R$.
Note that $|p-\bar p|\z=|p-\bar p'|=\ell$;
that is, the triangle $[p\bar p\bar p']$ is isosceles, and therefore the triangle $[p\bar p m]$ is right with the right angle at~$m$.
Since a leg of a right triangle is shorter than its hypotenuse, we have $|p-m|<\ell$, a contradiction. 

It remains to prove \ref{eq:short-cpp}.
We can assume that $\bar p\ne\bar q$; otherwise, there is nothing to prove.

}

Note that if $\measuredangle \hinge{\bar p}{p}{\bar q}< \tfrac\pi2$, then $\dist{p}{x}{}\z<\dist{p}{\bar p}{}$ for some point $x\in [\bar p,\bar q]$.
Since $[\bar p,\bar q]\subset K$,
the latter is impossible.

\begin{wrapfigure}{l}{37 mm}
\vskip-0mm
\centering
\includegraphics{mppics/pic-41}
\vskip-0mm
\end{wrapfigure}

Therefore, $p=\bar p$ or $\measuredangle \hinge{\bar p}{p}{\bar q}\ge \tfrac\pi2$.
In both cases, the orthogonal projection of $p$ to the line $\bar p\bar q$ lies behind $\bar p$, or coincides with $\bar p$.
The same way we show that the orthogonal projection of $q$ to the line $\bar p\bar q$ lies behind $\bar q$, or coincides with $\bar q$.
It implies that the orthogonal projection of the line segment $[p,q]$ to the line $\bar p\bar q$ contains the line segment $[\bar p,\bar q]$.
In particular,
\[|p-q|\ge |\bar p-\bar q|.\]
\qedsf

\begin{thm}{Corollary}\label{cor:shorts+convex}
Assume a surface $\Sigma$ bounds a closed convex region $R$, and $p,q\in \Sigma$.
Denote by $W$ the outer closed region of $\Sigma$;
in other words, $W$ is the union of $\Sigma$ and the complement of $R$.
Then 
\[\length\gamma\ge \dist{p}{q}\Sigma\]
for any path $\gamma$ in $W$ from $p$ to $q$.
Moreover, if  $\gamma$ does not lie in $\Sigma$, then the inequality is strict.
\end{thm}

\parit{Proof.}
The first part of the corollary follows from the lemma and the definition of length.
Indeed, consider the closest-point projection $\bar\gamma$ of~$\gamma$.
Note that $\bar\gamma$ lies in $\Sigma$ and connects $p$ to $q$ therefore 
\[\length\bar\gamma\ge \dist{p}{q}\Sigma.\]

To prove the first statement, it remains to show that 
\[\length\gamma\ge\length\bar\gamma.\eqlbl{bar-gamma=<gamma}\]

Consider a polygonal line $\bar p_0\dots \bar p_n$ inscribed in $\bar\gamma$.
Let $p_0\dots p_n$ be the corresponding polygonal line inscribed in $\gamma$;
that is $p_i=\gamma(t_i)$ if $\bar p_i=\bar\gamma(t_i)$.
By \ref{lem:closest-point-projection} $|p_i-p_{i-1}|\z\ge|\bar p_i-\bar p_{i-1}|$ for any $i$.
Therefore,
\[\length p_0\dots p_n\ge \length \bar p_0\dots \bar p_n.\]
Taking the least upper bound of each side of the inequality for all inscribed polygonal lines $p_0\dots p_n$ in $\gamma$, we get \ref{bar-gamma=<gamma}.\

\begin{wrapfigure}{o}{37 mm}
\vskip-0mm
\centering
\includegraphics{mppics/pic-82}
\vskip-0mm
\end{wrapfigure}

It remains to prove the second statement.
Suppose there is a point $w\z=\gamma(t_1)\notin\Sigma$;
note that $w\notin R$.
By the separation lemma (\ref{lem:separation}), there is a plane $\Pi$ that cuts $w$ from $\Sigma$.
The curve $\gamma$ must intersect $\Pi$ at two points: one point before $t_1$ and one after.
Let $a=\gamma(t_0)$ and $b=\gamma(t_2)$ be these points.
Note that the arc of $\gamma$ from $a$ to $b$ is strictly longer that $|a-b|$;
indeed its length is at least $|a-w|+|w-b|$, and $|a-w|+|w-b|>|a-b|$ since $w\notin[a,b]$.

Remove from $\gamma$ the arc from $a$ to $b$, and replace it with the line segment $[a,b]$;
denote the obtained curve by $\gamma_1$. 
From above, we have that
\[\length\gamma>\length \gamma_1\]
Note that $\gamma_1$ runs in $W$.
Therefore, by the first part of the corollary, we have
\[\length \gamma_1\ge \dist{p}{q}\Sigma.\]
Whence the second statement follows.
\qeds

\begin{thm}{Exercise}\label{ex:length-dist-conv}
Suppose $\Sigma$ is a proper smooth surface with positive Gauss curvature, and $\Norm$ is the unit normal field on $\Sigma$.
Show that for any two points $p,q\in \Sigma$ we have the following inequality:
\[\dist{p}{q}\Sigma\le 2\cdot \frac{|p-q|}{|\Norm(p)+\Norm(q)|}.\]

\end{thm}


\begin{wrapfigure}{r}{27 mm}
\vskip-10mm
\centering
\includegraphics{mppics/pic-240}
\end{wrapfigure}

\begin{thm}{Exercise}\label{ex:hat-convex}
Suppose $\Sigma$ is a closed smooth surface that bounds a convex region $R$ 
in $\mathbb{R}^3$
and $\Pi$ is a plane that cuts a hat $\Delta$ from $\Sigma$.
Assume that the reflection of the interior of $\Delta$ across $\Pi$ lies in the interior of $R$.
Show that $\Delta$ is \index{convex!set}\emph{convex} with respect to the intrinsic metric  of $\Sigma$;
that is, 
if both ends of a shortest path in $\Sigma$ 
lie in $\Delta$,
then the entire path lies in $\Delta$.
\end{thm}


Let us define the \index{intrinsic diameter}\emph{intrinsic diameter} of a closed surface $\Sigma$ as the least upper bound of the lengths of shortest paths in the surface.

\begin{thm}{Exercise}\label{ex:intrinsic-diameter}
Assume a closed smooth surface $\Sigma$ with positive Gauss curvature lies in a unit ball $B$.

\begin{subthm}{} Show that the intrinsic diameter of $\Sigma$ cannot exceed $\pi$.
\end{subthm}

\begin{subthm}{}
Show that the area of $\Sigma$ cannot exceed $4\cdot \pi$.
\end{subthm}

\end{thm}
