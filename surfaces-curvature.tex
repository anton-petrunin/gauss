\chapter{Curvatures}
\label{chap:surface-curvature}

In the previous chapter, we learned that the tangent plane $\T_p$ to any surface $\Sigma$ has first-order contact with it at $p$.
This means that up to first order, all surfaces locally look the same.
This is not the case for second-order approximations, which are determined by the principal curvatures of the surface.

\section{Tangent-normal coordinates} \label{sec:lmn}
\index{tangent-normal coordinates}

Fix a point $p$ in a smooth oriented surface~$\Sigma$.
Consider a coordinate system $(x,y,z)$ with origin at $p$ such that the $(x,y)$-plane coincides with $\T_p$, and the $z$-axis points in the direction of the normal vector $\Norm(p)$. Such coordinates are called \emph{tangent-normal coordinates}. 
By \ref{ex:vertical-tangent}, we can present $\Sigma$ locally at $p$ as a graph $z=f(x,y)$ of a smooth function. 
Note that 
\begin{align*}
f(0,0)&=0,
&
f_x(0,0)&=0,
&
f_y(0,0)&=0.
\end{align*}
The first equality holds since $p=(0,0,0)$ lies on the graph, and the other two mean that the tangent plane at $p$ is horizontal.


\begin{wrapfigure}[7]{o}{42 mm}
\vskip-4mm
\centering
\includegraphics{asy/paraboloid}
\vskip-3mm
\end{wrapfigure}

Set 
\begin{align*}
\ell&=f_{xx}(0,0),
\\
m&=f_{xy}(0,0)=f_{yx}(0,0),
\\
n&=f_{yy}(0,0).
\end{align*}
The \textit{Taylor series} 
for $f$ at $(0,0)$ up to the second-order term can be then written as
\[f(x,y)=\tfrac12(\ell\cdot x^2+2\cdot m\cdot x\cdot y+n\cdot y^2)+o(x^2+y^2).\]
Note that values $\ell$, $m$, and $n$ are completely determined by this equation.\index{10lmn@$\ell$, $m$, $n$ (Hessian-matrix components)}
The so-called \index{osculating!paraboloid}\emph{osculating paraboloid}
\[z=\tfrac12\cdot(\ell\cdot x^2+2\cdot m\cdot x\cdot y+n\cdot y^2)\]
has \index{order of contact}\emph{second-order contact} with $\Sigma$ at~$p$.

Note that 
\[\ell\cdot x^2+2\cdot m\cdot x\cdot y+n\cdot y^2=\langle M_p\cdot (\begin{smallmatrix}
x\\y
\end{smallmatrix}), (\begin{smallmatrix}
x\\y
\end{smallmatrix})\rangle,\]
where $M_p$ is the so-called \index{Hessian matrix}\emph{Hessian matrix} of $f$ at $(0,0)$,\index{10m@$M_p$ (Hessian matrix)}
\[M_p=\begin{pmatrix}
 \ell
 &m
 \\
 m
 &n
 \end{pmatrix}.
\eqlbl{eq:hessian}
\]


\section{Principal curvatures}\label{sec:Principal curvatures}

Note that tangent-normal coordinates give an almost canonical coordinate system in a neighborhood of $p$;
it is unique up to a rotation of the $(x,y)$-plane.
Rotating the $(x,y)$-plane results in rewriting 
the matrix $M_p$ in the new basis.

Since the Hessian matrix $M_p$ is symmetric, by the spectral theorem (\ref{thm:spectral}) it is diagonalizable by orthogonal matrices.
That is, by rotating the $(x,y)$-plane we can assume that $m=0$ in \ref{eq:hessian}.
In this case,
\[M_p=\begin{pmatrix}
 k_1
 &0
 \\
 0
 &k_2
 \end{pmatrix},
\]
the diagonal components $k_1$ and $k_2$ of $M_p$ are called the \index{principal curvatures and directions}\emph{principal curvatures} of $\Sigma$ at $p$;\index{10k@$k_1$, $k_2$ (principal curvatures)}
they might also be denoted as $k_1(p)$ and $k_2(p)$, or $k_1(p)_\Sigma$ and $k_2(p)_\Sigma$;
if we need to emphasize that we compute them at the point $p$ for the surface~$\Sigma$.
We will assume 
\[k_1\le k_2\]
unless indicated otherwise.


Note that if $z=f(x,y)$ is a local graph representation of $\Sigma$ in these coordinates, then 
\[f(x,y)=\tfrac12\cdot(k_1\cdot x^2+k_2\cdot y^2)+o(x^2+y^2).\]

The principal curvatures can be also defined as the eigenvalues of the Hessian matrix $M_p$.
The eigendirections of $M_p$ are called the {}\emph{principal directions} of $\Sigma$ at~$p$.
Note that if $k_1(p)\ne k_2(p)$, then $p$ has exactly two principal directions, which are perpendicular to each other;
if $k_1(p)\z= k_2(p)$, then all tangent directions at $p$ are principal.

Note that if we reverse the orientation of $\Sigma$, then the principal curvatures switch their signs and indexes; that is, $k_1$ becomes $-k_2$ and $k_2$ becomes $-k_1$.

A smooth curve on a surface $\Sigma$ that always runs in the principal directions is called a \index{line of curvature}\emph{line of curvature} of~$\Sigma$.
If $k_1(p)\ne k_2(p)$, then there is a chart $(u,v)\mapsto s(u,v)$ at $p$ with coordinate lines formed by lines of curvature (it follows from \ref{ex:lin-ind-chart}).
Since principal directions are orthogonal we have $s_u\perp s_v$.

\begin{thm}{Exercise}\label{ex:line-of-curvature}
Assume a smooth surface $\Sigma$ is mirror-symmetric with respect to a plane $\Pi$.
Suppose $\Sigma$ and $\Pi$ intersect along a smooth curve~$\gamma$.
Show that $\gamma$ is a line of curvature of~$\Sigma$.
\end{thm}

\section{More curvatures}\label{sec:More curvatures}

Let $p$ be a point on an oriented smooth surface~$\Sigma$.
Recall that $k_1(p)$ and $k_2(p)$ denote the principal curvatures at~$p$.

The product 
\[K(p)=k_1(p)\cdot k_2(p)\]
is called the \index{10k@$K$ (Gauss curvature)}\index{Gauss curvature}\emph{Gauss curvature} at~$p$.
We may denote it by $K$, $K(p)$, or $K(p)_\Sigma$ if we need to specify the point $p$ and the surface~$\Sigma$.

Since the determinant is equal to the product of the eigenvalues, we get
\[K=\ell\cdot n-m^2,\]
where 
$M_p\z=
(\begin{smallmatrix}
\ell&m
\\
m&n
\end{smallmatrix}
)
$ is the Hessian matrix.

Note that reversing the orientation of $\Sigma$ does not change the Gauss curvature.
In particular, the Gauss curvature is well-defined for nonoriented surfaces.

\begin{thm}{Exercise}\label{ex:gauss+orientable}
Show that any surface with positive Gauss curvature is orientable. 
\end{thm}

The sum \index{10h@$H$ (mean curvature)}
\[H(p)=k_1(p)+ k_2(p)\] 
is called the \index{curvature}\index{mean curvature}\emph{mean curvature}%
\footnote{Some authors define it as $\tfrac12\cdot(k_1(p)+ k_2(p))$ --- the mean value of the principal curvatures. It suits the name better, but it is not as convenient when it comes to computations.}
at~$p$.
We may also denote it by $H(p)_\Sigma$.
The mean curvature can be also interpreted as the trace of the Hessian matrix $M_p\z=
(\begin{smallmatrix}
\ell&m
\\
m&n
\end{smallmatrix}
)$;
that is,
\[H=\ell+n\] 

A surface with vanishing mean curvature is called \index{minimal surface}\emph{minimal}.

Note that reversing the orientation of $\Sigma$ changes the sign of the mean curvature.

\begin{thm}{Exercise}\label{ex:re-scale-surface-curvature}
Let $\Sigma$ be an oriented surface, and let $\Sigma_{\lambda}$ be a scaled copy of $\Sigma$ with factor $\lambda > 0$; that is, $\Sigma_{\lambda}$ consists of the points $\lambda \cdot x$ with $x \in \Sigma$. Show that
\[K(\lambda\cdot p)_{\Sigma_{\lambda}}
= \tfrac{1}{\lambda^2}\cdot K(p)_{\Sigma}
\quad\text{and}\quad
H(\lambda \cdot p)_{\Sigma_{\lambda}} = \tfrac1\lambda\cdot H(p)_{\Sigma}\]
for any $p\in \Sigma$.  
\end{thm}

\section{Shape operator}

In the following definitions, we use the notion of directional derivative and differential defined in \ref{sec:dirder} and \ref{sec:differential}.

Let $p$ be a point on a smooth surface $\Sigma$ with orientation defined by the unit normal field $\Norm$.
Given $\vec w\in \T_p$,
the \index{shape operator}\emph{shape operator} of $\vec w$ is defined by
\[\Shape_p\vec w=-D_{\vec w}\Norm.\]
Equivalently, the shape operator can be defined by
\[\Shape=-d\Norm,\eqlbl{eq:shape=-L}\] 
where $d\Norm$ denotes the differential of the spherical map $\Norm\:\Sigma\to\mathbb{S}^2$; that is, $d_p\Norm(\vec v)=(D_{\vec v}\Norm)(p)$.

Recall that $d_p\Norm$ is a linear map $\T_p\Sigma\to \T_{\Norm(p)}\mathbb{S}^2$.
Note that $\T_p\Sigma$ coincides with $\T_{\Norm(p)}\mathbb{S}^2$ --- both of them are normal subspaces to $\Norm(p)$.
Therefore, $\Shape_p$ is indeed a linear operator $\T_p\to \T_p$ (the latter also will follow from \ref{thm:shape-chart}).

For a point $p\in \Sigma$, the shape operator of a tangent vector $\vec w\in \T_p$ will be denoted by $\Shape\vec w$ if it is clear from the context which base point $p$ and which surface we are working with;
otherwise, we may use the notations 
\[\Shape_p(\vec w)\quad\text{or}\quad \Shape_p(\vec w)_\Sigma\]
as the situation requires.%
\footnote{
The following bilinear forms on a tangent plane  
\begin{align*}
\mathrm{I}(\vec v,\vec w)&=\langle\vec v,\vec w\rangle,
&
\mathrm{II}(\vec v,\vec w)&=\langle\Shape\vec v,\vec w\rangle,
&
\mathrm{III}(\vec v,\vec w)&=\langle\Shape\vec v,\Shape\vec w\rangle
\end{align*}
are called the \index{fundamental form}\emph{first, second, and third fundamental forms}, respectively.
These forms were introduced before the shape operator, but we will not touch them.
}

\begin{thm}{Theorem}\label{thm:shape-chart}
Suppose $(u,v)\mapsto s(u,v)$ is a smooth map to a smooth surface $\Sigma$ with unit normal field $\Norm$.
Then 
\begin{align*}
\langle \Shape(s_u), s_u\rangle 
&=\langle s_{uu},\Norm\rangle,
&
\langle \Shape(s_v), s_u\rangle 
&=\langle s_{uv},\Norm\rangle,
\\
\langle \Shape(s_u), s_v\rangle 
&=\langle s_{uv},\Norm\rangle,
&
\langle \Shape(s_v), s_v\rangle 
&=\langle s_{vv},\Norm\rangle,
\\
\langle \Shape(s_u), \Norm\rangle 
&=0,
&
\langle \Shape(s_v), \Norm\rangle 
&=0
\end{align*}
for any $(u,v)$.

\end{thm}

\parit{Proof.}
We will use the shortcut $\Norm=\Norm(u,v)$ for $\Norm(s(u,v))$,
so 
\[
\begin{aligned}
\Shape(s_u)&=-D_{s_u}\Norm=-\Norm_u,
&
\Shape(s_v)&=-D_{s_v}\Norm=-\Norm_v.
\end{aligned}
\eqlbl{eq:shape=norm_u}
\]

Note that $\Norm$ is a unit vector orthogonal to $s_u$ and $s_v$;
therefore
\begin{align*}
\langle \Norm,s_u\rangle&\equiv0,
&
\langle \Norm,s_v\rangle&\equiv0,
&
\langle \Norm,\Norm\rangle&\equiv1.
\end{align*}
Taking partial derivatives of these two identities we get
\begin{align*}
\langle \Norm_u,s_u\rangle+\langle \Norm,s_{uu}\rangle&=0,
&
\langle \Norm_v,s_u\rangle+\langle \Norm,s_{uv}\rangle&=0,
\\
\langle \Norm_u,s_v\rangle+\langle \Norm,s_{uv}\rangle&=0,
&
\langle \Norm_v,s_v\rangle+\langle \Norm,s_{vv}\rangle&=0,
\\
2\cdot\langle \Norm_u,\Norm\rangle&=0,
&
2\cdot\langle \Norm_v,\Norm\rangle&=0.
\end{align*}
It remains to plug in the expressions from \ref{eq:shape=norm_u}.
\qeds

\begin{thm}{Exercise}\label{ex:self-adjoint}
Show that the shape operator is \index{self-adjoint operator}\emph{self-adjoint}; that is,
\[\langle \Shape\vec u,\vec v\rangle=\langle \vec u,\Shape\vec v\rangle\]
for any $\vec u,\vec v\in\T_p$.
\end{thm}

Let us denote by $\vec i$, $\vec j$ and $\vec k$ the standard basis in the $\mathbb{R}^3$.\index{10i@$\vec i$, $\vec j$, $\vec k$ (standard basis)}
Recall that the components $\ell$, $m$, and $n$ of the Hessian matrix are defined in Section~\ref{sec:lmn}.

\begin{thm}{Corollary}\label{cor:Shape(ij)}
Let $z=f(x,y)$ be a local representation of a smooth surface $\Sigma$ in the tangent-normal coordinates at~$p$.
Suppose that its Hessian matrix at $p$ is $(\begin{smallmatrix}
\ell&m\\ m&n
\end{smallmatrix})$.
Then 
\begin{align*}
\Shape\vec i&=\ell\cdot \vec i+m\cdot \vec j,
&
\Shape\vec j&=m\cdot \vec i+n\cdot\vec j;
\end{align*}
that is, the multiplication by the Hessian matrix at $p$ describes its shape operator.
\end{thm}


This corollary illustrates the close relationship between the curvatures of a surface and its shape operator; the principal curvatures of $\Sigma$ at $p$ are the eigenvalues of $\Shape_p$, the principal directions are the eigendirections of $\Shape_p$, the Gaussian curvature is the determinant of $\Shape_p$, and the mean curvature is the trace of $\Shape_p$.


Since the Hessian matrix is symmetric, the corollary also implies that $\Shape$ is self-adjoint which gives another way to solve \ref{ex:self-adjoint}.

\parit{Proof.}
Note that $s\:(u,v)\mapsto (u,v,f(u,v))$ is a chart of $\Sigma$ that covers~$p$.
Further, note that 
\begin{align*}
s_u(0,0)&=\vec i,
&
s_v(0,0)&=\vec j,
&
\Norm(0,0)&=\vec k,
\\
s_{uu}(0,0)&=\ell\cdot \vec k,
&
s_{uv}(0,0)&=m\cdot \vec k,
&
s_{vv}(0,0)&=n\cdot \vec k.
\end{align*}
It remains to apply \ref{thm:shape-chart}.
\qeds

\begin{thm}{Corollary}\label{cor:intK}
Let $\Sigma$ be a smooth surface with orientation defined by a unit normal field $\Norm$.
Suppose the spherical map $\Norm\:\Sigma\to\mathbb{S}^2$ is injective.
Then 
\[\iint_\Sigma|K|=\area[\Norm(\Sigma)].\]
\end{thm}

\parit{Proof.}
Observe that the tangent planes $\T_p\Sigma=\T_{\Norm(p)}\mathbb{S}^2$ are parallel for any $p\in\Sigma$.
Indeed, both of these planes are perpendicular to $\Norm(p)$. 


Choose an orthonormal basis of $\T_p$ consisting of principal directions,
so the shape operator can be expressed by the matrix 
$(\begin{smallmatrix}
 k_1
 &0
 \\
 0
 &k_2
 \end{smallmatrix})$.

Since $\Shape_p=-d_p\Norm$, \ref{cor:Shape(ij)} implies that
\[\jac_p\Norm=|\det(\begin{smallmatrix}
 k_1
 &0
 \\
 0
 &k_2
 \end{smallmatrix})|=|K(p)|.\]
By the area formula (\ref{prop:surface-integral}), the statement follows.
\qeds


\begin{thm}{Exercise}\label{ex:normal-curvature=const}
Let $\Sigma$ be a smooth surface with orientation defined by a unit normal field $\Norm$.
Suppose the principal curvatures of $\Sigma$ are 1 at all points.

\begin{subthm}{ex:normal-curvature=const:a} Show that $\Shape_p(\vec w)=\vec w$ for any $p\in\Sigma$ and $\vec w\in \T_p\Sigma$.
\end{subthm}

\begin{subthm}{ex:normal-curvature=const:b}Show that $p+\Norm(p)$ is constant; that is, the point $c=p+\Norm(p)$ does not depend on $p\in\Sigma$.
Conclude that $\Sigma$ is a subset of the unit sphere centered at~$c$.
\end{subthm}

\end{thm}

We define the {}\emph{angle} between two oriented surfaces at a point of their intersection $p$ as the angle between their normal vectors at~$p$.
The following exercise is a result by Ferdinand Joachimsthal \cite{joachimsthal} generalized by Ossian Bonnet \cite{bonnet}.

\begin{thm}{Exercise}\label{ex:shape-curvature-line}
Assume two smooth oriented surfaces $\Sigma_1$ and $\Sigma_2$ intersect at constant angle along a smooth curve~$\gamma$.
Show that if $\gamma$ is a curvature line in $\Sigma_1$, then it is also a curvature line in $\Sigma_2$.

Conclude that if a smooth surface $\Sigma$ intersects a plane or sphere along a smooth curve $\gamma$ at a constant angle,
then $\gamma$ is a curvature line of~$\Sigma$.
\end{thm}

\begin{thm}{Exercise}\label{ex:equidistant}
Let $\Sigma$ be a closed smooth surface with orientation defined by a unit normal field $\Norm$.

\begin{subthm}{ex:equidistant:smooth}
Show that if $t$ is sufficiently close to zero, then the set 
\[\Sigma_t=\set{p+t\cdot \Norm(p)}{p\in\Sigma}\] 
is a smooth surface.
\end{subthm}

\begin{subthm}{ex:equidistant:area}
Show that for all $t$ sufficiently close to zero we have
\[\area\Sigma_t=\area\Sigma-t\cdot \iint_\Sigma H+t^2\cdot \iint_\Sigma K,\]
where $H$ and $K$ denote mean and Gauss curvature of~$\Sigma$.
\end{subthm}

\end{thm}

\begin{thm}{Advanced exercise}\label{ex:flat-plane}
Let $\Sigma$ be a smooth oriented surface parametrized by a coordinate rectangle as $(u,v)\mapsto s(u,v)$.
Denote by $\Norm(u,v)$ the normal unit vector at $s(u,v)$.
Set $\vec u=\tfrac{s_u}{|s_u|}$ and $\vec v=\tfrac{s_v}{|s_v|}$.
Suppose $\vec u$ and $\vec v$ are principal directions at each point;
let $0$ and $k=k(u,v)$ be their principal curvatures at $s(u,v)$.
Assume $k$ does not vanish.

\begin{subthm}{ex:flat-plane:orthonormal}
Show that $\Norm(u,v)$, $\vec u(u,v)$, and $\vec v(u,v)$ form an orthonormal frame for any $(u,v)$.
\end{subthm}

\begin{subthm}{ex:flat-plane:depend}
Show that $\Norm(u,v)$, $\vec u(u,v)$, and $\vec v(u,v)$ depend only on $v$.
Conclude that $v$-coordinate lines are line segments.
\end{subthm}

\begin{subthm}{ex:flat-plane:depend-u}
Show that $s_{uu}$ is proportional to $s_u$ at all points.
Use it to show that $|s_u|$ depend only on $u$.
Conclude that if $|s_u(0,t)|=1$ for any $t$, then  $|s_u|=1$ at all points.
\end{subthm}


\begin{subthm}{ex:flat-plane:linear}
Assume that $|s_u|=1$ at all points.
Show that for fixed $v$ the value $\tfrac1{k(u,v)}$ depends linearly on $u$.
\end{subthm}

\end{thm}

This exercise is the key part in the proof of the following theorem:
\textit{Any open surface with vanishing Gauss curvature is \index{cylindrical surface}\emph{cylindrical}};
that is, the surface is swept out by a family of parallel lines.
Exercise \ref{ex:line-cylinder} illustrates another part of this proof.
The mentioned theorem was originally proved by Aleksey Pogorelov \cite[II §3 Thm 2]{pogorelov1956} in much more general settings; it was rediscovered couple of times after that \cite{hartman-nirenberg,massey1962}.
The exercise is based on the proof given by Sergei Ivanov \cite[2$^\text{nd}$ Sem. Lect. 13]{ivanov}.

\chapter{Curves in a surface}

\section{Darboux frame}\label{sec:Darboux}

\begin{wrapfigure}{r}{42 mm}
\vskip-10mm
\centering
\begin{lpic}[t(-0mm),b(0mm),r(0mm),l(0mm)]{asy/paraboloid+curve(1)}
\lbl[ul]{34,14;$\tan$}
\lbl[b]{20,43;$\Norm$}
\lbl[bl]{38,35;$\mu$}
\end{lpic}
\vskip-0mm
\end{wrapfigure}

Suppose $\gamma$ is a smooth curve in a smooth oriented surface~$\Sigma$.
As usual, we denote by $\Norm$ the unit normal field on~$\Sigma$.

Without loss of generality, we may assume that $\gamma$ is unit-speed;
in this case, $\tan(s)=\gamma'(s)$ is its tangent indicatrix.
Let us use the shortcut notation $\Norm(s)\z=\Norm(\gamma(s))$.
Note that the unit vectors $\tan(s)$ and $\Norm(s)$ are orthogonal.
Therefore there is a unique unit vector $\mu(s)$\index{10tmn@$\tan$, $\mu$, $\Norm$ (Darboux frame)} such that 
$\tan(s),\mu(s),\Norm(s)$ is an oriented orthonormal basis;
it is called the \index{Darboux frame}\emph{Darboux frame} of $\gamma$ in~$\Sigma$.

Since $\T_{\gamma (s)}\z\perp\Norm(s)$, the vector $\mu(s)$ is tangent to $\Sigma$ at $\gamma(s)$.
In fact, $\mu(s)$ is a counterclockwise rotation of $\tan(s)$ by the angle $\tfrac\pi2$ in the tangent plane $\T_{\gamma(s)}$.
This vector can also be defined as the vector product $\mu(s)\z\df\Norm(s)\times \tan(s)$.

Since $\gamma$ is unit-speed, we have that $\gamma''\perp \gamma'$ (see \ref{prop:a'-pertp-a''}).
Therefore, the acceleration of $\gamma$ can be written as a linear combination of $\mu$ and $\Norm$;
that is, \index{10k@$k_g$ geodesic curvature}\index{10k@$k_n$ (normal curvature)}
\[\gamma''(s)=k_g(s)\cdot \mu(s)+k_n(s)\cdot\Norm(s).\]
The values $k_g(s)$ and $k_n(s)$ are called \index{curvature}\index{geodesic!curvature}\emph{geodesic} and \index{normal!curvature}\emph{normal curvatures} of $\gamma$ at $s$, respectively.
Since the frame $\tan(s),\mu(s),\Norm(s)$ is orthonormal, these curvatures can be also written as the following scalar products:
\begin{align*}
k_g(s)&=\langle \gamma''(s),\mu(s)\rangle= 
&
k_n(s)&=\langle \gamma''(s),\Norm(s)\rangle=
\\
&=\langle \tan'(s),\mu(s)\rangle,
&
&=\langle \tan'(s),\Norm(s)\rangle.
\end{align*}

Since $0=\langle \tan(s),\Norm(s)\rangle$ we have 
that 
\begin{align*}
0&=\langle \tan(s),\Norm(s)\rangle'=
\\
&=\langle \tan'(s),\Norm(s)\rangle+\langle \tan(s),\Norm'(s)\rangle=
\\
&=k_n(s)+\langle \tan(s),D_{\tan(s)}\Norm\rangle.
\end{align*}
Applying the definition of the shape operator,
we get the following.

\begin{thm}{Proposition}\label{prop:normal-shape}
Assume $\gamma$ is a smooth unit-speed curve in a smooth surface~$\Sigma$.
Let $p=\gamma(s_0)$ and $\vec v=\gamma'(s_0)$.
Then 
\[k_n(s_0)=\langle \Shape_p(\vec v),\vec v\rangle,\]
where $k_n$ denotes the normal curvature of $\gamma$ at $s_0$, and $\Shape_p$ is the shape operator at~$p$.
\end{thm}

Note that according to the proposition, the normal curvature of a smooth curve in $\Sigma$ is completely determined by the velocity vector $\vec v$ at the point~$p$.
For that reason, the normal curvature is also denoted by $k_{\vec v}$;\index{10k@$k_{\vec v}$ (normal curvature)}
according to the proposition,
\[k_{\vec v}=\langle \Shape_p(\vec v),\vec v\rangle\]
for any unit vector $\vec v$ in $\T_p$.

\section{Euler's formula}

Let $p$ be a point on a smooth surface~$\Sigma$.
Assume we choose tangent-normal coordinates at $p$ so that the Hessian matrix is diagonalized, then we have
\[M_p=\begin{pmatrix}
 k_1(p)
 &0
 \\
 0
 &k_2(p)
 \end{pmatrix}.
\]
Consider a vector ${\vec w}=a\cdot\vec i+b\cdot\vec j$ in the $(x,y)$-plane.
Then by \ref{cor:Shape(ij)}, we have
\[
\langle \Shape\vec w,\vec w\rangle
=a^2\cdot k_1(p) +b^2\cdot k_2(p). 
\]
If ${\vec w}$ is unit, then $a^2+b^2=1$ which implies the following.

\begin{thm}{Observation}\label{obs:k1-k2}
For any point $p$ on an oriented smooth surface $\Sigma$,
the principal curvatures $k_1(p)$ and $k_2(p)$ are respectively the minimum and maximum of the normal curvatures at~$p$.
Moreover, if $\theta$ is the angle between a unit vector ${\vec w}\in\T_p$ and the first principal direction at $p$, then 
\[k_{\vec w}(p)=k_1(p)\cdot(\cos\theta)^2+k_2(p)\cdot(\sin\theta)^2.\]

\end{thm}

The last identity is called \index{Euler's formula}\emph{Euler's formula}.

\begin{thm}{Exercise}\label{ex:mean-curvature}
Let $p$ be a point on a smooth surface.
Show that the sum of the normal curvatures for any pair of orthogonal directions, at a point $p$ is $H(p)$ --- the mean curvature at~$p$. 
\end{thm}

\begin{thm}{Exercise}\label{ex:average}
Let $p$ be a point on a smooth surface.
Show that $\tfrac38\cdot H(p)^2-\tfrac12\cdot K(p)$ is the average value of $k_{\vec w}^2$ for all unit vectors ${\vec w}\in\T_p$.
\end{thm}



\begin{thm}{Meusnier's theorem}
\label{thm:meusnier}
\index{Meusnier's theorem}
Let $\gamma$ be a smooth curve that runs along a smooth oriented surface~$\Sigma$.
Let $p=\gamma(t_0)$, ${\vec v}\z=\gamma'(t_0)$, and $\alpha\z=\measuredangle(\Norm(p),\norm(t_0))$;
that is, $\alpha$ is the angle between the unit normal to $\Sigma$ at $p$ and the unit normal vector in the Frenet frame of $\gamma$ at~$t_0$.
Then the following identity holds true: 
\[\kur(t_0)\cdot\cos\alpha=k_{n}(t_0);\]
here $\kur(t_0)$ and $k_n(t_0)$ are the curvature and the normal curvature of $\gamma$ at $t_0$, respectively. 
\end{thm}


\parit{Proof.} Since $\gamma''=\tan'=\kur\cdot \norm$, we get that
\begin{align*}
k_{n}(t_0)&=\langle\gamma'',\Norm\rangle=
\\
&=\kur(t_0)\cdot\langle\norm,\Norm\rangle=
\\
&=\kur(t_0)\cdot\cos\alpha.
\end{align*}
\qedsf

The theorem above, as well as the statement in the following exercise, were proved by Jean Baptiste Meusnier \cite{meusnier}.

\begin{thm}{Exercise}\label{ex:meusnier}
Let $\Sigma$ be a smooth surface, $p\in\Sigma$, and ${\vec v}\in \T_p\Sigma$ a unit vector.
Assume $k_{\vec v}(p)\ne 0$; that is, the normal curvature of $\Sigma$ at $p$ in the direction of ${\vec v}$ does not vanish.

Show that the osculating circles at $p$ of smooth curves in $\Sigma$ that run in the direction ${\vec v}$ sweep out a sphere $S$ with center at $p+\tfrac1{k_{\vec v}}\cdot\Norm$ and radius $r=\tfrac1{|k_{\vec v}|}$.
\end{thm}

\begin{thm}{Exercise}\label{ex:principal-revolution}
Let $\gamma(s)=(x(s),y(s))$ be a smooth unit-speed simple plane curve in the upper half-plane,
and $\Sigma$ be the surface of revolution around the $x$-axis with generatrix~$\gamma$.
Assume that $x'\ne 0$.

\begin{subthm}{ex:principal-revolution:a}
Show that the parallels and meridians are lines of curvature on~$\Sigma$.
\end{subthm}

\begin{subthm}{ex:principal-revolution:formula}
Show that 
\[\frac{|x'(s)|}{y(s)}
\quad
\text{and}
\quad
\frac{-y''(s)}{|x'(s)|}
\]
are the principal curvatures of $\Sigma$ at $(x(s),y(s),0)$ in the direction of the corresponding parallel and meridian, respectively.
\end{subthm}

\begin{subthm}{ex:principal-revolution:pseudosphere}
Show that $\Sigma$ has Gauss curvature $-1$ at all points if and only if $y$ satisfies the differential equation $y''=y$. 
The case $y=e^{-s}$ is shown; this is the so-called \index{pseudosphere}\emph{pseudosphere}.
\end{subthm}

\end{thm}

\begin{figure}[ht!]
\vskip-0mm
\hskip30mm
\includegraphics{asy/pseudosphere}
\vskip-3mm
\end{figure}

\begin{thm}{Exercise}\label{ex:catenoid-is-minimal}
Show that the \index{catenoid}\emph{catenoid} defined implicitly by the equation
\[(\cosh z)^2=x^2+y^2\]
is a minimal surface.
\end{thm}

\begin{thm}{Exercise}\label{ex:helicoid-is-minimal}
Show that the \index{helicoid}\emph{helicoid} defined by the following parametric equation
\[s(u,v)=(u\cdot \sin v,u\cdot \cos v,v)\]
is a minimal surface.
\end{thm}

\begin{wrapfigure}{r}{51 mm}
\vskip-6mm
\centering
\includegraphics{asy/sin-mini}
\vskip0mm
\end{wrapfigure}

\begin{thm}{Exercise}\label{ex:rev(sin)}
Let $\Sigma$ be the surface of revolution around the $x$-axis
with generatrix $y=a\cdot \sin x$ for a constant $a>0$ and $x\in (0,\pi)$.
Show that the Gauss curvature of $\Sigma$ does not exceed 1.
\end{thm}

\begin{thm}{Exercise}\label{ex:rev(lin)}
Let $f\:(a,b)\to\mathbb{R}$ be a smooth positive function and $\Sigma$ be a surface of revolution of the graph $y=f(x)$ around $x$-axis.
Suppose $\Sigma$ has vanishing Gauss curvature.
Show that $f$ is a linear function; that is $f(x)=c\cdot x+d$ for some constants $c$ and $d$.
\end{thm}

\section{Lagunov's fishbowl}
\index{Lagunov's fishbowl}

The following question is a 3-dimensional analog of the moon in a puddle problem (\ref{thm:moon}).

\begin{thm}{Question}\label{quest:lagunov}
Assume a set $V\subset \mathbb{R}^3$ is bounded by a closed connected surface $\Sigma$ with 
principal curvatures bounded in absolute value by 1.
Is it true that $V$ contains a ball of radius~1?
\end{thm}

\begin{thm}{Exercise}\label{ex:moon-revolution}
Show that Question \ref{quest:lagunov} has an affirmative answer if $V$ is a body of revolution.
\end{thm}

Later (see \ref{ex:convex-lagunov})
we will show that the answer is ``yes'' for convex surfaces.
Now we are going to show by an example that the answer is ``no'' in the general case;
it was constructed by Vladimir Lagunov \cite{lagunov-1961}.

\parit{Construction.}
Let us start with a body of revolution $V_1$ whose cross-section is shown on the next diagram.
The boundary curve of the cross-section consists of 6 long horizontal line segments included in 3 simple closed smooth curves.
(To make the curves smooth, one has to use cutoffs and mollifiers from Section~\ref{sec:analysis}.)
The boundary of $V_1$ has 3 components, each of which is a smooth sphere.

\begin{figure}[ht!]
\centering
\includegraphics{mppics/pic-33}
\vskip0mm
\end{figure}

We assume that the curves have curvature at most~1.
Moreover, except for the almost horizontal parts, the curve has absolute curvature close to 1 all the time.
The only thick part in $V$ is the place where all three boundary components come close together;
the remaining part of $V$ is assumed to be very thin.
It could be arranged that the radius $r$ of the maximal ball in $V$ is just a bit above 
$r_2=\tfrac2{\sqrt{3}}-1$.
(The small black disc on the diagram has radius $r_2$,
assuming that the three big circles are unit.)
In particular, we may assume that $r<\tfrac16$.

\begin{wrapfigure}{o}{21 mm}
\vskip-0mm
\centering
\includegraphics{mppics/pic-910}
\vskip0mm
\end{wrapfigure}

Exercise \ref{ex:principal-revolution} gives formulas for the principal curvatures of the boundary of $V$;
which imply that both principal curvatures are at most 1 in absolute value. 

It remains to modify $V_1$ to make its boundary connected without increasing the bounds on its principal curvatures and without allowing larger balls inside.

Note that each sphere in the boundary contains two flat discs;
they come into pairs closely lying to each other. 
Let us drill thru two of such pairs and reconnect the holes by another body of revolution whose 
axis is shifted but stays parallel to the axis of $V_1$.
Denote the obtained body by $V_2$; the cross-section of the obtained body is shown on the diagram. 

Then repeat the operation for the other two pairs.
Denote the obtained body by $V_3$.

Note that the boundary of $V_3$ is connected.
Assuming that the holes are large, its boundary can be made so that its principal curvatures are still at most $1$; the latter can be proved in the same way as for~$V_1$.
\qeds


The bound $r_2=\tfrac2{\sqrt{3}}-1$ is optimal;
this and more was proved by Vladimir Lagunov and Abram Fet \cite{lagunov-1960, lagunov-fet-1963, lagunov-fet-1965}.

The surface of $V_3$ in Lagunov's fishbowl has genus 2;
that is, it can be parametrized by a sphere with two handles.

\begin{wrapfigure}{o}{55 mm}
\centering
\vskip-0mm
\includegraphics{mppics/pic-920}
\vskip0mm
\end{wrapfigure}

Indeed, the boundary of $V_1$ consists of three smooth spheres.

When we drill, we make four holes --- two spheres get one hole in each, and one sphere gets two holes.
We reconnect two spheres with a tube and obtain one sphere.
By connecting the two holes of the other sphere with a tube we get a torus;
it is on the right side in the picture of $V_2$.
That is, the boundary of $V_2$ is formed by one sphere and one torus.

To construct $V_3$ from $V_2$, we make a torus from the remaining sphere and connect it to the other torus by a tube.
This way we get a sphere with two handles.

\begin{thm}{Exercise}\label{ex:lagunov-genus4}
Modify Lagunov's construction to make the boundary surface a sphere with 4 handles.
\end{thm}

%Recall that Lagunov's fishbowl contains a ball of radius $r_2=\tfrac2{\sqrt{3}}-1$.
%It turns out that this radius is optimal.
%Moreover, \textit{suppose a connected body $V\subset \mathbb{R}^3$ is bounded by a finite number of closed smooth surfaces with principal curvatures bounded by~1.
%Assume $V$ does not contain a ball of radius $r_2$.
%Then the boundary of $V$ has two diffeomorphic connected components.}
%Such $V$ could be the region squeezed between two close surfaces; for example, a region between two concentric spheres.

%For bodies bounded by a sphere, there is a better bound $r_3=\sqrt{\tfrac32}-1$;
%it is the radius of the smallest sphere tangent to four unit spheres that are tangent to each other.
%Note that $r_3>r_2$.
%\textit{If a body $V\subset \mathbb{R}^3$ is bounded by a smooth sphere with principal curvatures bounded  by~1.
%Then $V$ contains a ball of radius~$r_3$.
%Moreover, this bound is sharp; that is, for every $\epsilon>0$, there are examples of such $V$ not containing a ball of radius $r_3+\epsilon$.}

