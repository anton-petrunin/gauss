\chapter{Definitions}
\label{chap:surfaces-def}
\section{Topological surfaces}

We will be mostly interested in smooth regular surfaces defined in the following section.
However, we will sometimes use the following general definition.

A connected subset $\Sigma$ in the Euclidean space $\mathbb{R}^3$
is called a \index{surface}\index{topological!surface}\emph{topological surface} (more precisely an {}\emph{embedded surface without boundary}) 
if any point of $p\in \Sigma$ admits a neighborhood $W$ in $\Sigma$ 
that can be parametrized by an open subset in the Euclidean plane; 
that is, there is an injective continuous map $U\to W$ from an open set $U\subset \mathbb{R}^2$ such that its inverse $W\to U$ is also continuous.


\section{Smooth surfaces}\label{sec:def-smooth-surface}

Recall that a function $f$ of two variables $x$ and $y$ is called \index{smooth!function}\emph{smooth} if all its partial derivatives $\frac{\partial^{m+n}}{\partial x^m\partial y^n}f$ are defined and are continuous in the domain of definition of $f$. 

A connected set $\Sigma \subset \mathbb{R}^3$ is called a \index{surface}\index{smooth!surface}\emph{smooth surface} (we use it as a shortcut for the more precise term {}\emph{smooth regular embedded surface}) if it can be described locally as a graph of a smooth function in an appropriate coordinate system.

More precisely, for any point $p\in \Sigma$ one can choose a coordinate system $(x,y,z)$ and a neighborhood $U\ni p$ such that
the intersection $W=U\cap \Sigma$ is a graph $z=f(x,y)$ of a smooth function $f$ defined in an open domain of the $(x,y)$-plane.

\parbf{Examples.}
The simplest example of a smooth surface is the $(x,y)$-plane 
\[\Pi=\set{(x,y,z)\in\mathbb{R}^3}{z=0}.\]
The plane $\Pi$ is a surface since
it can be described as the graph of the function $f(x,y)=0$.

All other planes are smooth surfaces as well since one can choose a coordinate system so that it becomes the $(x,y)$-plane.
We may also present a plane as a graph of a linear function 
$f(x,y)=a\cdot x+b\cdot y+c$ for some constants $a$, $b$, and $c$
(assuming the plane is not perpendicular to the $(x,y)$-plane, in which case a different coordinate system is required to write the plane as the graph of a function).

A more interesting example is the unit sphere 
\[\mathbb{S}^2=\set{(x,y,z)\in\mathbb{R}^3}{x^2+y^2+z^2=1}.\]
This set is not the graph of any function,
but $\mathbb{S}^2$ is locally a graph;
it can be covered by the following 6 graphs:
\begin{align*}
z&=f_\pm(x,y)=\pm \sqrt{1-x^2-y^2},
\\
y&=g_\pm(x,z)=\pm \sqrt{1-x^2-z^2},
\\
x&=h_\pm(y,z)=\pm \sqrt{1-y^2-z^2},
\end{align*}
where each function $f_\pm,g_\pm,h_\pm$ is defined in an open unit disc.
Any point $p\in\mathbb{S}^2$ lies in one of these graphs therefore $\mathbb{S}^2$ is a surface.
Since each function is smooth, so is the surface $\mathbb{S}^2$.

\section{Surfaces with boundary}
A connected subset in a surface that is bounded by one or more 
curves is called a \index{surface!with boundary}\emph{surface with boundary}; such curves form the \index{boundary line}\emph{boundary line} of the surface.

When we say {}\emph{surface} we usually mean a {}\emph{smooth regular surface without boundary};
we may use the term {}\emph{surface without boundary} if we need to emphasize it;
otherwise, we may use the term {}\emph{surface with possibly nonempty boundary}.

\section{Proper, closed and open surfaces}
If the surface $\Sigma$ is formed by a closed set, then it is called \index{proper!surface}\emph{proper}.
For example, for any smooth function $f$ defined on the whole plane, its graph $z=f(x,y)$ is a proper surface.
The sphere $\mathbb{S}^2$ gives another example of a proper surface.

On the other hand, the open disc 
\[\set{(x,y,z)\in\mathbb{R}^3}{x^2+y^2<1,\  z=0}\]
is not proper; this set is neither open nor closed.

A compact surface without boundary is called \index{closed!surface}\emph{closed}
(it is closely related to the term {}\emph{closed curve}, but has nothing to do with {}\emph{closed set}).

A proper noncompact surface without boundary is called \index{open!surface}\emph{open} (again the term {}\emph{open set} is not relevant).

For example, the paraboloid $z=x^2+y^2$
is an open surface; the 
sphere $\mathbb{S}^2$ is a closed surface.

Note that any proper surface without boundary is either closed or open.

The following claim is a three-dimensional analog of the plane separation theorem (\ref{ex:proper-curve}).
Despite it might look obvious, its proof is not trivial at all; a standard proof uses the so-called {}\emph{Alexander's duality} which is a classical technique in algebraic topology \cite[see][]{hatcher}.
We omit its proof since it would take us far away from the main subject.

\begin{thm}{Claim}\label{clm:proper-divides}
The complement of any proper topological surface without boundary (or, equivalently any open or closed topological surface) has exactly two connected components. 
\end{thm}

\section{Implicitly defined surfaces}

\begin{thm}{Proposition}\label{prop:implicit-surface}
Let $f\:\mathbb{R}^3\to \mathbb{R}$ be a smooth function.
Suppose that $0$ is a regular value of $f$;
that is, $\nabla_p f\ne 0$ at any point $p$ such that $f(p)=0$.
Then any connected component $\Sigma$ of the set of solutions of the equation $f(x,y,z)=0$ is a smooth surface.
\end{thm}

\parit{Proof.}
Fix $p\in\Sigma$.
Since $\nabla_p f\ne 0$ we have 
\[f_x(p)\ne 0,\quad f_y(p)\ne 0,\quad \text{or}\quad f_z(p)\ne 0.\]
We may assume that $f_z(p)\ne 0$;
otherwise, permute the coordinates $x,y,z$.

The implicit function theorem (\ref{thm:imlicit}) implies that a neighborhood of $p$ in $\Sigma$ is the graph $z=h(x,y)$ of a smooth function $h$ defined on an open domain in $\mathbb{R}^2$.
It remains to apply the definition of smooth surfaces (Section~\ref{sec:def-smooth-surface}).
\qeds

\begin{thm}{Exercise}\label{ex:hyperboloinds}
For which constants $\ell$ 
does the following equation
\begin{align*}
x^2+y^2-z^2&=\ell
\end{align*}
describe a smooth regular surface.
\end{thm}

\section{Local parametrizations}
\index{parametrization}

Let $U$ be an open domain in $\mathbb{R}^2$, and $s\:U\to \mathbb{R}^3$ be a smooth map.
We say that $s$ is regular if its Jacobian has maximal rank;
in this case, it means that the vectors $s_u$ and $s_v$ are linearly independent at any $(u,v)\in U$;
equivalently $s_u\times s_v\ne 0$, where $\times$ denotes the vector product.

\begin{thm}{Proposition}\label{prop:graph-chart}
If $s\:U\to \mathbb{R}^3$ is a smooth regular embedding of an open connected set $U\subset \mathbb{R}^2$, then its image $\Sigma=s(U)$ is a smooth surface.
\end{thm}

\parit{Proof.}
Set 
\[s(u,v)=(x(u,v),y(u,v),z(u,v)).\]
Since $s$ is regular, its Jacobian matrix
\[\Jac s=
\renewcommand\arraystretch{1.3}
\begin{pmatrix}
x_u&x_v\\
y_u&y_v\\
z_u&z_v
\end{pmatrix}
\]
has rank two at any pint $(u,v)\in U$.

Choose a point $p\in \Sigma$; by shifting the $(x,y,z)$ and $(u,v)$ coordinate systems we may assume that $p$ is the origin and $p=s(0,0)$.
Permuting the coordinates $x,y,z$ if necessary, we may assume that 
the matrix 
\[
\renewcommand\arraystretch{1.3}
\begin{pmatrix}
x_u&x_v\\
y_u&y_v
\end{pmatrix},
\] 
is invertible at the origin.
Note that this is the Jacobian matrix of the map
\[(u,v)\mapsto (x(u,v),y(u,v)).\]

The inverse function theorem (\ref{thm:inverse}) implies that there is a smooth regular map
$w\:(x,y)\mapsto (u,v)$ defined on an open set $W\ni 0$ in the $(x,y)$-plane
such that $w(0,0)=(0,0)$ and  $s\circ w(x,y)=(x,y,f(x,y))$ for some smooth function~$f$.
That is, the graph $z=f(x,y)$ for $(x,y)\z\in W$ is a subset in $\Sigma$.
By the inverse function theorem, this graph is open in $\Sigma$.

Since $p$ is arbitrary, we get that $\Sigma$ is a surface.
\qeds

If we have $s$ and $\Sigma$ as in the proposition, then we say that $s$ is a \index{smooth!parametrization}\emph{smooth parametrization} of the surface $\Sigma$. 

Not all smooth surfaces can be described by such a parametrization;
for example, the sphere $\mathbb{S}^2$ cannot.
However, any smooth surface $\Sigma$ admits a local parametrization; that is, any point $p\in\Sigma$ admits an open neighborhood $W\subset \Sigma$ with a smooth regular parametrization~$s$.
In this case, any point in $W$ can be described by two parameters, usually denoted by $u$ and $v$, 
which are called \index{local!coordinates}\emph{local coordinates} at $p$.
The map $s$ is called a \index{chart}\emph{chart} of $\Sigma$.

If $W$ is a graph $z=h(x,y)$ of a smooth function $h$, then the map 
\[s\:(u,v)\mapsto (u,v,h(u,v))\] is a chart.
Indeed, $s$ has an inverse $(u,v,h(u,v))\mapsto (u,v)$ which is continuous;
that is, $s$ is an embedding.
Further,
$s_u=(1,0,h_u)$, and $s_v=(0,1,h_v)$. 
Whence the partial derivatives $s_u$ and $s_v$ are linearly independent;
that is, $s$ is a regular map.

Note that from \ref{prop:graph-chart}, we obtain the following corollary.

\begin{thm}{Corollary}\label{cor:reg-parmeterization}
A connected set $\Sigma\subset \mathbb{R}^3$ is a smooth regular surface if and only if a neighborhood of any point in $\Sigma$ can be covered by a chart.
\end{thm}


\begin{thm}{Exercise}\label{ex:inversion-chart}
Consider the following map 
\[s(u,v)=(\tfrac{2\cdot u}{1+u^2+v^2},\tfrac{2\cdot v}{1+u^2+v^2},\tfrac{2}{1+u^2+v^2}).\]
Show that $s$ is a chart of the unit sphere centered at $(0,0,1)$; describe the image of $s$.
\end{thm}

\begin{wrapfigure}{o}{31 mm}
\vskip-6mm
\centering
\includegraphics{mppics/pic-750}
\vskip0mm
\end{wrapfigure}

The map $s$ in the exercise can be visualized using the following map
\[(u,v,1)\mapsto (\tfrac{2\cdot u}{1+u^2+v^2},\tfrac{2\cdot v}{1+u^2+v^2},\tfrac{2}{1+u^2+v^2})\]
which is called \index{stereographic projection}\emph{stereographic projection} from the plane $z=1$ to the unit sphere with center at $(0,0,1)$.
Note that the point $(u,v,1)$ and its image lie on the same half-line emerging from the origin. 

\begin{wrapfigure}{o}{31 mm}
\vskip0mm
\centering
\includegraphics{asy/torus}
\vskip0mm
\end{wrapfigure}

Let $\gamma(t)=(x(t),y(t))$ be a plane curve.
Recall that the \index{surface!of revolution}\emph{surface of revolution} of the curve $\gamma$ around the $x$-axis can be described as the 
image of the map 
\[(t, s)\mapsto (x(t), y(t)\cdot\cos s,y(t)\cdot\sin s).\]
For fixed $t$ or $s$ the obtained curves are called \index{meridian}\emph{meridians} or \index{parallel}\emph{parallels} of the surface, respectively; note that parallels are formed by circles in the plane perpendicular to the axis of rotation.
The curve $\gamma$ is called the \index{generatrix}\emph{generatrix} of the surface.

\begin{thm}{Exercise}\label{ex:revolution}
Assume $\gamma$ is a closed simple smooth regular plane curve that does not intersect the $x$-axis.
Show that the surface of revolution around the $x$-axis with generatrix $\gamma$ is a smooth regular surface.
\end{thm}


\section{Global parametrizations}
\index{parametrization}

A surface can be described by an embedding from a known surface to the space.

For example, consider the ellipsoid
\[\Sigma_{a,b,c}=\set{(x,y,z)\in\mathbb{R}^3}{\tfrac{x^2}{a^2}+\tfrac{y^2}{b^2}+\tfrac{z^2}{c^2}=1}\]
for some positive numbers $a$, $b$, and $c$.
Note that by \ref{prop:implicit-surface}, $\Sigma_{a,b,c}$ is a smooth regular surface.
Indeed, set $f(x,y,z)=\tfrac{x^2}{a^2}+\tfrac{y^2}{b^2}+\tfrac{z^2}{c^2}$,
then
\[\nabla f(x,y,z)=(\tfrac{2}{a^2}\cdot x,\tfrac{2}{b^2}\cdot y,\tfrac{2}{c^2}\cdot z).\]
Therefore, $\nabla f\ne0$ if $f=1$; that is, $1$ is a regular value of $f$.

Note that $\Sigma_{a,b,c}$ can be defined as the image of the map $s\:\mathbb{S}^2\to\mathbb{R}^3$, defined as the restriction of the following map to the unit sphere $\mathbb{S}^2$:
\[(x,y,z)\z\mapsto (a\cdot x, b\cdot y,c\cdot z).\]

For a surface $\Sigma$, a map $s\: \Sigma \to \mathbb{R}^3$ is called a 
\emph{smooth parametrized surface} if $s$ is injective, and, for any chart $f\:U\to \Sigma$,
the composition $s\circ f$ is smooth and regular;
that is, all partial derivatives $\frac{\partial^{m+n}}{\partial u^m\partial v^n}(s\circ f)$ are defined and continuous in the domain of definition, and the two vectors 
$\frac{\partial}{\partial u}(s\circ f)$ and $\frac{\partial}{\partial v}(s\circ f)$ are linearly independent.

Note that in this case, the image $\Sigma^{*}=s(\Sigma)$ is a smooth surface.
The latter follows since for any chart $f\:U\to \Sigma$ the composition $s\circ f\:U\to \Sigma^{*}$ is a chart of $\Sigma^{*}$. 
The map $s$ is called a \index{diffeomorphism}\emph{diffeomorphism} from $\Sigma$ to $\Sigma^{*}$;
the surfaces $\Sigma$ and $\Sigma^{*}$ are said to be {}\emph{diffeomorphic} if there is a diffeomorphism $s\:\Sigma\to\Sigma^{*}$.


\begin{thm}{Advanced exercise}\label{ex:star-shaped-disc}
Show that the surfaces $\Sigma$ and $\Theta$ are diffeomorphic if

\begin{subthm}{ex:plane-n}
$\Sigma$ and $\Theta$ obtained from the plane by removing $n$ points.
\end{subthm}


\begin{subthm}{ex:star-shaped-disc:smooth}
$\Sigma$ and $\Theta$ are open convex subsets of a plane bounded by smooth curves.
\end{subthm}


\begin{subthm}{ex:star-shaped-disc:nonsmooth}
$\Sigma$ and $\Theta$ are open convex subsets of a plane.
\end{subthm}

\begin{subthm}{ex:star-shaped-disc:star-shaped}
$\Sigma$ and $\Theta$ are open star-shaped subsets of a plane.
\end{subthm}
\end{thm}

