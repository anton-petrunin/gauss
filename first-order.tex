\chapter{First-order structure}
\label{chap:first-order}
\section{Tangent plane}

\begin{thm}{Definition}\label{def:tangent-vector}
Let $\Sigma$ be a smooth surface.
A vector $\vec w$ is \index{tangent!vector}\emph{tangent} to $\Sigma$ at $p$ if and only if there is a curve $\gamma$ that runs in $\Sigma$ and has $\vec w$ as a velocity vector at $p$;
that is, $p=\gamma(t)$ and $\vec w=\gamma'(t)$ for some~$t$.
\end{thm}

\begin{thm}{Proposition-Definition}\label{def:tangent-plane}
Let $\Sigma$ be a smooth surface and $p\in \Sigma$.
Then the set of tangent vectors of $\Sigma$ at $p$ forms a plane;
this plane is called the \index{tangent!plane}\emph{tangent plane} of $\Sigma$ at~$p$.

Moreover, if $s\:U\to \Sigma$ is a local chart and $p=s(u_0,v_0)$, then 
the tangent plane of $\Sigma$ at $p$ is spanned by vectors $s_u(u_0,v_0)$ and $s_v(u_0,v_0)$.
\end{thm}

The tangent plane to $\Sigma$ at $p$ is usually denoted by $\T_p$ or $\T_p\Sigma$.
This plane $\T_p$ might be considered as a linear subspace of $\mathbb{R}^3$ or as a parallel plane passing thru $p$;
the latter is sometimes called the \index{tangent!plane}\emph{affine tangent plane}.
The affine tangent plane can be interpreted as the best approximation at~$p$ of the surface $\Sigma$ by a plane.
More precisely, 
it has \index{order of contact}\emph{first-order contact} with $\Sigma$ at $p$;
that is, $\rho(q)\z=o(|p-q|)$, where $q\in \Sigma$, and $\rho(q)$ denotes the distance from $q$ to $\T_p$.

An assignment of a tangent vector $\vec v_p$ to each point $p$ of the surface $\Sigma$ is called a \emph{tangent vector field} if $\vec v_p$ depends smoothly of $p$.
More precisely, for any chart $s$ of $\Sigma$ we have
\[\vec v_{s(u,v)}=a(u,v)\cdot s_u+b(u,v)\cdot s_v\]
for smooth functions $a$ and $b$ defined in the domain $s$.

\parit{Proof.}
Fix a chart $s$ at~$p$.
Assume $\gamma$ is a smooth curve that starts at~$p$.
Without loss of generality, we can assume that $\gamma$ is covered by the chart;
in particular, there are smooth functions $t\mapsto u(t)$ and $t\mapsto v(t)$ such that 
\[\gamma(t)=s(u(t),v(t)).\]
Applying the chain rule, we get
\[\gamma'=s_u\cdot u'+ s_v\cdot v';\]
that is, $\gamma'$ is a linear combination of $s_u$ and $s_v$.

The smooth functions $t\mapsto u(t)$ and $t\mapsto v(t)$ can be chosen arbitrarily.
Therefore any linear combination of $s_u$ and $s_v$ is a tangent vector at~$p$. 
\qeds


\begin{thm}{Exercise}\label{ex:tangent-normal}
Let $f:\mathbb{R}^3\to\mathbb{R}$ be a smooth function with $0$ as a regular value, and $\Sigma$ be a surface described as a connected component of the level set $f(x,y,z)=0$.
Show that the tangent plane $\T_p\Sigma$ is perpendicular to the gradient $\nabla_pf$ at any point $p\in\Sigma$.
\end{thm}

\begin{thm}{Exercise}\label{ex:vertical-tangent}
Let $\Sigma$ be a smooth surface and $p\in\Sigma$.
Choose $(x,y,z)$-coordinates.
Show that a neighborhood of $p$ in $\Sigma$ is a graph $z=f(x,y)$ of a smooth function $f$ defined on an open subset in the $(x,y)$-plane if and only if the tangent plane $\T_p$ is not {}\emph{vertical}; that is, if $\T_p$ is not perpendicular to the $(x,y)$-plane.
\end{thm}

\begin{thm}{Exercise}\label{ex:tangent-single-point}
Show that if a smooth surface $\Sigma$ meets a plane $\Pi$ at a single point $p$, then $\Pi$ is tangent to $\Sigma$ at~$p$.
\end{thm}


\section{Directional derivative}\label{sec:dirder}

In this section, we extend the definition of directional derivative to smooth functions defined on smooth surfaces.
First, let us recall the standard definition.

Suppose $f$ is a function defined at a point $p$ in the space, and $\vec w$ a vector.
Consider the function
$h(t)=f(p+t\cdot\vec w)$.
Then the directional derivative of $f$ at $p$ along $\vec w$ is defined as \index{10d@$D_{\vec{w}}f$ (directional derivative)}
\[D_{\vec w}f(p)\df h'(0).\]

Recall that a function $f: \Sigma \to \mathbb{R}$ is said to be smooth if for any chart $s : U \to \Sigma$, the composition $f \circ s$ is smooth.

\begin{thm}{Proposition-Definition}\label{def:directional-derivative}
Let $f$ be a smooth function defined on a smooth surface $\Sigma$.
Suppose $\gamma$ is a smooth curve in $\Sigma$ that starts at $p$ with velocity vector $\vec{w}\in \T_p$;
that is, $\gamma(0)=p$, and $\gamma'(0)=\vec{w}$.
Then the derivative $(f\circ\gamma)'(0)$
depends only on $f$, $p$, and $\vec{w}$;
it is called the \index{directional derivative}\emph{directional derivative} of $f$ along $\vec{w}$ at $p$
and is denoted by
\[D_{\vec{w}}f,\quad D_{\vec{w}}f(p), \quad\text{or}\quad D_{\vec{w}}f(p)_\Sigma\] 
--- we may omit $p$ and $\Sigma$ if it is clear from the context.

Moreover, if $(u,v)\mapsto s(u,v)$ is a local chart at $p$, then 
\[D_{\vec{w}}f=a\cdot f_u+b\cdot f_v,\]
where $\vec{w}=a\cdot s_u +b\cdot s_v$. 
\end{thm}

Note that our definition agrees with the standard definition of the directional derivative if $\Sigma$ is a plane.
Indeed, in this case, $\gamma(t)=p+\vec w\cdot t$ is a curve in $\Sigma$ that starts at $p$ with velocity vector $\vec{w}$.
For a general surface, the point $p+\vec w\cdot t$ might not lie on the surface.
Therefore the function $f$ might be undefined at this point.
In this case, the standard definition does not work.

\parit{Proof.}
Without loss of generality, we may assume that $p=s(0,0)$ and the curve $\gamma$ is covered by the chart $s$;
if not we can chop~$\gamma$.
In this case, 
\[\gamma(t)=s(u(t),v(t))\]
for smooth functions $u,v$ defined in a neighborhood of $0$ such that 
$u(0)\z=v(0)\z=0$.

Applying the chain rule, we get that
\begin{align*}
\gamma'(0)&=u'(0)\cdot s_u+v'(0)\cdot s_v
\end{align*}
at $(0,0)$.
Since $\vec{w}=\gamma'(0)$ and the vectors $s_u$, $s_v$ are linearly independent, we get that $a=u'(0)$ and $b=v'(0)$.

Applying the chain rule again, we get that
\[
(f\circ\gamma)'(0)=a\cdot f_u+b\cdot f_v.
\eqlbl{eq:f-gamma}
\]
at $(0,0)$.

Notice that the left-hand side in \ref{eq:f-gamma} does not depend on the choice of the chart $s$, and the right-hand side depends only on $p$, $\vec w$, $f$, and~$s$. 
It follows that $(f\circ\gamma)'(0)$ depends only on $p$, $\vec w$, and~$f$.

The last statement follows from \ref{eq:f-gamma}.
\qeds

\begin{thm}{Advanced exercise}\label{ex:lin-ind-chart}
Let $\vec x$ and $\vec y$ be vector fields on a smooth surface $\Sigma$.
Suppose that $\vec x_p$ and $\vec y_p$ are linearly independent at some point $p\in \Sigma$.
Construct two functions $u$ and $v$ in a neighborhood of $p$ such that 
\begin{align*}
D_{\vec x} u&>0,
&
D_{\vec y} u&=0,
&
D_{\vec x} v&=0,
&
D_{\vec y} v&>0.
\end{align*}

Conclude that there is a chart $(u,v)\mapsto s(u,v)$ of $\Sigma$ at $p$ such that $\vec x$ and $\vec y$ are tangent to the coordinate lines.
\end{thm}


\section{Tangent vectors as functionals}

In this section, we introduce a more conceptual way to define tangent vectors.
We will not use this approach in the sequel, but it is better to know about it.

A tangent vector $\vec w\in \T_p$ to a smooth surface $\Sigma$ 
defines a linear functional%
\footnote{Term {}\emph{functional} is used for functions that take a function as an argument and return a number.} $D_{\vec w}$ that swallows a smooth function $\phi$ defined in a neighborhood of $p$ in $\Sigma$ and spits its directional derivative $D_{\vec w}\phi$.
Note that the functional $D$ obeys the product rule:
\[D_{\vec w}(\phi\cdot\psi)=(D_{\vec w}\phi)\cdot \psi(p)+\phi(p)\cdot(D_{\vec w}\psi).
\eqlbl{eq:tangent-functional}\]

It is not hard to show that the tangent vector $\vec w$ is completely determined by the functional $D_{\vec w}$.
Moreover, tangent vectors at $p$ can be \textit{defined} as linear functionals on the space of smooth functions
that satisfy the product rule \ref{eq:tangent-functional}.

This definition grabs the key algebraic property of tangent vectors.
It might be a less intuitive way to think about tangent vectors, but it is often convenient to use in the proofs. 
For example, \ref{def:directional-derivative} becomes a tautology.

\section{Differential of map}\label{sec:differential}

Any smooth map $s$ from a surface $\Sigma$ to $\mathbb{R}^3$ can be described by its coordinate functions 
$ s(p)=(x(p),y(p),z(p))$. 
To take a directional derivative of the map we should take the  directional derivative of each of its coordinate functions.
\[D_{\vec{w}} s\df(D_{\vec{w}}x,D_{\vec{w}}y,D_{\vec{w}}z).\]

Assume $s$ maps one smooth surface $\Sigma_0$ to another $\Sigma_1$.
Let $p_0\in \Sigma_0$ and $p_1=s(p_0)$.
Note that $D_{\vec w} s\in \T_{p_1}\Sigma_1$ for any $\vec w\in \T_{p_0}$.
Indeed, choose a curve $\gamma_0$ in $\Sigma_0$ such that $\gamma_0(0)=p_0$ and $\gamma_0'(0)=\vec w$.
Observe that $\gamma_1= s\circ \gamma_0$ is a smooth curve in $\Sigma_1$. 
By the definition of the directional derivative, we have $D_{\vec w} s=\gamma_1'(0)$.
It remains to note that $\gamma_1(0)\z= p_1$.
Therefore its velocity $\gamma_1'(0)$ is in $\T_{ p_1}\Sigma_1$.

By \ref{def:directional-derivative},
$\vec w \mapsto D_{\vec w} s$ defines a linear map $\T_{p_0}\Sigma_0\z\to \T_{ p_1}\Sigma_1$;
that is,
\[D_{c\cdot \vec w} s=c\cdot D_{\vec w} s
\quad\text{and}\quad D_{\vec v+ \vec w} s=D_{\vec v} s+ D_{\vec w} s\]
for $c\in\mathbb{R}$ and $\vec v, \vec w\in\T_{p_0}$.
The map $d_{p_0} s\:\T_{p_0}\Sigma_0\z\to \T_{ p_1}\Sigma_1$ defined by
\[d_{p_0} s\:\vec w \mapsto D_{\vec w} s\]
is called the \index{differential}\emph{differential} of $s$ at~$p_0$.

The differential $d_{p_0} s$ can be described by a $2{\times}2$-matrix $M$ in orthonormal bases of $\T_{p_0}\Sigma_0$ and $\T_{p_1}\Sigma_1$.
Set $\jac_{p_0} s=|\det M|$; this value  
does not depend on the choice of orthonormal bases in $\T_{p_0}$ and $\T_{p_1}\Sigma_1$. \label{page:|L|}\index{10d@$d_p f$ (differential)}

The value $\jac_{p_0} s$ has the following geometric meaning:
if $S_0$ is a region in $\T_{p_0}$ and $S_1=(d_{p_0} s)(S_0)$, then
$\area S_1=\jac_{p_0} s \cdot \area S_0$.
This identity will play a key role in the definition of surface area.

Let $r\:\Sigma_1\to\Sigma_2$ be a smooth map between smooth surfaces $\Sigma_1$ and $\Sigma_2$.
Observe that 
\[d_{p_0}( r\circ s)=d_{p_1} r \circ d_{p_0} s .\]
It follows that
\[\jac_{p_0}( r\circ s)
=
\jac_{p_1} r\cdot\jac_{p_0} s .\eqlbl{eq:jac-composition}\]

The value $\jac_{p_0} s$ can be found using the following formulas:
\begin{align*}
\jac s
&=|s_v\times s_u|=
\\
&=\sqrt{\langle s_u, s_u\rangle\cdot\langle s_v, s_v\rangle -\langle s_u, s_v\rangle^2}=
\\
&=\sqrt{\det[(\Jac s)^\top\cdot \Jac s ]},
\end{align*}
where $\Jac s$ denotes the Jacobian matrix of $s$; see Section~\ref{sec:Multivariable calculus}.

\section{Surface integral and area}

Let $\Sigma$ be a smooth surface, and $h\:\Sigma\to\mathbb{R}$ be a smooth function.
Let us define the integral of $h$ along a region $R\subset \Sigma$.
The {}\emph{region} is defined as any Borel set $R\subset\Sigma$,
but most of the time our regions will be surfaces with boundary.

Recall that $\jac_ps$ is defined in the previous section.
Assume there is a chart $(u,v)\mapsto s(u,v)$ of $\Sigma$ defined on an open set $U\subset\mathbb{R}^2$ such that $R\subset s(U)$.
In this case, set
\[\iint_R h\df \iint_{s^{-1}(R)} h\circ s(u,v)\cdot \jac_{(u,v)}s  \cdot du\cdot dv.\eqlbl{eq:area-def}\]


By the substitution rule (\ref{thm:mult-substitution}), the right-hand side in \ref{eq:area-def} does not depend on the choice of~$s$.
That is, if $s_1\:U_1\to \Sigma$ is another chart such that $s_1(U_1)\supset R$, then 
\[\iint_{s^{-1}(R)} h\circ s(u,v)\cdot \jac_{(u,v)}s  \cdot du\cdot dv=\iint_{s_1^{-1}(R)} h\circ s_1(u,v)\cdot \jac_{(u,v)}s_1  \cdot du\cdot dv.\]
In other words, the defining identity \ref{eq:area-def} makes sense.

A general region $R$ can be subdivided into regions $R_1,R_2\dots$ such that each $R_i$ lies in the image of a chart.
After that one could define the integral along $R$ as the sum
\[\iint_Rh
\df
\iint_{R_1}h+\iint_{R_2}h+\dots\]
In case $R$ is compact or $h \geq 0$, it is straightforward to check that the value $\iint_Rh$ does not depend on the choice of such subdivision. 

The area of a region $R$ in a smooth surface $\Sigma$ is defined as the surface integral 
\[\area R=\iint_R 1.\]

The following proposition is the substitution rule for surface integral.

\begin{thm}{Area formula}\label{prop:surface-integral}
Suppose $s\:\Sigma_0\to \Sigma_1$ is a smooth parametrization of a smooth surface $\Sigma_1$ by  a smooth surface $\Sigma_0$.
Consider a region $R\subset \Sigma_0$ and a smooth function $f\:\Sigma_1\to\mathbb{R}$. Then if either $R$ is compact or $f$ is non-negative, we have
\[\iint_R (f\circ s)\cdot \jac  s=\int_{s(R)}f.  \]
In particular, if $f\equiv 1$, we have
\[\iint_R \jac  s=\area [s(R)].\]
\end{thm}


\parit{Proof.}
Follows from \ref{eq:jac-composition} and the definition of surface integral.
\qeds

Let $\Sigma_1$ and $\Sigma_2$ be two smooth surfaces in the Euclidean space.
A map $f\:\Sigma_1\to \Sigma_2$ is called \index{length-nonincreasing}\emph{length-nonincreasing} if, for any curve $\gamma$ in $\Sigma$ we have $\length\gamma\ge \length (f\circ\gamma)$. 
The following theorem provides a more natural definition of area.
Despite its intuitive statement, the proof is well beyond the scope of this book;
it is based on a generalization of the area formula that works for Lipschitz maps \cite[3.2.3]{federer}.

\begin{thm}{Theorem}\label{thm:area-axioms}
The area functional satisfies the following properties:

\begin{subthm}{thm:area-axioms:aditivity}
Sigma-aditivity: 
Let $R_1,R_2,\dots$ be a sequence of disjoint regions in a smooth surface.
Then 
\[\area (R_1\cup R_2\cup \dots)=\area R_1+\area R_2+{}\dots\]
\end{subthm}

\begin{subthm}{thm:area-axioms:monotonicity}
Monotonicity:
Let $f\:\Sigma_1\to \Sigma_2$ be length-nonincreasing map between two smooth surfaces.
Suppose that $R_1\subset \Sigma_1$ and $R_2\subset \Sigma_2$ are regions such that $f(R_1)\supset R_2$.
Then 
\[\area R_1\ge \area R_2.\]
\end{subthm}

\begin{subthm}{thm:area-axioms:unit}
Unit square has unit area. 
\end{subthm}

Moreover, the area functional is uniquely defined by these properties.
\end{thm}

\parit{Remark.}
The notion of the area of a surface is closely related to the length of a curve.
However, to define length we use a different idea --- it was defined as the least upper bound on the lengths of inscribed polygonal lines.
It turns out that an analogous definition does not work even for very simple surfaces.
The latter is shown by a classical example --- \textit{Schwarz's boot}.
This example and different approaches to the notion of the area are discussed in a popular article by Vladimir Dubrovsky~\cite{dubrovsky}.

\section{Normal vector and orientation}
A unit vector that is normal to $\T_p$ is usually denoted by $\Norm(p)$;
it is uniquely defined up to sign.\index{10nu@$\Norm$ (normal field)}

A surface $\Sigma$ is called \index{oriented surface}\emph{oriented} if it is equipped with a unit normal vector field $\Norm$;
that is, a continuous map $p\mapsto \Norm(p)$ such that $\Norm(p)\perp\T_p$ and $|\Norm(p)|=1$ for any~$p$.
The choice of the field $\Norm$ is called the {}\emph{orientation} of~$\Sigma$.
A surface $\Sigma$ is called {}\emph{orientable} if it can be oriented.
Note that each orientable surface admits two orientations: $\Norm$ and $-\Norm$.

Let $\Sigma$ be a smooth oriented surface with unit normal field $\Norm$.
The map $\Norm\:\Sigma\to \mathbb{S}^2$ defined by $p\mapsto \Norm(p)$ is called the \index{spherical!map}\emph{spherical map} or \index{Gauss map}\emph{Gauss map}.

\begin{wrapfigure}{r}{42 mm}
\vskip-7mm
\centering
\includegraphics{asy/moebius}
\vskip-1mm
\end{wrapfigure}

For surfaces, the spherical map plays essentially the same role as the tangent indicatrix for curves.

The Möbius strip shown on the diagram gives an example of a nonorientable surface --- there is no choice of a normal vector field that is continuous along the middle of the strip (it changes the sign if you try to go around).

Note that each surface is locally orientable.
In fact, each chart $s(u,v)$ admits an orientation 
\[\Norm=
\frac{s_u\times s_v}
{\left|s_u\times s_v\right|}.\]
Indeed, the vectors $s_u$ and $s_v$ are tangent vectors at $p$; 
since they are linearly independent, their vector product does not vanish, and it is perpendicular to the tangent plane.
Evidently, $(u,v)\mapsto \Norm(u,v)$ is a continuous map.
Therefore, $\Norm$ is a unit normal field. 

\begin{thm}{Exercise}\label{ex:const-normal}
Suppose a smooth surface $\Sigma$ has constant unit normal vector $\nu$.
Show that $\Sigma$ lies in a plane perpendicular to $\nu$.
\end{thm}

\begin{thm}{Exercise}\label{ex:implicit-orientable}
Let $h:\mathbb{R}^3\to\mathbb{R}$ be a smooth function with $0$ as a regular value and $\Sigma$ a surface described as a connected component of the level set $h(x,y,z)=0$.
Show that $\Sigma$ is orientable.
\end{thm}

Recall that any proper surface $\Sigma$ without boundary in the Euclidean space divides it into two connected components (\ref{clm:proper-divides}).
Therefore, we can choose the unit normal field on $\Sigma$ that points into one of the components of the complement; so, we obtain the following.

\begin{thm}{Observation}
Any smooth proper surface without boundary in the Euclidean space is orientable.
\end{thm}

In particular, it follows that the Möbius strip does not lie on a proper smooth surface without boundary.

\section{Sections}

\begin{thm}{Lemma}\label{lem:reg-section}
Let $\Sigma$ be a smooth surface.
Suppose $f\:\mathbb{R}^3\z\to\mathbb{R}$ is a smooth function.
Then for any constant $r_0$ there is an arbitrarily close value $r$ such that 
each connected component of the intersection of $\Sigma$ with the level set $L_r=\set{x\in\mathbb{R}^3}{f(x)=r}$ is a smooth curve.
\end{thm}

\parit{Proof.}
The surface $\Sigma$ can be covered by a countable set of charts $s_i\:U_i\z\to \Sigma$.
Note that the composition $f\circ s_i$ is a smooth function for any~$i$.
By Sard's lemma (\ref{lem:sard}), $r$ is a regular value of each $f\circ s_i$ for almost all $r$.

Fix such a value $r$ sufficiently close to $r_0$, and consider the level set $L_r$ described by the equation $f(x,y,z)=r$.
Any point in the intersection $\Sigma\cap L_r$ lies in the image of one of the charts.
From above, it admits a neighborhood which is a smooth curve;
hence the result.
\qeds

\begin{thm}{Advanced exercise}\label{ex:plane-section}
Let $\Pi$ be the $(x,y)$-plane and $A \subset \Pi$ be any closed subset.
Construct an open smooth surface $\Sigma$ such that $\Sigma \cap \Pi = A$.
\end{thm}

The exercise above says that plane sections of a smooth surface might look complicated.
The following corollary makes it possible to perturb the plane so that the section becomes nice.

\begin{thm}{Corollary}
Let $\Sigma$ be a smooth surface.
Then for any plane $\Pi$ there is a parallel plane $\Pi^{*}$ that lies arbitrarily close to $\Pi$ and such that the intersection $\Sigma\cap\Pi^{*}$ is a union of disjoint smooth curves.
\end{thm}



