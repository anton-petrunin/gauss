\chapter{Polygonal lines}
\label{chap:poly}

In this chapter should help to build a firm geometric intuition about curvature;
it reinterpret the curvature of curves via inscribed polygonal lines.

\section{Piecewise smooth curves}

\begin{wrapfigure}{o}{25 mm}
\vskip-0mm
\centering
\includegraphics{mppics/pic-54}
\end{wrapfigure}

Assume $\alpha\:[a,b]\to \mathbb{R}^3$ and $\beta\:[b,c]\z\to \mathbb{R}^3$ are two curves such that $\alpha(b)\z=\beta(b)$.
Note that these two curves can be combined into one $\gamma\:[a,c]\z\to \mathbb{R}^3$ by the rule 
\[\gamma(t)=
\begin{cases}
\alpha(t)&\text{if}\quad t\le b,
\\
\beta(t)&\text{if}\quad t\ge b.
\end{cases}
\]
The obtained curve $\gamma$ is called the 
\emph{concatenation} of $\alpha$ and $\beta$.
(The condition $\alpha(b)=\beta(b)$ ensures that the map $t\mapsto\gamma(t)$ is continuous.)

The same definition of concatenation can be applied if $\alpha$ and/or $\beta$ are defined on semiopen intervals 
$(a,b]$ and/or $[b,c)$.

The assumption that the intervals of definition of $\alpha$ and $\beta$ fit together is not essential --- one can concatenate any of the curves as long as the endpoint of $\alpha$ coincides with the starting point of~$\beta$.
If this is the case, then the time intervals of both curves can be shifted so that they fit together. 

If in addition $\beta(c)=\alpha(a)$, then we can do cyclic concatenation of these curves;
this way we obtain a closed curve.

If $\alpha'(b)$ and $\beta'(b)$ are defined, then the angle $\theta\z=\measuredangle(\alpha'(b),\beta'(b))$ is called the \index{external angle}\emph{external angle} of $\gamma$ at time~$b$.
If $\theta=\pi$, then we say that $\gamma$ has a \index{cusp}\emph{cusp} at  time~$b$.

A space curve $\gamma$ is called \index{piecewise smooth regular curve}\emph{piecewise smooth and regular} if it can be presented as an iterated concatenation of a finite number of smooth regular curves; if $\gamma$ is closed, then the  concatenation is assumed to be cyclic.

If $\gamma$ is a concatenation of smooth regular arcs $\gamma_1,\dots,\gamma_n$, then the total curvature of $\gamma$ is defined as a sum of the total curvatures of $\gamma_i$ and the external angles;
that is, 
\[\tc\gamma=\tc{\gamma_1}+\dots+\tc{\gamma_n}+\theta_1+\dots+\theta_{n-1}\]
where $\theta_i$ is the external angle at the joint between $\gamma_i$ and $\gamma_{i+1}$.

If $\gamma$ is closed, then the total curvature of $\gamma$ is defined by
\[\tc\gamma=\tc{\gamma_1}+\dots+\tc{\gamma_n}+\theta_1+\dots+\theta_{n},\]
where $\theta_n$ is the external angle at the joint between $\gamma_n$ and $\gamma_1$.

{

\begin{wrapfigure}{r}{23 mm}
\vskip-3mm
\centering
\includegraphics{mppics/pic-354}
\end{wrapfigure}

In particular, for a smooth regular loop $\gamma\:[a,b] \z\to \mathbb{R}^3$, the total curvature of the corresponding closed curve $\hat\gamma$ is defined as
\[\tc{\hat\gamma}\df\tc\gamma + \theta,\]
where $\theta=\measuredangle(\gamma'(a),\gamma'(b))$.

}

\section{Generalized Fenchel's theorem}

\begin{thm}{Theorem}\label{thm:gen-fenchel}
Let $\gamma$ be a closed piecewise smooth regular space curve.
Then 
\[\tc\gamma\ge2\cdot\pi.\]

\end{thm}

\parit{Proof.}
Suppose $\gamma$ is a cyclic concatenation of $n$ smooth regular arcs $\gamma_1,\dots,\gamma_n$.
Denote by $\theta_1,\dots,\theta_n$ its external angles.
We need to show that \index{10phi@$\tc{\gamma}$}
\[\tc{\gamma_1}+\dots+\tc{\gamma_n}+\theta_1+\dots+\theta_n\ge2\cdot\pi.\eqlbl{eq:gen-fenchel}\]

Consider the tangent indicatrix $\tan_i$ for each arc $\gamma_i$;
these are smooth spherical arcs.

The same argument as in the proof of Fenchel's theorem shows that the curves $\tan_1,\dots,\tan_n$ cannot lie in an open hemisphere.

Note that the spherical distance from the endpoint of $\tan_i$ to the starting point of $\tan_{i+1}$ is equal to the external angle $\theta_i$ (we enumerate the arcs modulo $n$, so $\gamma_{n+1}=\gamma_1$).
Let us connect the endpoint of $\tan_i$ to the starting point of $\tan_{i+1}$ by a short arc of a great circle in the sphere.
This way we get a closed spherical curve that is $\theta_1+\dots+\theta_n$ longer than the total length of $\tan_1,\dots,\tan_n$.

Applying the hemisphere lemma (\ref{lem:hemisphere}) to the obtained closed curve, we get that
\[\length\tan_1+\dots+\length\tan_n+\theta_1+\dots+\theta_n\ge 2\cdot\pi.\]
By \ref{obs:tantrix}, the statement follows.
\qedsf

\begin{thm}{Chord lemma}\label{lem:chord}
Let $\gamma\:[a,b]\z\to\mathbb{R}^3$
be a smooth regular arc, and
$\ell$ be its chord.
Assume $\gamma$ meets $\ell$ at angles $\alpha$ and $\beta$ at $\gamma (a)$ and $\gamma (b)$, respectively;
that is,
\[\alpha=\measuredangle(\vec w,\gamma'(a))\quad\text{and}\quad \beta=\measuredangle(\vec w,\gamma'(b)),\]
where $\vec w=\gamma(b)-\gamma(a)$.
Then 
\[\tc\gamma\ge \alpha+\beta.\eqlbl{tc>a+b}\] 

\end{thm}

\parit{Proof.}
Let us parametrize the chord $\ell$ from $\gamma(b)$ to $\gamma(a)$ and consider the cyclic concatenation $\hat\gamma$ of $\gamma$ and $\ell$.
The closed curve $\hat\gamma$ has two external angles $\pi-\alpha$ and $\pi-\beta$.

\begin{wrapfigure}{r}{45 mm}
\vskip-5mm
\centering
\includegraphics{mppics/pic-53}
\vskip0mm
\end{wrapfigure}

Since the curvature of $\ell$ vanishes, we get 
\[\tc{\hat\gamma}=\tc\gamma+(\pi-\alpha)+(\pi-\beta).\]
According to the generalized Fenchel's theorem (\ref{thm:gen-fenchel}),
$\tc{\hat\gamma}\ge 2\cdot\pi$;
hence \ref{tc>a+b} follows.
\qeds

\begin{thm}{Exercise}\label{ex:chord-lemma-optimal}
Show that the estimate in the chord lemma is optimal.

More precisely, given two points $p, q$ and two unit vectors $\vec u,\vec v$ in $\mathbb{R}^3$,
construct a smooth regular curve $\gamma$ that starts at $p$ in the direction $\vec u$ and ends at $q$ in the direction $\vec v$ such that 
$\tc\gamma$ is arbitrarily close to $\measuredangle(\vec w,\vec u)+\measuredangle(\vec w,\vec v)$, where $\vec w=q-p$.

\end{thm}

\section{Polygonal lines} 

Polygonal lines are a particular case of piecewise smooth regular curves;
each arc in its concatenation is a line segment.
Since the curvature of a line segment vanishes, the total curvature of a polygonal line is the sum of its external angles.

\begin{thm}{Exercise}\label{ex:monotonic-tc}
Let $a$, $b$, $c$, $d$, and $x$ be distinct points in $\mathbb{R}^3$.
Show that the total curvature of the polygonal line $abcd$ cannot exceed the total curvature of $abxcd$; that is, 
\[\tc {abcd} \le \tc {abxcd}.\]

Use this statement to show that any closed polygonal line has curvature at least $2\cdot\pi$.
\end{thm}



\begin{thm}{Proposition}\label{prop:inscribed-total-curvature}
Assume a polygonal line $p_0\dots p_n$ is inscribed in a smooth regular curve~$\gamma$.
Then 
\[\tc\gamma\ge \tc{p_0\dots p_n}.\]
Moreover, if $\gamma$ is closed, we allow the inscribed polygonal line $p_0\dots p_n$ to be closed.

\end{thm}

\begin{wrapfigure}[7]{o}{40 mm}
\vskip-4mm
\centering
\includegraphics{mppics/pic-55}
\vskip0mm
\end{wrapfigure}

\parit{Proof.}
Assume $p_i=\gamma(t_i)$.
Set 
\begin{align*}
\vec w_i&=p_{i+1}-p_i,& \vec v_i&=\gamma'(t_i),
\\
\alpha_i&=\measuredangle(\vec w_i,\vec v_i),&\beta_i&=\measuredangle(\vec w_{i-1},\vec v_i),
\\
\theta_i&=\measuredangle(\vec w_{i-1},\vec w_i).
\end{align*}
In the case of a closed curve, we use indexes modulo $n$;
so in this case, we have $p_{n+1}\z=p_1$.

Since the curvature of line segments vanishes, 
the total curvature of the polygonal line is the sum of external angles $\theta_i$.

By triangle inequality for angles \ref{thm:spherical-triangle-inq}, we get that
\[\theta_i\le \alpha_i+\beta_i.\]
By the chord lemma, the total curvature of the arc of $\gamma$ from $p_i$ to $p_{i+1}$ is at least $\alpha_i+\beta_{i+1}$. 

Therefore, if $\gamma$ is a closed curve, we have
\begin{align*}
\tc{p_0\dots p_n}&=\theta_1+\dots+\theta_n\le 
\\
&\le\beta_1+\alpha_1+\dots+\beta_n+\alpha_n = 
\\
&=(\alpha_1+\beta_2)+\dots+(\alpha_n+\beta_1) \le 
\\
&\le \tc\gamma.
\end{align*}
If $\gamma$ is an arc, the argument is analogous:
\begin{align*}
\tc{p_0\dots p_n}&=\theta_1+\dots+\theta_{n-1}\le 
\\
&\le\beta_1+\alpha_1+\dots+\beta_{n-1}+\alpha_{n-1} \le
\\
&\le (\alpha_0+\beta_1)+\dots+(\alpha_{n-1}+\beta_n) \le 
\\
&\le \tc\gamma.
\end{align*}
\qedsf

\begin{thm}{Exercise}\label{ex:sef-intersection}

\begin{subthm}{ex:sef-intersection:<2pi}
Draw a smooth regular plane curve $\gamma$ that has a self-intersection and such that $\tc\gamma<2\cdot\pi$.
\end{subthm}

\begin{subthm}{ex:sef-intersection:>pi} Show that if a smooth regular curve $\gamma\:[a,b]\to\mathbb{R}^3$ has a self-intersection, then $\tc\gamma>\pi$.
\end{subthm}

\end{thm}

\begin{wrapfigure}{r}{30 mm}
\vskip-2mm
\centering
\includegraphics{mppics/pic-20}
\vskip0mm
\end{wrapfigure}

\begin{thm}{Exercise}\label{ex:quadrisecant}
Suppose a closed curve $\gamma$ crosses a line at four points $a$, $b$, $c$, and~$d$.
Assume these points appear on the line in the order $a$, $b$, $c$, $d$
and they appear on the curve $\gamma$ in the order $a$, $c$, $b$,~$d$.
Show that 
\[\tc\gamma\ge 4\cdot\pi.\]

\end{thm}

Lines crossing a curve at four points as in the exercise are called \index{alternating quadrisecants}\emph{alternating quadrisecants}.
It turns out that any {}\emph{nontrivial knot} admits an alternating quadrisecant \cite{denne};
according to the exercise, the latter implies the so-called \index{F\'ary--Milnor theorem}\emph{F\'ary--Milnor theorem} --- the total curvature of any knot exceeds~$4\cdot \pi$; see \cite{petrunin-stadler} and the references therein.

The following exercise states that the inequality in \ref{prop:inscribed-total-curvature} is optimal.

\begin{thm}{Exercise}\label{ex:total-curvature=}
Show that for any regular smooth space curve $\gamma$ we have 
\[\tc\gamma=\sup\{\tc\beta\},\]
where the least upper bound is taken over all polygonal lines~$\beta$ inscribed in $\gamma$
(if $\gamma$ is closed, then we assume so is $\beta$).
\end{thm}

This exercise can be used to define the total curvature of an arbitrary curve~$\gamma$.
Namely, it can be defined as {}\textit{the least upper bound on the total curvatures of ondegenerate polygonal lines inscribed in~$\gamma$.}

It is possible to generalize most of the statements in this chapter to the (nonsmooth) curves of finite total curvature.
This theory was developed by Alexandr Alexandrov and Yuri Reshetnyak \cite{aleksandrov-reshetnyak};
a good survey on the subject is written by John Sullivan \cite{sullivan-curves}.

\section[\texorpdfstring{What if $\Phi(\gamma)=2\cdot \pi$?}{What if Φ(γ)=2·π?}]{What if $\bm{\Phi(\gamma)=2\cdot \pi}$?}

\begin{thm}{Proposition}\label{prop:fenchel=}
The equality case in Fenchel's theorem holds only for convex plane curves;
that is, the total curvature of a smooth regular space curve $\gamma$ equals $2\cdot\pi$ if and only if $\gamma$ is a convex plane curve.
\end{thm}

\parit{Proof of \ref{prop:fenchel=}.}
The if part is proved in Corollary~\ref{cor:fenchel=convex};
it remains to prove the only-if part.

Consider an inscribed quadrangle $abcd$ in~$\gamma$.
By the definition of total curvature, we have that
\begin{align*}
\tc{abcd}&=(\pi-
\measuredangle\hinge adb)+(\pi-
\measuredangle\hinge bac)+(\pi-
\measuredangle\hinge cbd)+(\pi-
\measuredangle\hinge dca)=
\\
&=4\cdot\pi -(
\measuredangle\hinge adb
+
\measuredangle\hinge bac
+
\measuredangle\hinge cbd
+
\measuredangle\hinge dca))
\end{align*}


By \ref{thm:spherical-triangle-inq},
\[
\measuredangle\hinge bac
\le
\measuredangle\hinge bad
+ 
\measuredangle\hinge bdc
\quad\text{and}\quad
\measuredangle\hinge dca\le
\measuredangle\hinge dcb
+ 
\measuredangle\hinge dba.
\eqlbl{eq:spheric-triangle}
\]

\begin{wrapfigure}[9]{r}{37 mm}
\vskip-2mm
\centering
\includegraphics{mppics/pic-56}
\vskip0mm
\end{wrapfigure}

The sum of angles in any triangle is $\pi$, so combining these inequalities, we get that 
\begin{align*}
\tc{abcd}\ge 4\cdot \pi 
&- (\measuredangle\hinge adb+\measuredangle\hinge bad+ 
\measuredangle\hinge dba)-
\\
&-(\measuredangle\hinge cbd+\measuredangle\hinge dcb 
+\measuredangle\hinge  bdc)=
\\
=2\cdot\pi.&
\end{align*}

By \ref{prop:inscribed-total-curvature},
\[\tc{abcd}\le \tc\gamma\le 2\cdot\pi.\]
Therefore, we have equalities in \ref{eq:spheric-triangle}.
It means that the point $d$ lies in the angle $abc$ 
and the point $b$ lies in the angle $cda$.
The latter means that $abcd$ is a convex plane quadrangle.

It follows that any quadrangle inscribed in $\gamma$ is a convex plane quadrangle.
Therefore, all points of $\gamma$ lie in one plane defined by three points on~$\gamma$.
Further, since any quadrangle inscribed in $\gamma$ is convex,
we get that $\gamma$ is convex as well. 
\qeds

\section{Genralized DNA theorem}\label{sec:DNA-poly}

Recall that DNA theorem (\ref{thm:DNA}) states that \textit{total curvature of closed smooth regular curve in a unit ball cannot be smaller than its length}.
In this section we prove the following polygonal analog of DNA theorem.

\begin{thm}{Genralized DNA theorem}\label{thm:DNA-poly}
Let $p_1\dots p_n$ be a closed polygonal line in a unit ball.
Then 
\[\tc{p_1\dots p_n}>\length(p_1\dots p_n).\]
\end{thm}

Note that by Exercise \ref{ex:total-curvature=}, this therem implies the original DNA theorem (\ref{thm:DNA}).
Therefore, it is indeed a generalization of DNA theorem.

\parit{Proof.}
We assume that $p_n=p_0$, $p_{n+1}=p_1$ and so on.
Denote by $\theta_i$ the external angle at $p_i$.

\begin{figure}[ht!]
\vskip-0mm
\centering
\includegraphics{mppics/pic-16}
\vskip0mm
\end{figure}

Denote by $o$ the center of the disc.
Consider a sequence of triangles 
\[\triangle q_0q_1s_0\cong \triangle p_0p_1o,\ \ \triangle q_1q_2s_1\cong \triangle p_1p_2o,\ \dots\]
such that the points $q_0,q_1\dots$ lie on one line in that order and all the $s_i$'s lie on one side from this line.

Note that 
\[|s_n-s_0|=\length (p_1\dots p_n).\]
Therefore 
\[|s_0-s_1|+\dots+|s_{n-1}-s_n|\ge \length (p_1\dots p_n).\]

Note that 
\[|q_i-s_{i-1}|=|q_i-s_i|=|p_i-o|\le 1\]
and
\[\measuredangle s_{i-1}q_is_i\le \theta_i\]
for each $i$.
It follows that
\[|s_{i-1}-s_i|<\theta_i\]
for each $i$.
Therefore,
\begin{align*}
\tc {p_1\dots p_n}
&=\theta_1+\dots+\theta_n>
\\
&> |s_{0}-s_1|+\dots |s_{n-1}-s_n|\ge 
\\
&\ge\length (p_1\dots p_n).
\end{align*}
\qedsf

\begin{thm}{Exercise}\label{ex:tc-rectifiable}
Suppose that a curve $\gamma\:[0,1]\to\mathbb{R}^3$ has bounded total curvature in the generalized sense;
that is, there is an upper bound on the total curvatures of polygonal lines inscribed in~$\gamma$.

Show that $\gamma$ is rectifiable.
Construct an example showing that the converse does not hold. 
\end{thm}

Let us mention the following related result;
it was proved by Jeffrey Lagarias and Thomas Richardson \cite{lagarias-richardso}, another proof is given by Alexander Nazarov and Fedor Petrov \cite{nazarov-petrov}.
The proofs are annoyingly tricky.

\begin{thm}{Theorem}
Let $\alpha$ be a closed curve that lies in a convex plane figure bounded by a curve $\gamma$
Then the average curvature of $\alpha$ is not less than the average curvature of $\gamma$.

\end{thm}
