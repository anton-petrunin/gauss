\chapter{Semigeodesic charts}
\label{chap:semigeodesic}

This chapter contains computational proofs of several statements discussed above, including 
\begin{itemize}
\item Proposition~\ref{prop:inj-rad} --- an alternative definition of injectivity radius.
\item Proposition~\ref{prop:gamma''} --- shortest paths are geodesics.
\item The Gauss--Bonnet formula (\ref{thm:gb}).
\end{itemize}
In addition, we discuss intrinsic isometries between surfaces and prove Gauss' remarkable theorem, stating that Gauss curvature can be defined intrinsically.

\section{Polar coordinates}

The property of the exponential map in \ref{prop:exp} can be used to define \index{polar coordinates}\emph{polar coordinates} in a smooth surface $\Sigma$ with respect to a point $p\in \Sigma$.

Namely, fix some polar coordinates $(r,\theta) \in \mathbb{R}_{\ge0} \times \mathbb{S}^1  $ on the tangent plane $\T_p$.
If $\vec v\in \T_p$ has coordinates $(r,\theta)$,
then we say that $s(r,\theta)\z=\exp_p\vec v$ is the point in $\Sigma$ with  polar coordinates $(r,\theta)$.


Since there might be many geodesics from $p$ to a given point $x$,
the point $x$ might be written differently in the polar coordinates.
However, according to Proposition~\ref{prop:exp} polar coordinates behave in  the usual way for small values of~$r$.
More precisely, the following statement holds:

\begin{thm}{Observation}\label{obs:polar}
Let $(r,\theta)\mapsto s(r,\theta)$ describe polar coordinates with respect to a point $p$ in a  smooth surface~$\Sigma$.
Then there is $r_0>0$ such that $s$ is  regular at any pair $(r,\theta)$ with $0<r<r_0$.

Moreover, if $0\le r_1,r_2<r_0$, then $s(r_1,\theta_1) \z= s(r_2,\theta_2)$ if and only if
$r_1=r_2=0$ or $r_1=r_2$ and $\theta_1\z=\theta_2+2\cdot n\cdot\pi$ for an integer~$n$.
\end{thm}

The following statement will play a key role in the formal proof that shortest paths are geodesics; see Section~\ref{sec:proof-of-gamma''}.

\begin{thm}{Gauss lemma}\label{lem:palar-perp}
Let $(r,\theta)\mapsto s(r,\theta)$ describe polar coordinates with respect to a point $p$ in a smooth surface~$\Sigma$.
Then
$s_\theta\perp s_r$
for any $r$ and~$\theta$.
\end{thm}


\parit{Proof.}
Choose $\theta$.
By the definition of the exponential map, the curve $\gamma(t)\z=s(t,\theta)$ is a unit-speed geodesic that starts at $p$;
in particular, we have the following two identities:
\begin{enumerate}[(i)]
\item Since the geodesic $\gamma$ has unit speed, we have $|s_r|=|\gamma'|=1$.
In particular,
 \[
 \tfrac{\partial}{\partial \theta}
 \langle s_r,s_r\rangle=0\]
\item Since $\gamma$ is a geodesic, we have $s_{rr}(r,\theta)=\gamma''(r)\perp\T_{\gamma(r)}$.
Therefore 
\[
\langle s_\theta, s_{rr}\rangle=0\]
\end{enumerate}
It follows that
\[
\begin{aligned}
\tfrac{\partial}{\partial r}
\langle s_\theta, s_r\rangle
&=
\langle s_{\theta r},s_r\rangle
+
\cancel{\langle s_\theta,s_{rr}\rangle}=
\\
&=
\tfrac12
\cdot 
\tfrac{\partial}{\partial \theta}
\langle s_r, s_r\rangle=
\\
&=0.
\end{aligned}
\eqlbl{eq:<s',s'>'=0}
\]
Therefore, for fixed $\theta$, the value 
$\langle  s_\theta, s_r\rangle$ does not depend on~$r$.

Note that $s(0,\theta)=p$ for any $\theta$.
Therefore,
$s_\theta(0,\theta)=0$
and, in particular,
\[\langle s_\theta, s_r\rangle=0\]
for $r=0$.
By \ref{eq:<s',s'>'=0} the same holds for any~$r$.
\qeds

\section[Shortest paths are geodesics: a formal proof]{Shortest paths are geodesics:
\\
a formal proof}

\label{sec:proof-of-gamma''}

In this section we use the construction of polar coordinates and the Gauss lemma (\ref{lem:palar-perp}) to prove Proposition~\ref{prop:gamma''}.

\parit{Proof of \ref{prop:gamma''}.}
Let $\gamma\:[0,\ell]\to\Sigma$ be a shortest path parametrized by arc-length.
Suppose $\ell=\length\gamma$ is sufficiently small, so $\gamma$ can be described in the polar coordinates at $p$;
say as $\gamma(t)=s(r(t),\theta(t))$ for some functions $t\mapsto \theta(t)$ and $t\mapsto r(t)$ with $r(0)=0$.

Note that by the chain rule, we have
\[\gamma'= s_\theta\cdot \theta'+ s_r\cdot r'
\eqlbl{eq:chain(gamma)}\]
whenever the left side is defined.
By the Gauss lemma \ref{lem:palar-perp}, $s_\theta\perp s_r$, and by definition of polar coordinates, $|s_r|=1$.
Therefore, \ref{eq:chain(gamma)} implies
\[|\gamma'(t)|\ge r'(t).\eqlbl{eq:|gamma'|=r'}\]
for any $t$ where $\gamma'(t)$ is defined.

Since $\gamma$ parametrized by arc-length, we have 
\[|\gamma(t_2)-\gamma(t_1)|\le |t_2-t_1|.\]
In particular, $\gamma$ is Lipschitz.
Therefore, by Rademacher's theorem (\ref{thm:rademacher}) the derivative $\gamma'$ is defined almost everywhere.
By~\ref{adex:integral-length:a}, we have that
\begin{align*}
\length\gamma&=\int_0^\ell|\gamma'(t)|\cdot dt\ge
\\
&\ge\int_0^\ell r'(t)\cdot dt=
\\
&=r(\ell).
\end{align*}

Note that by the definition of polar coordinates, there is a geodesic of length $r(\ell)$ from $p=\gamma(0)$ to $q=\gamma(\ell)$.
Since $\gamma$ is a shortest path, we get that $r(\ell)=\ell$, and, moreover, $r(t)=t$ for any~$t$.
This equality holds if and only if we have equality in \ref{eq:|gamma'|=r'} for almost all~$t$.
The latter implies that $\gamma$ is a geodesic.

It remains to prove the partial converse.

Fix a point $p\in\Sigma$.
Let $\epsilon>0$ be as in \ref{prop:exp}.
Assume a geodesic $\gamma$ of length less than $\epsilon$ from $p$ to $q$ does not minimize the length between its endpoints.
Then there is a shortest path from $p$ to $q$, which becomes a geodesic when it is parametrized by its arc-length.
That is, there are two geodesics from $p$ to $q$ of length smaller than $\epsilon$.
In other words there are two vectors ${\vec v},\vec w\in\T_p$ such that $|{\vec v}|<\epsilon$, $|\vec w|<\epsilon$, and 
$q=\exp_p\vec v\z=\exp_p\vec w$.
But according to \ref{prop:exp}, the exponential map $\T_p \to \Sigma$ is injective in the $\epsilon$-neighborhood of zero --- a contradiction.\qeds

\section{Gauss curvature}

Let $s$ be a smooth map from a (possibly unbounded) coordinate rectangle in the $(u,v)$-plane to a smooth surface~$\Sigma$.
The map $s$ is called \index{semigeodesic}\emph{semigeodesic} if, for any fixed $v$, the map $u\mapsto s(u,v)$ is a unit-speed geodesic and $s_u\perp s_v$ for any $(u,v)$.

Note that according to the Gauss lemma (\ref{lem:palar-perp}), the polar coordinates on $\Sigma$ are described by a semigeodesic map.

Note that we can choose a unit vector field~$\Norm=\Norm(u,v)$ that is normal to $\Sigma$ at $s(u,v)$.
For each pair $(u,v)$, consider an orthonormal frame $\Norm$, $\vec u=s_u$, and $\vec v=\Norm\times \vec u$.
Recall that since the vector $s_v(u,v)$ is tangent to $\Sigma$ at $s(u,v)$, we get $s_v\perp \vec u$ and $s_v\perp \Norm$. 
Therefore, we have that $s_v=b\cdot\vec v$ for some smooth function $(u,v)\z\mapsto b(u,v)$.

(For a fixed value $v_0$, the vector field $s_v=b\cdot\vec v$ describes the difference between $\gamma_0$ and an {}\emph{infinitesimally close} geodesic $\gamma_1\:u\mapsto s(u,v_1)$.
The fields with this property are called \index{Jacobi field}\emph{Jacobi fields} along $\gamma_0$.)


\begin{thm}{Proposition}\label{prop:jaccobi}
Suppose $(u,v)\mapsto s(u,v)$ is a semigeodesic map to a smooth surface $\Sigma$ and $\Norm$, $\vec u$, $\vec v$, and $b$ are as above.
Then 
\[b\cdot K+b_{uu}=0,\]
where $K=K(u,v)$ is the Gauss curvature of $\Sigma$ at  $s(u,v)$.

Moreover, 
\[
\langle\vec u_u,\vec u\rangle=
\langle\vec u_u,\vec v\rangle=
\langle\vec u_v,\vec u\rangle=0,
\quad\text{and}\quad
\langle\vec u_v,\vec v\rangle=b_u.
\]

\end{thm}

The proof is lengthy but straightforward.

\parit{Proof.}
Suppose $\ell=\ell(u,v)$, $m=m(u,v)$, and $n=n(u,v)$ be the components of the matrix describing the shape operator in the frame $\vec u, \vec v$;
that is,
\[
\begin{aligned}
\Shape\vec u&=\ell\cdot \vec u+ m\cdot \vec v,
&
\Shape\vec v&=m\cdot \vec u+ n\cdot \vec v.
\end{aligned}
\eqlbl{eq:Shape(u,v)}
\]
Recall that (see Section \ref{sec:More curvatures})
\[K=\ell\cdot n-m^2.\]

The following four identities are the key to the proof:
\[
\begin{aligned}
\vec u_u&=\ell\cdot \Norm,
&
\vec u_v&=\phantom{-}b_u\cdot \vec v+b\cdot m\cdot\Norm,
\\
\vec v_u&=m\cdot \Norm,
&
\vec v_v&=-b_u\cdot \vec u+b\cdot n\cdot\Norm.
\end{aligned}
\eqlbl{eq:uu-vv}
\]

Suppose the identities in \ref{eq:uu-vv} are proved already.
Then the proposition can be proved via the following calculations:
\begin{align*}
b\cdot K&=b\cdot (\ell\cdot n-m^2)=
\\
&=\langle\vec u_u,\vec v_v\rangle-\langle\vec u_v,\vec v_u\rangle=
\\
&= 
\left(
\tfrac{\partial}{\partial v}
\langle\vec u_u,\vec v\rangle
-
\cancel{\langle\vec u_{uv},\vec v\rangle}
\right)-
\left(
\tfrac{\partial}{\partial u}
\langle\vec u_v,\vec v\rangle
-
\cancel{\langle\vec u_{uv},\vec v\rangle}
\right)=
\\
&=-b_{uu}.
\end{align*}
It remains to prove the four identities in \ref{eq:uu-vv}.

\parit{$\vec u_u=\ell\cdot \Norm$.}
Since the frame $\Norm$, $\vec u$, and $\vec v$ is orthonormal, this vector identity can be rewritten as the following three scalar identities:
\[
\begin{aligned}
\langle\vec u_u,\vec u\rangle&=0,
&
\langle\vec u_u,\vec v\rangle&=0,
&
\langle\vec u_u,\Norm\rangle&=\ell.
\end{aligned}
\eqlbl{eq:uu-vv:1}
\]
Since $u\mapsto s(u,v)$ is a geodesic we have that $\vec u_u=s_{uu}(u,v)\perp\T_{s(u,v)}$.
Hence, the first two identities follow.

The remaining identity 
$\langle\vec u_u,\Norm\rangle=\ell$ follows from \ref{thm:shape-chart} and \ref{eq:Shape(u,v)}.
Indeed,
\begin{align*}
\langle\vec u_u,\Norm\rangle
&=\langle s_{uu},\Norm\rangle=
\\
&=\langle \Shape s_u,s_u\rangle=
\\
&=\langle \Shape \vec u,\vec u\rangle=
\\
&=\ell.
\end{align*}

\parit{$\vec u_v=-b_u\cdot \vec v+b\cdot m\cdot\Norm$.}
This vector identity can be rewritten as the following three scalar identities:
\[
\begin{aligned}
\langle\vec u_v,\vec u\rangle&=0,
&
\langle\vec u_v,\vec v\rangle&=b_u,
&
\langle\vec u_v,\Norm\rangle&=b\cdot m.
\end{aligned}
\eqlbl{eq:uu-vv:2}
\]

Since $\langle\vec u,\vec u\rangle=1$, we get 
$0=\tfrac{\partial}{\partial v}\langle\vec u,\vec u\rangle=2\cdot\langle\vec u_v,\vec u\rangle$; 
hence the first identity in \ref{eq:uu-vv:2} follows.

Further, since
\begin{align*}
\langle \vec v,\vec v\rangle=1,
\quad
0=\tfrac{\partial}{\partial u}\langle \vec v,\vec v\rangle=2\cdot\langle \vec v_u,\vec v\rangle,
\quad
\text{and}
\quad
s_v=b\cdot \vec v,
\end{align*}
we get
\begin{align*}
\langle\vec u_v,\vec v\rangle&=\langle s_{vu},\vec v\rangle=
\\
&=\langle \tfrac{\partial}{\partial u} (b\cdot\vec v),\vec v\rangle =
\\
&=b_u\cdot \langle \vec v,\vec v\rangle+b\cdot \langle \vec v_u,\vec v\rangle=
\\
&=b_u; 
\end{align*}
hence the second identity in \ref{eq:uu-vv:2} follows.

Finally, applying \ref{thm:shape-chart}, \ref{eq:Shape(u,v)}, and $s_v=b\cdot \vec v$, we get
\begin{align*}
\langle\vec u_v,\Norm\rangle
&=\langle s_{uv},\Norm\rangle=
\\
&=\langle \Shape s_u,s_v\rangle=
\\
&=\langle \Shape \vec u,b\cdot \vec v\rangle=
\\
&=b\cdot m.
\end{align*}

\parit{$\vec v_u=m\cdot \Norm$ and $\vec v_v=-b_u\cdot \vec u+b\cdot n\cdot\Norm$.}
Recall that $\vec v=\Norm\times \vec u$.
Therefore,
\[\begin{aligned}
\vec v_u&=\Norm_u\times \vec u+\Norm\times \vec u_u,
&
\vec v_v&=\Norm_v\times \vec u+\Norm\times \vec u_v,
\end{aligned}
\eqlbl{eq:uu-vv:3+4}
\]
Expressions for $\vec u_u$ and $\vec u_v$ in \ref{eq:uu-vv} are proved already.
Further,
\begin{align*}
-\Norm_u&=\Shape s_u=
&
-\Norm_v&=\Shape s_v=
\\
&=\Shape \vec u=
&
&=b\cdot\Shape \vec v=
\\
&=\ell\cdot\vec u+m\cdot\vec v,
&
&=b\cdot(m\cdot \vec u+ n\cdot \vec v),
\end{align*}
It remains to plug into \ref{eq:uu-vv:3+4} the expressions for $\vec u_u$, $\vec u_v$, $\Norm_u$, and~$\Norm_v$. 
\qeds

A chart $(u,v)\mapsto s(u,v)$ is called \index{semigeodesic}\emph{semigeodesic} if the map $(u,v)\z\mapsto s(u,v)$ is semigeodesic.
Note that the function $b=b(u,v)$ for a semigeodesic chart $s$ has a constant sign.
Therefore, by changing the sign of $\Norm$, we can (and always will) assume that $b>0$;
in other words, $b=|s_v|$.

\begin{thm}{Exercise}\label{ex:semigeodesc-chart}
Show that any point $p$ in a smooth surface $\Sigma$ can be covered by a semigeodesic chart.
\end{thm}

\begin{thm}{Exercise}\label{ex:inj-rad}
Let $p$ be a point on a smooth surface~$\Sigma$.
Assume $\exp_p$ is injective in the ball $B=B(0,r_0)_{\T_p}$.
Suppose the semigeodesic map $(r,\theta)\mapsto s(r,\theta)$ describes polar coordinates with respect to $p$, and the function $(r,\theta)\mapsto b(r,\theta)$ is as above.

Prove the following statements:

\begin{subthm}{ex:inj-rad:sign}
$b(r,\theta)$ does not change its sign for $0\z\le r\z<r_0$.
\end{subthm}

\begin{subthm}{ex:inj-rad:0}
$b(r,\theta)\ne0$ if $0\le r<r_0$.
\end{subthm}

\begin{subthm}{ex:inj-rad:prop:inj-rad}
Apply \ref{SHORT.ex:inj-rad:sign} and \ref{SHORT.ex:inj-rad:0} to prove \ref{prop:inj-rad}.
\end{subthm}
 
\end{thm}



A chart $(u,v)\mapsto s(u,v)$ is called \index{orthogonal chart}\emph{orthogonal} if $s_u\perp s_v$ for any $(u,v)$.
Note that any semigeodesic chart is orthogonal.

A solution of the following exercise is very similar to \ref{prop:jaccobi}.

\begin{thm}{Exercise}\label{lem:K(orthogonal)}
Let $(u,v)\mapsto s(u,v)$ be an orthogonal chart of a smooth surface~$\Sigma$.
Denote by $K=K(u,v)$ the Gauss curvature of $\Sigma$ at $s(u,v)$.
Set 
\begin{align*}
a=a(u,v)&\df|s_u|,&
b=b(u,v)&\df|s_v|,\\
\vec u=\vec u(u,v)&\df\tfrac{s_u}a,&
\vec v=\vec v(u,v)&\df\tfrac{s_v}b.
\end{align*}
Let $\Norm=\Norm(u,v)$ be the unit normal vector at $s(u,v)$.

\begin{subthm}{lem:K(orthogonal):uu-vv}
Show that 
\[
\begin{aligned}
\vec u_u
&=
-\tfrac1{b}\cdot a_v
\cdot
\vec v 
+
a\cdot \ell\cdot \Norm
,
&
\vec v_u
&=
\tfrac1{b}\cdot a_v
\cdot \vec u
+
a\cdot m\cdot \Norm
\\
\vec u_v
&=
\tfrac1{a}\cdot b_u\cdot\vec v
+
b\cdot m\cdot \Norm
,
&
\vec v_v
&=
-\tfrac1{a}\cdot b_u\cdot\vec u
+
b\cdot n\cdot \Norm,
\end{aligned}
\]
where $\ell=\ell(u,v)$, $m=m(u,v)$, and $n=n(u,v)$ be the components of the matrix describing the shape operator in the frame $\vec u, \vec v$.
\end{subthm}

\begin{subthm}{lem:K(orthogonal):K}
Show that
\[K=-\frac1{a\cdot b}\cdot
\left(
\frac{\partial}{\partial u}
\left(\frac{b_u}a \right)
+
\frac{\partial}{\partial v}
\left(\frac{a_v}b\right)
\right).\]
\end{subthm}
\end{thm}


\begin{thm}{Exercise}\label{ex:conformal}
Suppose $(u,v)\mapsto s(u,v)$ is a \index{conformal chart}\emph{conformal chart};
that is, there is a function $(u,v)\mapsto b(u,v)$ such that $b=|s_u|=|s_v|$ and $s_u\perp s_v$ for any $(u,v)$.
(The function $b$ is called a \index{conformal factor}\emph{conformal factor} of~$s$.)

Use \ref{lem:K(orthogonal)} to show that the Gauss curvature can be expressed as 
\[K=-\frac{\triangle (\ln b)}{b^2},\]
where $\triangle$ denotes the \index{Laplacian}\emph{Laplacian}; that is, $\triangle=\tfrac{\partial^2}{\partial u^2}+\tfrac{\partial^2}{\partial v^2}$, and 
 $K=K(u,v)$ is the Gauss curvature of $\Sigma$ at $s(u,v)$.
\end{thm}

\section{Rotation of a vector field}

Let $\Sigma$ be a smooth oriented surface and $\gamma$ a simple closed path in~$\Sigma$.
Suppose $\vec u$ is a field of unit tangent vectors on $\Sigma$ defined in a neighborhood of~$\gamma$.
Denote by $\vec v$ the field obtained from $\vec u$ by a counterclockwise rotation by $\tfrac{\pi}2$ of the tangent plane at each point; it could also be defined by $\vec v\df\Norm\times\vec u$.
Then the \index{rotation}\emph{rotation} of $\vec u$ around $\gamma$ is defined as the integral
\[\rot_\gamma\vec u
\df
\int_0^1\langle\vec u'(t),\vec v(t)\rangle\cdot dt.\]

\begin{thm}{Lemma}\label{lem:rotation-parallel}
Suppose $\gamma$ is a loop with the base at a point $p$ in a smooth oriented surface $\Sigma$, and $\vec u$ is a field of tangent unit vectors to $\Sigma$ defined in a neighborhood of~$\gamma$.
Then the parallel transport $\iota_\gamma\:\T_p\to\T_p$ is a {}\emph{clockwise} rotation by the angle $\rot_\gamma\vec u$.

In particular, rotations of different vector fields around $\gamma$ may only differ by a multiple of $2\cdot\pi$.
\end{thm}

\parit{Proof.}
As above, set $\vec v=\Norm\times\vec u$. 
Denote by $\vec u(t)$ and $\vec v(t)$ the vectors at $\gamma(t)$ of the fields $\vec u$ and $\vec v$, respectively.

Let $t\mapsto \vec x(t)\in \T_{\gamma(t)}$ be the parallel vector field along $\gamma$ with  $\vec x(0)\z=\vec u(0)$.
Set $\vec y=\Norm\times\vec x$.

Note that there is a continuous function $t\mapsto \phi(t)$ such that 
$\vec u(t)$ is a counterclockwise rotation of $\vec x(0)$ by angle $\phi(t)$.
Since $\vec x(0)=\vec u(0)$, we can (and will) assume that $\phi(0)=0$.

Note that
\begin{align*}
\vec u&=\cos\phi\cdot \vec x+\sin\phi\cdot \vec y
\\
\vec v&=-\sin\phi\cdot \vec x+\cos\phi\cdot \vec y
\end{align*}
It follows that 
\begin{align*}
\langle\vec u',\vec v\rangle
=\phi'\cdot\biggl(&(\sin \phi)^2\cdot \langle\vec x,\vec x\rangle+(\cos \phi)^2\cdot \langle\vec y,\vec y\rangle
\biggr)=
\\
=\phi'.\ &
\end{align*}

Therefore,
\begin{align*}
\rot_\gamma\vec u&=\int_0^1\langle\vec u'(t),\vec v(t)\rangle\cdot dt=
\\
&=\int_0^1\phi'(t)\cdot dt=
\\
&=\phi(1).
\end{align*}

Observe that 
\begin{itemize}
\item $\iota_\gamma(\vec x(0))=\vec x(1)$,

\item  $\vec x (0) = \vec u (0) = \vec u (1),$ 

\item $\vec u(1)$ is a counterclockwise rotation of $\vec x(1)$ by the angle $\phi(1)\z=\rot_\gamma\vec u$,

\end{itemize}
It follows that $\vec x(1)$ is a {}\emph{clockwise} rotation of $\vec x(0)$ by angle $\rot_\gamma\vec u$, and the result follows.
\qeds

The following lemma will play a key role in the proof of the Gauss--Bonnet formula given in the next section.

\begin{thm}{Lemma}\label{lem:rotation-semigeoesic}
Let $(u,v)\mapsto s(u,v)$ be a semigeodesic chart on a smooth surface~$\Sigma$.
Suppose a simple loop $\gamma$ bounds a disc $\Delta$ that is covered completely by~$s$.
Then 
\[\rot_\gamma\vec u+\iint_\Delta K=0,\]
where $\vec u=s_u$, and $K$ denotes the Gauss curvature of~$\Sigma$.
\end{thm}

The proof is done by a calculation using the so-called \index{Green formula}\emph{Green formula} which can be formulated the following way:

Let $D$ be a compact region in the $(u,v)$-coordinate plane that is bounded by a piecewise smooth simple closed curve $\alpha$.
Suppose $\alpha$ is oriented in such a way that $D$ lies on its left.
Then for any two smooth functions $P$ and $Q$ defined on $D$ we have
\[\iint_D (Q_u- P_v)\cdot du\cdot dv=\int_\alpha (P\cdot du+Q\cdot dv).\]

Note that the Green and Gauss--Bonnet formulas are similar --- they relate the integral along a disc and its boundary curve.
So it shouldn't be  surprising that Green helps to prove Gauss--Bonnet.

\parit{Proof.}
Let $\vec u$, $\vec v$, and $b$ be as in Section~\ref{prop:jaccobi}.
Let us write $\gamma$ in the $(u,v)$-coordinates: $\gamma(t)=s(u(t),v(t))$.
Then
\begin{align*}
\rot_\gamma \vec u&=\int_0^1\langle\vec u',\vec v\rangle\cdot dt=
\intertext{by the chain rule}
&=\int_0^1[\langle\vec u_u,\vec v\rangle\cdot u'+\langle\vec u_v,\vec v\rangle\cdot v']\cdot dt=
\intertext{by \ref{prop:jaccobi},}
&=\int_0^1b_u\cdot v'\cdot dt=
\\
&=\int_{s^{-1}\circ\gamma}b_u\cdot dv=
\intertext{by the Green formula}
&=\iint_{s^{-1}(R)}b_{uu}\cdot du\cdot dv=
\intertext{Since $s_u\perp s_v$, we get $\jac s=|s_u|\cdot|s_v|=b$; therefore}
&=\iint_R\frac{b_{uu}}{b}=
\intertext{by \ref{prop:jaccobi}, $K=-\tfrac{b_{uu}}{b}$, so we get}
&=-\iint_{R}K.
\end{align*}
\qedsf


\section{Gauss--Bonnet formula: a formal proof}\label{sec:gauss--bonnet:formal}

Recall that for a topological disc $\Delta$ in a smooth oriented surface $\Sigma$ we set
\[\GB(\Delta)=\tgc{\partial\Delta}+\iint_\Delta K-2\cdot \pi,\]
where we assume that $\partial \Delta$ is oriented in such a way that $\Delta$ lies on its left.
So the Gauss--Bonnet formula can be written as $\GB(\Delta)=0$.

\parit{Proof of the Gauss--Bonnet formula (\ref{thm:gb}).}
First, assume $\Delta$ is covered by a semigeodesic chart.
Note that the following weaker formula follows from \ref{prop:pt+tgc},
\ref{lem:rotation-parallel},
and \ref{lem:rotation-semigeoesic}:
\[\GB(\Delta)=2\cdot n\cdot \pi,
\eqlbl{eq:gb(n)}\]
where $n=n(\Delta)$ is an integer.

By \ref{ex:semigeodesc-chart}, any point can be covered by a semigeodesic chart.
Therefore, applying \ref{lem:GB-sum} finite number of times, we get that 
\ref{eq:gb(n)} holds for any disc $\Delta$ in~$\Sigma$.

Assume $\Delta$ lies in a local graph realization $z=f(x,y)$ of~$\Sigma$.
Consider the one-parameter family $\Sigma_t$ of graphs $z=t\cdot f(x,y)$ and denote by $\Delta_t$ the corresponding disc in $\Sigma_t$, so $\Delta_1=\Delta$ and $\Delta_0$ is its projection to the $(x,y)$-plane.
Note that the function $f\:t\mapsto \GB(\Delta_t)$ is continuous.
From above, $f(t)$ is a multiple of $2\cdot\pi$ for any~$t$.
It follows that $f$ is a constant function.
In particular,
\begin{align*}
\GB(\Delta)&=\GB(\Delta_0)=
\\
&=0,
\end{align*}
where the last equality follows from $\ref{prop:total-signed-curvature}$.

We proved that 
\[\GB(\Delta)=0\eqlbl{eq:GB=0}\]
if $\Delta$ lies in a graph $z=f(x,y)$ for some $(x,y,z)$-coordinate system.
Since  any point of $\Sigma$ has a neighborhood that can be covered by such a graph, applying Lemma~\ref{lem:GB-sum} as above we get that \ref{eq:GB=0} holds for any disc~$\Delta$ in~$\Sigma$.
\qeds





\section{Rauch comparison}

The following proposition is a special case of the so-called \index{Rauch comparison theorem}\emph{Rauch comparison theorem}.

\begin{thm}{Proposition}\label{prop:rauch}
Suppose $p$ is a point on a smooth surface $\Sigma$ and $r\le \inj(p)$.
Given a curve $\tilde\gamma$ in the $r$-neighborhood of $0$ in $\T_p$, set 
\[\gamma=\exp_p\circ\tilde\gamma
\quad
\text{or, equivalently}
\quad
\log_p\circ\gamma=\tilde\gamma
;\]
note that $\gamma$ is a curve in~$\Sigma$.

\begin{subthm}{prop:rauch:K>=0}
If $\Sigma$ has nonnegative Gauss curvature, then the exponential map $\exp_p$ is length nonexpanding in the $r$-neighborhood of $0$ in $\T_p$;
that is, 
\[\length \gamma\le \length \tilde\gamma\]
for any curve $\tilde\gamma$ in the open ball $B(0,r)_{\T_p}$.
\end{subthm}

\begin{subthm}{prop:rauch:K=<0}
If $\Sigma$ has nonpositive Gauss curvature, then the logarithmic map $\log_p$ is length nonexpanding in the $r$-neighborhood of $p$ in $\Sigma$;
that is, 
\[\length \gamma\ge \length \tilde\gamma\]
for any curve $\gamma$ in the open ball $B(p,r)_{\Sigma}$.
\end{subthm}

\end{thm}

\parit{Proof.}
Suppose $(r(t),\theta(t))$ are the polar coordinates of $\tilde\gamma(t)$.
Note that $\gamma(t)\z=s(r(t),\theta(t))$; that is, $(r(t),\theta(t))$ are the polar coordinates of $\gamma(t)$ with respect to $p$ on~$\Sigma$.

Set $b(r,\theta)\df|s_\theta|$.
By \ref{prop:jaccobi}
\[b_{rr}=-K\cdot b.\]
If $K\ge 0$, then $r\mapsto b(r,\theta)$ is concave
and
if $K\le 0$, then $r\mapsto b(r,\theta)$ is convex for any fixed $\theta$.
Note that $b(0,\theta)=0$, and by \ref{obs:d(exp)=1}, $b_\theta(0,\theta)=1$.
Therefore, 
\[
\begin{aligned}
b(r,\theta)\le r\quad\text{if}\quad K&\ge 0 \quad\text{and}
\\
b(r,\theta)\ge r\quad\text{if}\quad K&\le 0.
\end{aligned}
\eqlbl{eq:b-K}
\]

Without loss of generality, we may assume that $\tilde\gamma\:[a,b]\to \T_p$ is parametrized by arc-length;
in particular, it is a Lipschitz curve.
Note that
\begin{align*}
\length\tilde\gamma&=\int_a^b\sqrt{r'(t)^2+r(t)^2\cdot\theta'(t)^2}.
\intertext{Applying \ref{lem:palar-perp}, we get}
\length\gamma&=\int_a^b\sqrt{r'(t)^2+b(r(t),\theta(t))^2\cdot\theta'(t)^2}.
\end{align*}
Both statements \ref{SHORT.prop:rauch:K>=0} and \ref{SHORT.prop:rauch:K=<0} follow from \ref{eq:b-K}.
\qeds

\section{Intrinsic isometries}

Let $\Sigma$ and $\Sigma^{*}$ be two smooth surfaces in the Euclidean space.
A map $f\:\Sigma\to \Sigma^{*}$ is called \index{length-preserving}\emph{length-preserving} if, for any curve $\gamma$ in $\Sigma$, the curve $\gamma^{*}=f\circ\gamma$ in $\Sigma^{*}$ has the same length. 
If in addition $f$ is smooth and bijective, then it is called an  \index{intrinsic isometry}\emph{intrinsic isometry}. 

\begin{thm}{Exercise}\label{ex:K=0}
Suppose the Gauss curvature of a smooth surface $\Sigma$ vanishes.
Show that $\Sigma$ is \index{locally flat surface}\emph{locally flat};
that is, a neighborhood of any point in $\Sigma$ admits an intrinsic isometry to an open domain in the Euclidean plane.  
\end{thm}

\begin{thm}{Exercise}\label{ex:K=1}
Suppose a smooth surface $\Sigma$ has a unit Gauss curvature at every point.
Show that a neighborhood of any point in $\Sigma$ admits an intrinsic isometry to an open domain in the unit sphere.
\end{thm}

A simple example of intrinsic isometry can be obtained by wrapping a plane into a cylinder with the map $s\:\mathbb{R}^2 \to \mathbb{R}^3$ given by 
\[s(x,y) = (\cos x ,\sin x, y).\]
The following exercise describes a more interesting example:

\begin{thm}{Exercise}\label{ex:deformation}
Given $a>0$, show that there is a smooth unit speed curve 
$\gamma(t)=(x(t),y(t))$ with $y(t) = a\cdot \cos t$ and $y>0$.
Describe its interval of definition.

Let $\Sigma_a$ be the surface of revolution of $\gamma$ around the $x$-axis.
\begin{figure}[h!]
\vskip-0mm
\centering
\begin{lpic}[t(-0mm),b(6mm),r(0mm),l(0mm)]{asy/deformation(1.2)}
\lbl[t]{8,-.5;$a=2$}
\lbl[t]{24,3;$a=\sqrt{2}$}
\lbl[t]{41,4;$a=1$}
\lbl[t]{57,7;$a=\tfrac1{\sqrt{2}}$}
\lbl[t]{73,8;$a=\tfrac12$}
\end{lpic}
\vskip-0mm
\end{figure}
Show that the surface $\Sigma_a$ has a unit Gauss curvature at each point.

Use \ref{ex:K=1} to conclude that any small round disc $\Delta$ in $\mathbb{S}^2$ admits a smooth length-preserving deformation;
that is, there is a one-parameter family of surfaces with boundary $\Delta_t$, such that $\Delta_0=\Delta$ and $\Delta_t$ is not congruent to $\Delta_0$ for $t\ne0$.%
\footnote{In fact any disc in $\mathbb{S}^2$ admits a smooth length preserving deformation.
However, if the disc is larger than half-sphere, then the proof requires more;
it can be obtained as a corollary of two deep results of Alexandr Alexandrov: the gluing theorem and the theorem on the existence of a convex surface with an abstractly given metric \cite[p. 44]{pogorelov}.
}
\end{thm}

\section{The remarkable theorem}


\begin{thm}{Theorem}\label{thm:remarkable}
Suppose $f\:\Sigma\to \Sigma^{*}$ is an intrinsic isometry between two smooth surfaces in the Euclidean space; $p\in \Sigma$ and $p^{*}\z=f(p)\in \Sigma^{*}$.
Then 
\[K(p)_{\Sigma}=K(p^{*})_{\Sigma^{*}};\]
that is, the Gauss curvature of $\Sigma$ at $p$ is the same as the Gauss curvature of $\Sigma^{*}$ at $p^{*}$.
\end{thm}

This theorem was proved by Carl Friedrich Gauss \cite{gauss} who justifiably called it \index{Remarkable theorem}\emph{remarkable} (\index{Theorema Egregium}\emph{Theorema Egregium}).
The theorem is indeed remarkable because the Gauss curvature is defined as a product of principal curvatures which might be different at the points $p$ and $p^*$; however, according to the theorem, their products are the same.
In other words, the Gaussian curvature is an {}\emph{intrinsic invariant}.

In fact the Gauss curvature of the surface at the given point can be found {}\emph{intrinsically},
by measuring the lengths of curves in the surface.
For example, the Gauss curvature $K(p)$ appears in the following formula for the circumference $c(r)$ of a geodesic circle centered at $p$ in a surface: 
\[c(r)=2\cdot\pi\cdot r-\tfrac\pi3\cdot K(p)\cdot r^3+o(r^3).\]

The theorem implies, for example, that there is no smooth length-preserving map that sends an open region in the unit sphere to the plane.%
\footnote{Smoothness is essential --- there are plenty of non-smooth length-preserving maps from the sphere to the plane; see \cite{petrunin-yashinski} and the references therein.}
This follows since the Gauss curvature of the plane is zero and the unit sphere has Gauss curvature~1. 
In other words, there is no geographic map of any region on Earth without distortion.

\parit{Proof.}
Choose a chart $(u,v)\mapsto s(u,v)$ on $\Sigma$, and set
$s^{*} =f\circ s$.
Note that $s^{*}$ is a chart of $\Sigma^{*}$, and 
\begin{align*}
\langle s_u,s_u\rangle
&=
\langle s_u^{*}, s_u^{*}\rangle,
&
\langle s_u, s_v\rangle
&=
\langle s_u^{*}, s_v^{*}\rangle,
&
\langle s_v, s_v\rangle
&=
\langle s_v^{*}, s_v^{*}\rangle
\end{align*}
at any $(u,v)$.
Indeed, the first and the third identity hold since otherwise $f$ would not preserve the lengths of the coordinate lines $\gamma\:t\mapsto s(t,v)$ or  $\gamma\:t\z\mapsto s(u,t)$.
Taking this into account, the second identity holds since otherwise $f$ would not preserve the lengths of the coordinate lines $\gamma\:t\z\mapsto s(t,c-t)$ for some constant~$c$.

It follows that if $s$ is a semigeodesic chart,
then so is $s^{*}$.
It remains to apply \ref{prop:jaccobi} and \ref{ex:semigeodesc-chart}.
\qeds
