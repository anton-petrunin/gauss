\chapter{Definitions}
\label{chap:curves-def}

In most cases, by looking at tire tracks of a bicycle,
it is possible to say which direction it traveled and to distinguish the trails made by the front and rear wheels;
plus one can find the distance between the wheels.

Try to figure out how to do this.
This question might motivate the reader to reinvent a considerable part of the differential geometry of curves.
To learn more about this problem check the reference given after Exercise~\ref{ex:bike}.

\section{Before we start}

The notion of curve comes with many variations.
Some of them are described by nouns (path, arc, and so on)
others by adjectives (closed, open, proper, simple, smooth, and so on).
The following diagram gives an idea about four of these variations.

\vskip-0mm
\begin{figure}[h!]
\begin{minipage}{.48\textwidth}
\centering
\includegraphics{mppics/pic-110}
\end{minipage}\hfill
\begin{minipage}{.48\textwidth}
\centering
\includegraphics{mppics/pic-115}
\end{minipage}
\bigskip
\begin{minipage}{.48\textwidth}
\centering
\includegraphics{mppics/pic-120}
\end{minipage}\hfill
\begin{minipage}{.48\textwidth}
\centering
\includegraphics{mppics/pic-125}
\end{minipage}
\end{figure}
\vskip-0mm

This chapter does all these definitions.
One may skip this chapter and use it as a quick reference when reading the book.

\section{Simple curves}

In the following definition, we use the notion of metric space which is discussed in Section~\ref{sec:metric-spcaes}.
The Euclidean plane and space are the main examples of metric spaces that one should keep in mind.

Recall that a \index{real interval}\emph{real interval} is a connected subset of the real numbers.
A bijective continuous map $f\:X\to Y$ between subsets of some metric spaces is called a {}\emph{homeomorphism} if its inverse $f^{-1}\:Y\z\to X$ is continuous.  

\begin{thm}{Definition} 
A connected subset $\gamma$ in a metric space is called a \index{simple curve}\emph{simple curve} if it is \index{locally homeomorphic}\emph{locally} homeomorphic to a real interval;
that is, any point in $\gamma$ has a neighborhood in $\gamma$ that is homeomorphic to a real interval.
\end{thm}

It turns out that any simple curve $\gamma$ can be \index{parametrization}\emph{parametrized} by a real interval or circle.
That is, there is a homeomorphism $\GG\to\gamma$ 
where $\GG$ is a real interval (open, closed, or semi-open) or the circle
\[\mathbb{S}^1=\set{(x,y)\in\mathbb{R}^2}{x^2+y^2=1}.\] 
A complete proof of the latter statement is given by David Gale \cite{gale}.
The proof is not hard, but it would take us away from the main subject.
A~finicky reader may add this property in the definition of curves.

If $\GG$ is an open interval or a circle, we say that $\gamma$ is a {}\emph{curve without endpoints}; otherwise, it is 
called a {}\emph{curve with endpoints}.
In the case when $\GG$ is a circle, we say that the curve is \index{closed!curve}\emph{closed}. 
When $\GG$ is a closed interval, $\gamma$ is called a {}\emph{simple arc}.


A parametrization $\GG\to\gamma$ describes a curve completely.
We will denote a curve and its parametrization by the same letter;
for example, we may say a plane curve $\gamma$ is given with a parametrization $\gamma\:(a,b)\z\to \mathbb{R}^2$.
Note, however, that any simple curve admits many parametrizations. 

\begin{thm}{Exercise}\label{ex:9}

\begin{subthm}{ex:9:compact}
Show that the image of any continuous injective map $\gamma\:[0,1]\to\mathbb{R}^2$ is a simple arc.
\end{subthm}

\begin{subthm}{ex:9:9}
Find a continuous injective map $\gamma\:(0,1)\to\mathbb{R}^2$ such that its image is \textit{not} a simple curve.
\end{subthm}

\end{thm}


\section{Parametrized curves}\label{sec:Parametrized curves}

A \index{parametrized curve}\emph{parametrized curve} is defined as a continuous map $\gamma\:\GG\to\spc{X}$ from a circle or a real interval (open, closed, or semi-open) $\GG$ to a metric space $\spc{X}$. 
For a parametrized curve, we do not assume that $\gamma$ is injective; in other words, a parametrized curve might have {}\emph{self-intersections}.

\begin{wrapfigure}{o}{15 mm}
\vskip-0mm
\centering
\includegraphics{mppics/pic-130}
\end{wrapfigure}

If we use the term \index{curve}\emph{curve}, then it means that we do not want to specify whether it is a parametrized curve or a simple curve.

If the domain of a parametrized curve is a closed interval $[a,b]$, then the curve is called an \index{arc}\emph{arc}.
Further, if it is the unit interval $[0,1]$, then it is also called a \index{path}\emph{path}.
If in addition $p=\gamma(0)=\gamma(1)$, then $\gamma$ is called a \index{loop}\emph{loop};
in this case, the point $p$ is called the \index{base of loop}\emph{base} of the loop.

Suppose $\GG_1$ and $\GG_2$ are either real intervals or circles.
A continuous onto map $\tau\:\GG_1\to\GG_2$ is called \index{monotone map}\emph{monotone} if, for any $t\in \GG_2$, the set $\tau^{-1}\{t\}$ is connected.
Note that if $\GG_1$ and $\GG_2$ are intervals, then, by the intermediate value theorem, a monotone map is either nondecreasing or nonincreasing;
that is, our definition agrees with the standard one when $\GG_1$ and $\GG_2$ are intervals.

Suppose $\gamma_1\:\GG_1\to \spc{X}$ and $\gamma_2\:\GG_2\to \spc{X}$ are two parametrized curves such that 
$\gamma_1=\gamma_2\circ\tau$ for a monotone map $\tau\:\GG_1\to\GG_2$.
Then we say that $\gamma_2$ is \index{reparametrization}\emph{reparametrization}%
\footnote{Note that in general, $\gamma_1$ is \textit{not} a reparametrization of $\gamma_2$.
In other words, according to our definition, the described relation \textit{being a reparametrization} is not symmetric;
in particular, it is not an equivalence relation.
Usually, it is fixed by extending it to the minimal equivalence relation that includes ours \cite[2.5.1]{burago-burago-ivanov}.
But we will stick to our version.}
of $\gamma_1$ by $\tau$.


\begin{thm}{Advanced exercise}\label{aex:simple-curve}
Let $X$ be a subset of the plane.
Suppose two distinct points $p,q\in X$ can be connected by a path in~$X$.
Show that there is a simple arc in~$X$ connecting $p$ to~$q$.
\end{thm}

\section{Smooth curves}\label{sec:Smooth curves}

Curves in the Euclidean space or plane are called \index{space curve}\emph{space curves} or, respectively, \index{plane curve}\emph{plane curves}.

A parametrized space curve can be described by its coordinate functions 
\[\gamma(t)=(x(t),y(t),z(t)).\]
Plane curves can be considered as a special case of space curves with $z(t)\equiv 0$.

Recall that a real-to-real function is called \index{smooth!function}\emph{smooth} if its derivatives of all orders are defined everywhere in the domain of definition.  
If each of the coordinate functions $x(t), y(t)$, and $z(t)$ are smooth, then the parametrization is called 
\index{smooth!parametrization}\emph{smooth}.

If the \index{velocity vector}\emph{velocity vector} 
\[\gamma'(t)=(x'(t),y'(t),z'(t))\] 
does not vanish at any point, then the parametrization $\gamma$ is called \index{regular!parametrization}\emph{regular}.

A parametrized curve is called {}\emph{smooth} if its parametrization is smooth and regular.
A simple space curve is called \index{smooth!curve}\emph{smooth} if it admits a regular smooth parametrization.
Pedantically, they could be called {}\emph{regular smooth curves}. 
Smooth curves are among the main objects in differential geometry.

The following exercise shows that curves with smooth parametrizations might have corners, so they do not look smooth.

\begin{wrapfigure}{o}{15 mm}
\vskip-0mm
\centering
\includegraphics{mppics/pic-140}
\end{wrapfigure}

\begin{thm}{Exercise}\label{ex:L-shape}
Recall that (see Section \ref{sec:analysis}) that the following function is smooth: 
\[f(t)=
\begin{cases}
0&\text{if}\ t\le 0,
\\
\frac{t}{e^{1\!/\!t}}&\text{if}\ t> 0.
\end{cases}
\]

Show that $\alpha(t)=(f(t),f(-t))$ gives a smooth parametrization of the curve shown on the diagram;
it is a simple curve formed by the union of two half-axis in the plane.

Show that any smooth parametrization of this curve has a vanishing velocity vector at the origin.
Conclude that this curve is not smooth;
that is, it does not admit a regular smooth parametrization.
\end{thm}


\begin{thm}{Exercise}\label{ex:cycloid}
Describe the set of real numbers $\ell$
such that the the parametrization $\gamma_\ell (t)= (t+\ell \cdot \sin t,\ell \cdot \cos t)$, $t\in\mathbb{R}$ is

\begin{minipage}{.30\textwidth}
\begin{subthm}{ex:cycloid:smooth}
smooth; 
\end{subthm}
\end{minipage}
\hfill
\begin{minipage}{.30\textwidth}
\begin{subthm}{ex:cycloid:regular}
regular;
\end{subthm}
\end{minipage}
\hfill
\begin{minipage}{.30\textwidth}
\begin{subthm}{ex:cycloid:simple}
simple.
\end{subthm}
\end{minipage}

\end{thm}

\begin{thm}{Exercise}\label{ex:nonregular}
Find a parametrization of the cubic parabola $y=x^3$ in the $(x,y)$-plane that is smooth, but \textit{not} regular.
\end{thm}


\section{Periodic parametrizations}

Any smooth closed curve can be described by a {}\emph{periodic} parametrized curve $\gamma\: \mathbb{R}\to \spc{X}$;
that is, a curve such that $\gamma(t+\ell)=\gamma(t)$ for a fixed period $\ell>0$ and all~$t$.
For example, the unit circle in the plane can be described by the $2{\cdot}\pi$-periodic parametrization $\gamma(t)=(\cos t,\sin t)$.

{

\begin{wrapfigure}{o}{17 mm}
\vskip-4mm
\centering
\includegraphics{mppics/pic-51}
\end{wrapfigure}

Any smooth closed curve can be also described by a smooth loop.
But in general, the closed curve described by a smooth loop might fail to be smooth at its base; an example is shown on the diagram.

}

\section{Implicitly defined curves}\label{sec:implicit-curves}

Suppose $f\:\mathbb{R}^2\to \mathbb{R}$ is a smooth function; 
that is, all its partial derivatives are defined everywhere.
Let $\gamma\subset \mathbb{R}^2$ be its level set described by  the equation $f(x,y)=0$.

Assume $\gamma$ is connected.
According to the implicit function theorem (\ref{thm:imlicit}), the set $\gamma$ is a smooth simple curve if $0$ is a \index{regular!value}\emph{regular value} of~$f$; that is, the gradient $\nabla_p f$ does not vanish at any point $p\in \gamma$.
In other words, if $f(p)=0$, then   
$f_x(p)\ne 0$ or $f_y(p)\ne 0$.%
\footnote{Here $f_x$ is a shortcut notation for the partial derivative
$\tfrac{\partial f}{\partial x}$.\index{10f@$f_x$ (partial derivative)}}

The described condition is sufficient but \textit{not necessary}.
For example, zero is \textit{not} a regular value the function $f(x,y)=y^2$, but the equation $f(x,y)=0$ describes a smooth curve --- the $x$-axis.

Similarly, assume $(f,h)$ is a pair of smooth functions defined in $\mathbb{R}^3$.
The system of equations
\[\begin{cases}
   f(x,y,z)=0,
   \\
   h(x,y,z)=0
  \end{cases}
\]
defines a smooth space curve if the set $\gamma$ of solutions is connected, and $0$ is a regular value of the map $F\:\mathbb{R}^3\to\mathbb{R}^2$ defined by
\[F\:(x,y,z)\mapsto (f(x,y,z),h(x,y,z)).\]
It means that the gradients $\nabla f$ and $\nabla h$ are linearly independent at any point $p\in \gamma$.
In other words, the Jacobian matrix
\[
\Jac_pF=\begin{pmatrix}
f_x&f_y&f_z\\
h_x&h_y&h_z
\end{pmatrix}
\]
for the map $F\:\mathbb{R}^3\to\mathbb{R}^2$ has rank 2 at any point $p \in \gamma$.

If a curve $\gamma$ is described in such a way,
then we say that it is \index{implicitly defined curve}\emph{implicitly defined}.

The implicit function theorem guarantees the existence of regular smooth parametrizations for any implicitly defined curve.
However, when it comes to calculations, it is usually easier to work directly with implicit representations. 

\begin{thm}{Exercise}\label{ex:y^2=x^3}
Consider the set in the plane described by the equation
$y^2=x^3$.
Is it a simple curve?
Is it a smooth curve?
\end{thm}

\begin{thm}{Exercise}\label{ex:viviani}
Describe the set of real numbers $\ell$
such that the system of equations
\[\begin{cases}
x^2+y^2+z^2&=1
\\
x^2+\ell\cdot x+y^2&=0
\end{cases}\]
describes a smooth curve.
\end{thm}

\section{Proper, closed, and open curves}\label{sec:proper-curves}

A parametrized curve $\gamma$ in a metric space $\spc{X}$ is called \index{proper!curve}\emph{proper} if, for any compact set $K \subset \spc{X}$, the inverse image $\gamma^{-1}(K)$ is compact.

For example, the curve $\gamma(t)=(e^t,0,0)$ defined on the real line is not proper.
Indeed, the half-line $(-\infty,0]$ is not compact, but it is the inverse image of the closed unit ball around the origin.

\begin{thm}{Exercise}\label{ex:open-curve}
Suppose $\gamma\:\mathbb{R}\to\mathbb{R}^2$ is a proper curve.
Show that  $|\gamma(t)|\z\to\infty$ as $t\to\pm\infty$.
\end{thm}


Recall that a closed interval is compact (\ref{thm:Heine--Borel}) and closed subsets of a compact set are compact;
see Section~\ref{sec:topology}.
Since circles and closed intervals are compact,
it follows that closed curves and arcs are proper.

A simple curve is called proper if it admits a proper parametrization.

\begin{thm}{Exercise}\label{ex:proper-closed}
Show that a simple space curve is proper if and only if its image is a closed set.

\end{thm}

A proper simple plane curve is called \index{open!curve}\emph{open} if it is not closed and has no endpoints.
So any simple proper curve without endpoints is either closed or open.
The terms \textit{open curve} and \textit{closed curve} have nothing to do with open and closed sets.

\begin{thm}{Exercise}\label{ex:proper-curve}
Use Jordan's theorem (\ref{thm:jordan}) to show that any simple open plane curve divides the plane in two connected components.  
\end{thm}



