\chapter{Gauss--Bonnet formula}
\label{chap:gauss-bonnet}
\section{Formulation}

The following theorem was proved by Carl Friedrich Gauss \cite{gauss}
for geodesic triangles;
Pierre Bonnet and Jacques Binet independently 
generalized the statement for arbitrary curves.
\index{Gauss--Bonnet formula}

\begin{thm}{Theorem}\label{thm:gb}
Let $\Delta$ be a topological disc in a smooth oriented surface $\Sigma$ bounded by a simple piecewise smooth regular curve $\partial\Delta$.
Suppose $\partial \Delta$ is oriented in such a way that $\Delta$ lies on its left.
Then 
\[\tgc{\partial\Delta}+\iint_\Delta K=2\cdot \pi,\eqlbl{eq:g-b}\]
where $K$ denotes the Gauss curvature of~$\Sigma$.
\end{thm}

In this chapter,
we give an informal proof of this formula in a leading special case.
A formal computational proof will be given in Section~\ref{sec:gauss--bonnet:formal}.

Before going into the proofs, we suggest solving the following exercises using the Gauss--Bonnet formula.

\begin{thm}{Exercise}\label{ex:1=geodesic-curvature}
 Assume $\gamma$ is a closed simple curve with constant geodesic curvature $1$ in a smooth closed surface $\Sigma$ with positive Gauss curvature.
 Show that 
 \[\length\gamma\le 2\cdot\pi;\]
that is, the length of $\gamma$ cannot exceed the length of the unit circle in the plane.  
\end{thm}

\begin{thm}{Exercise}\label{ex:geodesic-half}
Let $\gamma$ be a simple closed geodesic on a smooth closed surface $\Sigma$ with positive Gauss curvature.
Denote by $\Norm$ the spherical map on $\Sigma$.
Show that the curve $\alpha=\Norm\circ\gamma$ divides the sphere into two regions of equal area.
Conclude that $\length \alpha\ge 2\cdot\pi.$
\end{thm}

\begin{thm}{Exercise}\label{ex:closed-geodesic}
Let $\gamma$ be a closed geodesic with self-intersections on a smooth closed surface $\Sigma$ with positive Gauss curvature.
Suppose $R$ is one of the regions that $\gamma$ cuts from~$\Sigma$.
Show that 
\[\iint_R K\le 2\cdot\pi.\]

Conclude that any two closed geodesics on $\Sigma$ have a common point.
\end{thm}

\begin{thm}{Exercise}\label{ex:self-intersections}
Let $\Sigma$ be a smooth regular sphere with positive Gauss curvature. 
Suppose $\gamma$ is a closed geodesic whose image is covered by a single coordinate chart.
Show that $\gamma$ cannot look like one of the curves on the following diagrams.

\begin{figure}[h]
\begin{minipage}{.48\textwidth}
\centering
\includegraphics{mppics/pic-46}
\end{minipage}
\hfill
\begin{minipage}{.48\textwidth}
\centering
\includegraphics{mppics/pic-47}
\end{minipage}

\medskip

\begin{minipage}{.48\textwidth}
\centering
\caption*{\textit{(easy)}}
\end{minipage}\hfill
\begin{minipage}{.48\textwidth}
\centering
\caption*{\textit{(tricky)}}
\end{minipage}
\vskip-4mm
\end{figure}

\end{thm}


\begin{wrapfigure}{r}{30 mm}
\vskip-0mm
\centering
\includegraphics{mppics/pic-471}
\end{wrapfigure}

In fact $\gamma$ also cannot look like the curve on the right, but the proof requires a more advanced technique;
a solution is given in \cite{petrunin2021}.

The following exercise gives the optimal bound on the Lipschitz constant of a convex function that guarantees that its geodesics have no self-intersections;
compare to \ref{ex:ruf-bound-mountain}.

\begin{thm}{Exercise}\label{ex:sqrt(3)}
Suppose $f\:\mathbb{R}^2\to\mathbb{R}$ is a $\sqrt{3}$-Lipschitz smooth convex function.
Show that any geodesic in the graph $z=f(x,y)$ has no self-intersections.
\end{thm}

A surface $\Sigma$ is called \index{simply-connected}\emph{simply-connected} if any closed simple curve in $\Sigma$ bounds a disc.
Equivalently any closed curve in $\Sigma$ can be continuously deformed into a \index{trivial curve}\emph{trivial curve}; that is, a curve that stands at one point all the time.

Observe that a plane and a sphere are examples of simply-connected surfaces, while the torus and the cylinder are not simply-connected.

\begin{thm}{Exercise}\label{ex:unique-geod}
Suppose $\Sigma$ is a simply-connected open surface with nonpositive Gauss curvature.

\begin{subthm}{ex:unique-geod:unique}
Show that any two points in $\Sigma$ are connected by a unique geodesic.
\end{subthm}

\begin{subthm}{ex:unique-geod:diffeomorphism}
Conclude that for any point $p\in \Sigma$,
the exponential map $\exp_p$ is a diffeomorphism from the tangent plane $\T_p$ to~$\Sigma$.
In particular, $\Sigma$ is diffeomorphic to the plane.
\end{subthm}

\end{thm}

\section{Additivity}

Let $\Delta$ be a topological disc in a smooth oriented surface $\Sigma$ bounded by a simple piecewise smooth regular curve $\partial \Delta$.
As before we suppose $\partial \Delta$ is oriented in such a way that $\Delta$ lies on its left.
Set \index{10gb@$\GB$}
\[\GB(\Delta)=\tgc{\partial\Delta}+\iint_\Delta K-2\cdot \pi,
\eqlbl{eq:GB}\]
where $K$ denotes the Gauss curvature of~$\Sigma$.
Here $\GB$ stands for Gauss--Bonnet formula;
 our goal is to show that
\[\GB(\Delta)=0.\]

\begin{thm}{Lemma}\label{lem:GB-sum}
Suppose the disc $\Delta$ is subdivided into two discs $\Delta_1$ and $\Delta_2$ by a curve $\delta$.
Then
\[
\GB(\Delta)=\GB(\Delta_1)+\GB(\Delta_2).
\]
\end{thm}

\begin{wrapfigure}[8]{r}{40 mm}
\vskip-4mm
\centering
\includegraphics{mppics/pic-1750}
\end{wrapfigure}

\parit{Proof.}
Let us subdivide $\partial \Delta$ into two curves $\gamma_1$ and $\gamma_2$ that share endpoints with $\delta$, so that
 $\Delta_i$ is bounded by the arc $\gamma_i$  and~$\delta$ for $i=1,2.$

Denote by $\phi_1$, $\phi_2$, $\psi_1$, and $\psi_2$ the angles between $\delta$ and $\gamma_i$ marked on the diagram.
Suppose the arcs $\gamma_1$, $\gamma_2$, and $\delta$ are oriented as on the diagram. 
Then
\begin{align*}
\tgc{\partial \Delta}&= \tgc{\gamma_1}-\tgc{\gamma_2}+(\pi-\phi_1-\phi_2)+(\pi-\psi_1-\psi_2),
\\
\tgc{\partial \Delta_1}&= \tgc{\gamma_1}-\tgc\delta+(\pi-\phi_1)+(\pi-\psi_1),
\\
\tgc{\partial \Delta_1}&= \tgc\delta-\tgc{\gamma_2}+(\pi-\phi_2)+(\pi-\psi_2),
\\
\iint_\Delta K&=\iint_{\Delta_1} K+\iint_{\Delta_2} K.
\end{align*}
It remains to plug in the results in the formulas for $\GB(\Delta)$, $\GB(\Delta_1)$, and $\GB(\Delta_2)$.
\qeds

\section{Spherical case}

Note that if $\Sigma$ is a plane, then the Gauss curvature vanishes;
therefore the Gauss--Bonnet formula \ref{eq:g-b} can be written as 
\[\tgc{\partial\Delta}=2\cdot \pi,\]
and it follows from \ref{prop:total-signed-curvature}.
In other words, $\GB(\Delta)=0$ for any disc~$\Delta$ in the plane.

If $\Sigma$ is the unit sphere, then $K\equiv1$;
in this case, Theorem~\ref{thm:gb} can be formulated in the following way:

\begin{thm}{Proposition}\label{prop:area-of-spher-polygon}
Let $P$ be a spherical polygon bounded by a simple closed broken geodesic $\partial P$.
Assume $\partial P$ is oriented such that $P$ lies on the left of $\partial P$.
Then 
\[\GB(P)=\tgc{\partial P}+\area P-2\cdot \pi=0.\]

Moreover, the same formula holds for any spherical region bounded by a piecewise smooth simple closed curve.
\end{thm}

This proposition will be used in the informal proof given below.

\parit{Sketch of proof.}
Suppose a spherical triangle $\Delta$ has angles 
$\alpha$, $\beta$, and~$\gamma$.
According to \ref{lem:area-spher-triangle},
\[\area\Delta=\alpha+\beta+\gamma-\pi.\]

Recall that $\partial\Delta$ is oriented so that $\Delta$ lies on its left. 
Then its oriented external angles are  $\pi-\alpha$, $\pi-\beta$, and $\pi-\gamma$.
Therefore,
\[\tgc{\partial\Delta}=3\cdot\pi-\alpha-\beta-\gamma.\]
It follows that $\tgc{\partial\Delta}+\area \Delta=2\cdot\pi$ or, equivalently,
\[\GB(\Delta)=0.\]
 
Note that we can subdivide a given spherical polygon $P$ into triangles by dividing a polygon in two on each step.
By the additivity lemma (\ref{lem:GB-sum}), we get
\[\GB(P)=0\]
for any spherical polygon~$P$.

The second statement can be proved by approximation. One has to show that
one could approximate the total geodesic curvature of  
a piecewise smooth simple closed curve by  
the total geodesic curvature of inscribed broken geodesics.
We omit the proof of the latter statement,
but it goes along the same lines as \ref{ex:total-curvature=}.
\qeds


\begin{thm}{Exercise}\label{ex:half-sphere-total-curvature}
Assume $\gamma$ is a simple piecewise smooth loop on $\mathbb{S}^2$ that divides its area in two equal parts.
Denote by $p$ the base point of~$\gamma$.
Show that the parallel transport $\iota_\gamma\:\T_p\mathbb{S}^2\to\T_p\mathbb{S}^2$ is the identity map.
\end{thm}



\section{Intuitive proof}

In this section we prove a special case of the Gauss--Bonnet formula.
This case is leading --- the general case can be proved similarly, but one has to use the signed area counted with multiplicity.

\parit{Proof of \ref{thm:gb} for open and closed surfaces with positive Gauss curvature.}
By \ref{cor:intK}, in this case, we have
\[\GB(\Delta)=\tgc{\partial\Delta}+\area[\Norm(\Delta)]-2\cdot \pi.
\eqlbl{eq:gb-area}\]

Fix $p\in \partial\Delta$;
assume the loop $\alpha$ runs along $\partial\Delta$ so that $\Delta$ lies on the left from it.
Consider the parallel translation $\iota_\alpha\:\T_p\to\T_p$ along $\alpha$.
According to \ref{prop:pt+tgc}, $\iota_\alpha$ is the clockwise rotation by the angle $\tgc{\alpha}_\Sigma$.

Set $\beta=\Norm\circ\alpha$.
By \ref{obs:parallel=}, we have $\iota_\alpha=\iota_\beta$, where $\beta$ is considered as a curve in the unit sphere.
In particular, $\iota$ is a clockwise rotation by angle $\tgc{\beta}_{\mathbb{S}^2}$.
By \ref{prop:area-of-spher-polygon} 
\[\GB(\Norm(\Delta))=\tgc{\beta}_{\mathbb{S}^2}+\area[\Norm(\Delta)]-2\cdot \pi=0.\]
Therefore, 
$\iota$ is a counterclockwise rotation by $\area[\Norm(\Delta)]$

Summarizing, the clockwise rotation by $\tgc{\alpha}_\Sigma$ is identical to a counterclockwise rotation by $\area[\Norm(\Delta)]$.
The rotations are identical if the angles are equal modulo $2\cdot\pi$.
Therefore, 
\[
\begin{aligned}
\GB(\Norm(\Delta))&=\tgc{\partial\Delta}_\Sigma+\area[\Norm(\Delta)]-2\cdot \pi=
\\
&=2\cdot n \cdot \pi
\end{aligned}
\eqlbl{eq:sum=2pin}\]
for some integer~$n$.

It remains to show that $n=0$.
By \ref{prop:total-signed-curvature}, this is so for a topological disc in a plane.
One can think of a general disc $\Delta$ as the result of a continuous deformation of a plane disc.
The integer $n$ cannot change in the process of deformation since the left-hand side in \ref{eq:sum=2pin} is continuous along the deformation; 
whence $n=0$ for the result of the deformation.
\qeds

\section{Simple geodesic}

The following theorem provides an interesting application of the Gauss--Bonnet formula; it is proved by Stephan Cohn-Vossen \cite[Satz 9 in][]{convossen}.

\begin{thm}{Theorem}\label{thm:cohn-vossen}
Any open smooth surface with positive Gauss curvature has a simple two-sided infinite geodesic.
\end{thm}

\parit{Proof.}
Let $\Sigma$ be an open surface with positive Gauss curvature and $\gamma$ a two-sided infinite geodesic in~$\Sigma$.

If $\gamma$ has a self-intersection, then it contains a simple loop;
that is, for some closed interval $[a,b]$,
the restriction $\ell=\gamma|_{[a,b]}$ is a simple loop.

By \ref{ex:convex-proper-plane}, $\Sigma$ is parametrized by an open convex region $\Omega$ in the plane.
By Jordan's theorem (\ref{thm:jordan}), $\ell$ bounds a disc in $\Sigma$; denote it by~$\Delta$.
If $\phi$ is the angle at the base of the loop, then by the Gauss--Bonnet formula,
\[\iint_\Delta K=\pi+\phi.\] 

Recall that by \ref{ex:intK:2pi}, we have
\[\iint_\Sigma K\le 2\cdot\pi.\eqlbl{intK=<2pi+}\]
Therefore, $0<\phi<\pi$; that is, $\gamma$ has no concave simple loops.

Assume $\gamma$ has two simple loops, say $\ell_1$ and $\ell_2$ that bound discs $\Delta_1$ and $\Delta_2$.
Then the discs $\Delta_1$ and $\Delta_2$ have to overlap;
otherwise, the curvature of $\Sigma$ would exceed $2\cdot\pi$  contradicting \ref{intK=<2pi+}.


It follows that after leaving $\Delta_1$, the geodesic $\gamma$ has to enter it again before creating another simple loop.
\begin{figure}[h!]
\vskip-0mm
\centering
\includegraphics{mppics/pic-1550}
\end{figure}
Consider the moment when $\gamma$ enters $\Delta_1$ again;
two possible scenarios are shown in the picture.
On the left picture, we get two nonoverlapping discs which, as we know, is impossible.
The right picture is impossible as well --- in this case, we get a concave simple loop.

It follows that $\gamma$ contains only one simple loop.
This loop cuts a disc from $\Sigma$ 
and goes around it either clockwise or counterclockwise.
This way we divide all the self-intersecting geodesics on $\Sigma$
into two sets which we call {}\emph{clockwise} and {}\emph{counterclockwise}.

Note that the geodesic $t\mapsto \gamma(t)$ is clockwise 
if and only if the same geodesic traveled backwards
$t\mapsto \gamma(-t)$
is counterclockwise.

Let us shoot a unit-speed geodesic at each direction from a given point $p\z=\gamma(0)$.
This gives a one-parameter family of geodesics $\gamma_s$ for $s\in[0,\pi]$ connecting the geodesic $t\mapsto \gamma(t)$ with $t\mapsto \gamma(-t)$; that is, $\gamma_0(t)\z=\gamma(t)$, and $\gamma_\pi(t)=\gamma(-t)$.

Observe that the subset of values $s\in [0,\pi]$ such that $\gamma_s$ is right (or left) is open.
That is, if $\gamma_s$ is right, then so is $\gamma_t$ for all $t$ sufficiently close to~$s$.
Indeed, denote by $\phi_s$ the angle of the simple loop of $\gamma_s$.
From above, we have $0<\phi_s<\pi$.
Therefore, the self-intersection at the base of the loop of $\gamma_s$ is transverse.
Since geodesic depends smoothly on the initial data 
(\ref{prop:geod-existence}), the self-intersection survives in $\gamma_t$ for all $t$ sufficiently close to~$s$.

Since $[0,\pi]$ is connected, it cannot be subdivided into two nonempty open sets.
It follows that for some $s$, the geodesic $\gamma_s$ is neither  clockwise nor counterclockwise;
that is, $\gamma_s$ has no self-intersections.
\qeds

\begin{wrapfigure}{r}{17 mm}
\vskip-3mm
\centering
\includegraphics{mppics/pic-1575}
\end{wrapfigure}

\begin{thm}{Exercise}\label{ex:cohn-vossen}
Let $\Sigma$ be an open smooth surface with positive Gauss curvature.
Suppose $\alpha\:[0,1]\z\to \Sigma$ is a smooth regular loop such that $\alpha'(0)+\alpha'(1)=0$.
Show that there is a simple two-sided infinite geodesic $\gamma$ that is tangent to $\alpha$ at some point.
\end{thm}

\section{General domains}
\index{Gauss--Bonnet formula}

The following generalization of Gauss--Bonnet formula is due to Walther von Dyck \cite{dyck}.

\begin{thm}{Theorem}\label{thm:GB-generalized}
Let $\Lambda$ be a compact domain bounded by a finite collection (possibly empty) of simple piecewise smooth regular curves $\gamma_1,\dots,\gamma_n$ in a smooth surface~$\Sigma$.
Suppose each $\gamma_i$ 
is oriented in such a way that $\Lambda$ lies on its left.
Then \[\tgc{\gamma_1}+\dots+\tgc{\gamma_n}+\iint_\Lambda K=2\cdot \pi\cdot \chi\eqlbl{eq:g-b++}\]
for an integer $\chi=\chi(\Lambda)$.

Moreover, if $\Lambda$ can be subdivided into $f$ discs by an embedded graph with $v$ vertices and $e$ edges, then $\chi=v-e+f$.
\end{thm}

The number $\chi=\chi(\Lambda)$ is called the \index{Euler characteristic}\emph{Euler characteristic} of $\Lambda$. 
Note that $\chi$ does not depend on the choice of the subdivision since the left-hand side in \ref{eq:g-b++} does not.

\parit{Proof.}
Suppose a graph with $v$ vertices and $e$ edges subdivides $\Lambda$ into $f$ discs.
Apply the Gauss--Bonnet formula for each disc, and sum up the results.
Observe that each disc and each vertex contributes $2\cdot\pi$, and each edge contributes $-2\cdot\pi$ to the total sum.
Whence \ref{eq:g-b++} follows.
\qeds



\begin{thm}{Exercise}\label{ex:g-b-chi}
Find the integral of the Gauss curvature on each of the following surfaces:

\begin{subthm}{ex:g-b-chi:torus}
Torus.
\end{subthm}

\begin{subthm}{ex:g-b-chi:moebius}
Möbius band with geodesic boundary.
\end{subthm}

\begin{subthm}{ex:g-b-chi:pair-of-pants}
Pair of pants with geodesic boundary components.
\end{subthm}

\begin{subthm}{ex:g-b-chi:two-handles}
Sphere with two handles.
\end{subthm}

\end{thm}
