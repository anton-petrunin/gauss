\chapter{Curvature}
\label{chap:curve-curvature}

In mathematics, the term curvature is used for anything that measures how much a geometric object deviates from being \textit{straight}.
For curves, the curvature is a pointwise quantitative measure of how much the curve differs from a straight line.




\section{Acceleration of a unit-speed curve}

Recall that any smooth curve can be reparametrized by its arc-length (\ref{prop:arc-length-smooth}).
The obtained parametrized curve, say $\gamma$, remains to be smooth, and it has unit speed; 
that is, $|\gamma'(s)|=1$ for all~$s$.
The following proposition states that in this case
the acceleration vector stays perpendicular to the velocity vector.

\begin{thm}{Proposition}\label{prop:a'-pertp-a''}
Assume $\gamma$ is a smooth unit-speed space curve.
Then $\gamma'(s)\perp \gamma''(s)$ for any~$s$.
\end{thm}

The \index{scalar product}\emph{scalar product} (also known as \textit{dot product}) of two vectors $\vec v$ and $\vec w$ will be denoted by $\langle \vec v,\vec w\rangle$.
Recall that the derivative of a scalar product satisfies the product rule;
that is, if $\vec v=\vec v(t)$ and $\vec w=\vec w(t)$ are smooth vector-valued functions of a real parameter $t$, then
\[\langle \vec v,\vec w\rangle'=\langle \vec v',\vec w\rangle+\langle \vec v,\vec w'\rangle.\]

\parit{Proof.}
The identity $|\gamma'|=1$ can be rewritten as $\langle\gamma',\gamma'\rangle=1$.
Differentiating both sides, we get
$2\cdot\langle\gamma'',\gamma'\rangle=\langle\gamma',\gamma'\rangle'=0$;
whence $\gamma''\perp\gamma'$.
\qeds

\section{Curvature}\label{sec:curvature}

For a unit-speed smooth space curve $\gamma$, the magnitude of its acceleration $|\gamma''(s)|$ is called its \index{10k@$\kur$ (curvature)}\index{curvature}\emph{curvature} at  time~$s$.
If $\gamma$ is simple, then we can say that $|\gamma''(s)|$ is the curvature at the point $p=\gamma(s)$ without ambiguity.
The curvature is usually denoted by $\kur(s)$ or $\kur(s)_\gamma$, and, in the case of simple curves, it might be also denoted by $\kur(p)$ or $\kur(p)_\gamma$.



\begin{thm}{Exercise}\label{ex:zero-curvature-curve}
Show that a smooth simple space curve has zero curvature at each point if and only if it is a segment of a straight line.
\end{thm}

\begin{thm}{Exercise}\label{ex:scaled-curvature}
Let $\gamma$ be a smooth simple space curve, and let $\gamma_{\lambda}$ be a scaled copy of $\gamma$ with factor $\lambda >0$;
that is, $\gamma_{\lambda}(t)=\lambda \cdot\gamma(t)$ for any~$t$.
Show that 
\[\kur(\lambda \cdot p )_{\gamma_{\lambda}}
= \frac{\kur(p)_{\gamma}}\lambda\]
for any $p \in \gamma$.
\end{thm}

\begin{thm}{Exercise}\label{ex:curvature-of-spherical-curve}
Show that any smooth unit-speed spherical curve has curvature at least 1 at each time.
\end{thm}

\section{Tangent indicatrix}\label{sec:Tangent indicatrix}

Let $\gamma$ be a smooth space curve.
The curve 
\[\tan(t)=\tfrac{\gamma'(t)}{|\gamma'(t)|};
\eqlbl{eq:tantrix}\] 
it is called a \index{tangent!indicatrix}\emph{tangent indicatrix} of~$\gamma$.
Note that $|\tan(t)|=1$ for any $t$;
that is, $\tan$ is a spherical curve.


If $s\mapsto \gamma(s)$ is a unit-speed parametrization, then $\tan(s)=\gamma'(s)$.
In this case, we have the following expression for curvature:
\[\kur(s)\z=|\tan'(s)|\z=|\gamma''(s)|.\]

For general parametrization $t\mapsto \gamma(t)$,
we have instead
\[ \kur(t)=\frac{|\tan'(t)|}{|\gamma'(t)|}.\eqlbl{eq:curvature}\]
Indeed, for an arc-length reparametrization by $s(t)$, we have $s'(t)=|\gamma'(t)|$.
Therefore,
\begin{align*}
\kur&=\left|\frac{d\tan}{ ds}\right|=
\left|\frac{d\tan}{ dt}\right|/\left|\frac{ds}{ dt}\right|=
\frac{|\tan'|}{|\gamma'|}.
\end{align*}
It follows that the indicatrix of a smooth curve $\gamma$ is smooth if the curvature of $\gamma$ does not vanish.

\begin{thm}{Exercise}\label{ex:curvature-formulas}
Use the formulas \ref{eq:tantrix} and \ref{eq:curvature} to show that 
for any smooth space curve $\gamma$ we have the following expressions for its curvature:
\setlength{\columnseprule}{0.4pt}
\begin{multicols}{2}

\begin{subthm}{ex:curvature-formulas:a} 
\[\kur=\frac{|\vec w|}{|\gamma'|^2},\]
where $\vec w=\vec w(t)$ denotes the projection of $\gamma''(t)$ to the plane normal to $\gamma'(t)$;
\end{subthm}

\begin{subthm}{ex:curvature-formulas:b}
\[\kur=\frac{|\gamma''\times \gamma'|}{|\gamma'|^{3}},\]
where $\times$ denotes the \index{vector product}\emph{vector product} (also known as \index{cross product}\emph{cross product}).
\end{subthm}
\end{multicols}
\end{thm}


\begin{thm}{Exercise}\label{ex:curvature-graph}
Apply the formulas in the previous exercise to show that if $f$ is a smooth real function,
then its graph $y=f(x)$  has curvature
\[\kur(p)=\frac{|f''(x)|}{(1+f'(x)^2)^{\frac32}}\]
at the point $p=(x,f(x))$.
\end{thm}

\begin{thm}{Advanced exercise}\label{ex:approximation-const-curvature}
Show that any smooth curve $\gamma\:\mathbb{I}\z\to\mathbb{R}^3$ with curvature at most 1 can be approximated by smooth curves with constant curvature~1.

In other words, construct a sequence $\gamma_n\:\mathbb{I}\to\mathbb{R}^3$ of smooth curves  with constant curvature 1 such that $\gamma_n(t)\to \gamma(t)$ for any $t$ as $n\to\infty$.
\end{thm}

\section{Tangents}

Let $\gamma$ be a smooth space curve and $\tan$ its tangent indicatrix.
The line thru $\gamma(t)$ in the direction of $\tan(t)$ is called the \index{tangent!line}\emph{tangent line} to $\gamma$  at~$t$.
Any vector proportional to $\tan(t)$ is called the \index{tangent!vector}\emph{tangent} to $\gamma$ at~$t$.

The tangent line could be also defined as a unique line that has \index{order of contact}\emph{first-order contact} with $\gamma$ at $s$;
that is, $\rho(\ell)=o(\ell)$, where $\rho(\ell)$ denotes the distance from $\gamma(s+\ell)$ to the line.

\begin{thm}{Advanced exercise}\label{ex:no-parallel-tangents}
Construct a smooth closed space curve without parallel tangent lines.
\end{thm}

We say that a smooth curve $\gamma_1$ at $s_1$ is \index{tangent!curves}\emph{tangent} to a smooth curve $\gamma_2$ at $s_2$
if $\gamma_1(s_1)=\gamma_2(s_2)$ and the tangent line of $\gamma_1$ at $s_1$ coincides with the tangent line of $\gamma_2$ at $s_2$;
if both curves are simple we can also say that they are tangent at the point $p=\gamma_1(s_1)\z=\gamma_2(s_2)$ without ambiguity.

\section{Total curvature}

Let $\gamma\:\mathbb{I}\to\mathbb{R}^3$ be a smooth unit-speed curve and $\tan$ its tangent indicatrix.
The integral 
\[\tc\gamma\df\int_{\mathbb{I}}\kur(s)\cdot ds\]
is called the \index{total!curvature}\emph{total curvature of}\label{page:total curvature of:smooth-def}
$\gamma$, and it measures the total change of direction along $\gamma$.

Rewriting the above integral using a change of variables produces a formula for a general parametrization $t\mapsto \gamma(t)$:
\[\tc\gamma\df\int_{\mathbb{I}}\kur(t)\cdot|\gamma'(t)| \cdot dt.
\eqlbl{eq:tocurv}\]


\begin{thm}{Exercise}\label{ex:helix-curvature}
Find the curvature of the helix 
\[\gamma_{a,b}(t)=(a\cdot \cos t,a\cdot \sin t,b\cdot t),\]
its tangent indicatrix, and the total curvature of its arc $\gamma_{a,b}|_{[0,2\cdot\pi]}$.
\end{thm}

Note that for a unit-speed smooth curve,
the speed of its tangent indicatrix equals its curvature.
Therefore, \ref{ex:integral-length} implies the following.

\begin{thm}{Observation}\label{obs:tantrix}
The total curvature of a smooth curve is the length of its tangent indicatrix.
\end{thm}

\begin{thm}{Fenchel's theorem}
\label{thm:fenchel}
\index{Fenchel's theorem}
The total curvature of any closed smooth space curve is at least $2\cdot\pi$.
\end{thm}

\parit{Proof.}
Fix a closed smooth space curve~$\gamma$.
We can assume that $\gamma$ is described by a unit-speed loop $\gamma\:[a,b]\to \mathbb{R}^3$;
in this case, $\gamma(a)=\gamma(b)$ and $\gamma'(a)=\gamma'(b)$.

Consider its tangent indicatrix $\tan=\gamma'$.
Recall that $|\tan(s)|=1$ for any $s$; that is, $\tan$ is a closed spherical curve.

Let us show that $\tan$ cannot lie in a hemisphere.
Assume the contrary; without loss of generality, we can assume that it lies in the northern hemisphere defined by the inequality $z>0$ in $(x,y,z)$-coordinates.
In other words, if $\gamma(t)=(x(t), y(t), z(t))$, then $z'(t)>0$ for any~$t$.
Therefore,
\[z(b)-z(a)=\int_a^b z'(s)\cdot ds>0.\]
In particular, $\gamma(a)\ne \gamma(b)$, a contradiction.

Applying the observation (\ref{obs:tantrix}) and the hemisphere lemma (\ref{lem:hemisphere}), we get  
\[\tc\gamma=\length \tan\ge2\cdot\pi.\]
\qedsf

\begin{thm}{Exercise}\label{ex:length>=2pi}
Show that a closed space curve $\gamma$ with curvature at most~$1$ cannot be shorter than the unit circle;
that is, 
\[\length\gamma\ge 2\cdot \pi.\]

\end{thm}


\begin{thm}{Advanced exercise}\label{ex:gamma/|gamma|}
Suppose $\gamma$ is a smooth space curve that does not pass thru the origin.
Consider the spherical curve $\sigma$ defined by $\sigma(t)\z\df\frac{\gamma(t)}{|\gamma(t)|}$.
Show that 
\[\length \sigma< \tc\gamma+\pi.\]
Moreover, if $\gamma$ is closed, then
\[\length \sigma\le \tc\gamma.\]
\end{thm}

Note that the last inequality gives an alternative proof of Fenchel's theorem.
Indeed, without loss of generality, we can assume that the origin lies on a chord of~$\gamma$.
In this case, the closed spherical curve $\sigma$ goes from a point to its antipode and comes back; 
it takes length $\pi$ each way, 
whence 
\[\length\sigma\ge 2\cdot\pi.\]

Recall that the curvature of a spherical curve is at least $1$
(see \ref{ex:curvature-of-spherical-curve}).
In particular, the length of a spherical curve cannot exceed its total curvature.
The following theorem shows that the same inequality holds for \textit{closed} curves in a unit ball.

\begin{thm}{DNA theorem}\label{thm:DNA}
Let $\gamma$ be a smooth closed curve that lies in a unit ball.
Then 
\[\tc\gamma\ge \length\gamma.\]

\end{thm}

The 2-dimensional case of this theorem was proved by Istv\'{a}n F\'{a}ry \cite{fary1950}.
It was generalized by Don Chakerian \cite{chakerian1962} to higher dimensions.
This theorem has many interesting and very different proofs;
a number of them are collected by Serge Tabachnikov~\cite{tabachnikov}.
The following exercise will guide you thru another proof by Don Chakerian \cite{chakerian1964}.
Yet another proof is given in Section~\ref{sec:DNA-poly}.

\begin{thm}{Exercise}\label{ex:DNA}
Let $\gamma\:[0,\ell]\to\mathbb{R}^3$ be a smooth unit-speed closed curve that lies in the unit ball; that is, $|\gamma|\le 1$.

\begin{subthm}{ex:DNA:c''c>=k}
Show that 
\[\langle\gamma''(s),\gamma(s)\rangle\ge-\kur(s)\]
for any~$s$.
\end{subthm}

\begin{subthm}{ex:DNA:int>=length-tc}
Use part \ref{SHORT.ex:DNA:c''c>=k} to show that 
\[\int_0^\ell\langle\gamma(s),\gamma'(s)\rangle'\cdot ds\ge
\ell-\tc\gamma.\]

\end{subthm}

\begin{subthm}{ex:DNA:end}
Suppose $\gamma(0)=\gamma(\ell)$ and $\gamma'(0)=\gamma'(\ell)$.
Show that 
\[\int_0^\ell\langle\gamma(s),\gamma'(s)\rangle'\cdot ds=0.\]
Use this equality together with  part \ref{SHORT.ex:DNA:int>=length-tc} to prove \ref{thm:DNA}.
\end{subthm}
\end{thm}

\section{Convex curves}

In this section, we show that the tangent indicatrix of a convex curve rotates monotonically. 
The following exercise provides a key observation for the proof.

\begin{thm}{Exercise}\label{ex:tangent-support}
Let $\gamma$ be a convex plane curve;
denote by $F$ the convex set bounded by $\gamma$.
Show that a line $\ell$ is tangent to $\gamma$ at point $p$ if and only if it \index{supporting!plane}\emph{supports} $F$ at $p$;
that is, $\ell\ni p$ and $F$ lies in a closed half-plane bounded by $\ell$.
\end{thm}

Recall that a map is monotone if the inverse image of any point in the target space is a nonempty connected set.

\begin{thm}{Proposition}\label{prop:convex-monotone}
Let $\gamma$ be a smooth convex plane curve.

\begin{subthm}{prop:convex-monotone:closed}
If $\gamma$ is closed, then its tangent indicatrix $\tan$ defines a monotone map $\mathbb{S}^1\to \mathbb{S}^1$;
in other words, $\tan$ is a parametrization of the unit circle.
\end{subthm}

\begin{subthm}{prop:convex-monotone:open}
If $\gamma$ is open, then its tangent indicatrix $\tan$ defines a monotone map to an interval in a unit semicircle;
in other words, $\tan$ is a parametrization of an interval in a unit semicircle.
\end{subthm}

\end{thm}

The following corollary says that \textit{for convex curves we have equality in Fenchel's theorem} (\ref{thm:fenchel}).
Latter we, in \ref{prop:fenchel=}, we will show that it holds \textit{only} for convex curves.

\begin{thm}{Corollary}\label{cor:fenchel=convex}
Let $\gamma$ be a convex plane curve.

\begin{subthm}{}
If $\gamma$ is closed, then $\tc\gamma=2\cdot\pi$.
\end{subthm}

\begin{subthm}{}
If $\gamma$ is open, then $\tc\gamma\le\pi$.
\end{subthm}

\end{thm}

\parit{Proof.}
Follows from \ref{prop:convex-monotone:closed}, \ref{obs:tantrix} and \ref{ex:integral-length-0}.
\qeds

\begin{wrapfigure}{r}{32 mm}
\vskip-0mm
\centering
\includegraphics{mppics/pic-3500}
\vskip0mm
\end{wrapfigure}

\parit{Proof of \ref{prop:convex-monotone}; \ref{SHORT.prop:convex-monotone:closed}.}
Since $\gamma$ is closed, it bounds a compact convex set $F$.
We may assume that $F$ lies on the left from $\gamma$.

Choose a unit vector $\vec u$.
Let $(x,y)$ be the coordinates of the plane with $x$-axis in the direction of $\vec u$.

From Exercise~\ref{ex:tangent-support}, it follows that  $\vec u=\tan(s)$ if and only if $p=\gamma(s)$ is a minimum point of $y$-coordinate function on~$F$.
Indeed suppose $p$ is a minimum point.
Let $\ell$ be the horizontal line thru $p$.
Then $\ell$ supports $F$ at $p$.
By the exercise, $\ell$ is tangent to $\gamma$ at $p$.
Since $F$ lies on the left from $\gamma$, we get that $\tan(s)=\vec u$.
And the other way around, if $\tan(s)=\vec u$ then, the tangent line at $p=\gamma(s)$ is horizontal,
and, by the exercise, it supports $F$.
Since $F$ lies on the left from $\gamma$, we get that the $y$-coordinate on $F$ has a minimum at $p$.

Since $F$ is compact, there is a minimum point $p=\gamma(s)$ for the $y$-coordinate.
The point $p$ might be unique, in this case, $\tan^{-1}\{\vec u\}=s$,
or $f$ might have a line segment of minimal points in $\gamma$, in this case, $\tan^{-1}\{\vec u\}$ is an arc of $\mathbb{S}^1$.
It follows that $\tan\:\mathbb{S}^1\to \mathbb{S}^1$ is monotone.

\parit{\ref{SHORT.prop:convex-monotone:open}} 
The same argument in \ref{SHORT.prop:convex-monotone:closed} shows that $\tan$ is a monotone map to its image.
Evidently, the image is a connected set in $\mathbb{S}^1$.
It remains to show that the image lies in a semicircle;
in other words, there is a unit vector $\vec w$ such that 
\[\measuredangle(\vec w,\tan(s))>\tfrac\pi2
\quad\text{for any}\quad
s.
\eqlbl{eq:<(w,tan).pi/2}
\]

\begin{figure}[ht!]
\centering
\includegraphics{mppics/pic-3502}
\end{figure}

Since $\gamma$ is open, it bounds an unbounded convex closed region $F$.
As before we assume that $F$ lies on the left from $\gamma$.
Note that there is a half-line, say $h$, that lies in~$F$.

Indeed, we can assume that the origin $o$ lies in~$F$.
Consider a sequence of points $q_n\in F$ such that $|q_n|\z\to \infty$ as $n\to \infty$.
Denote by $\vec v_n$ the unit vector in the direction of $q_n$; that is $\vec v_n=\tfrac{q_n}{|q_n|}$.

Since the unit circle is compact, we can pass to a subsequence of $q_n$ such that $\vec v_n$ converges to a unit vector, say $\vec v$.
Since $F$ is closed, the half-line $h$ from the origin in the direction of $\vec v$ lies in $F$.

Denote by $\vec w$ the counterclockwise rotation of $\vec v$ by angle $\tfrac\pi 2$.
Suppose $\vec u=\tan (s)$ for some $s$;
choose $(x,y)$-coordinates so that $x$-axis points in the direction of $\vec u$.
The same argument as in \ref{SHORT.prop:convex-monotone:closed} shows that 
$p=\gamma(s)$ is a minimum pint in $F$ of the $y$-coordinate.
It follows that $h$ and therefore $\vec v$ point in the closed upper half-plane of the $(x,y)$-plane.
The latter is equivalent to \ref{eq:<(w,tan).pi/2}.
\qeds

\section{Bow lemma}

\begin{wrapfigure}{r}{39 mm}
\vskip-4mm
\centering
\includegraphics{mppics/pic-3510}
\vskip0mm
\end{wrapfigure}

The following lemma was proved by Axel Schur \cite{shur};
it is a differential-geometric analog of the so-called {}\emph{arm lemma} of Augustin-Louis Cauchy.
The arm lemma says that 
\textit{if $p_0\dots p_n$ is a convex plane polygon and $q_0\dots q_n$ is a polygonal line in the space such that 
\begin{align*}
|p_i-p_{i-1}|&=|q_i-q_{i-1}|,
\\
\measuredangle\hinge{p_i}{p_{i+1}}{p_{i-1}}&\le \measuredangle\hinge{q_i}{q_{i+1}}{q_{i-1}}
\end{align*}
for each $i$, then $|p_0-p_n|\le |q_0-q_n|$.}


\begin{thm}{Lemma}\label{lem:bow}\index{bow lemma}
Let $\gamma_1\:[a,b]\to\mathbb{R}^2$ and $\gamma_2\:[a,b] \to\mathbb{R}^3$ be two smooth unit-speed curves.
Suppose $\kur(s)_{\gamma_1}\ge\kur(s)_{\gamma_2}$ for any $s$ 
and the curve
$\gamma_1$ is an arc of a convex curve; that is, it runs in the boundary of a convex plane figure.
Then the distance between the endpoints of $\gamma_1$ cannot exceed the  distance between the endpoints of $\gamma_2$; that is,
\[|\gamma_1(b)-\gamma_1(a)|\le |\gamma_2(b)-\gamma_2(a)|.\]

\end{thm}

The following exercise states that the condition that $\gamma_1$ is a convex arc is necessary.
It is instructive to do this exercise before reading the proof of the lemma.

{\sloppy 

\begin{thm}{Exercise}\label{ex:anti-bow}
Construct two simple smooth unit-speed plane curves $\gamma_1,\gamma_2\:[a,b]\to\mathbb{R}^2$ such that $\kur(s)_{\gamma_1}>\kur(s)_{\gamma_2}>0$ for any $s$ and
\[|\gamma_1(b)-\gamma_1(a)|> |\gamma_2(b)-\gamma_2(a)|.\]
\end{thm}

}

\parit{Proof.}
We can assume that $\gamma_1(a)\ne \gamma_1(b)$;
otherwise the statement is evident.
By convexity, the curve $\gamma_1$ lies on one side of the line $\ell$ thru $\gamma(a)$ and $\gamma(b)$;
we may assume $\ell$ is horizontal and $\gamma_1$ lies below $\ell$.

Let $s_0$ be the lowest point on $\gamma_1$;
that is, $\gamma_1(s_0)$ has minimal $y$-coordinate.

Denote by $\tan_1$ and $\tan_2$ the tangent indicatrixes of $\gamma_1$ and $\gamma_2$, respectively.
Consider two unit vectors 
\[\vec u_1=\tan_1(s_0)=\gamma_1'(s_0)
\quad\text{and}\quad
\vec u_2=\tan_2(s_0)=\gamma_2'(s_0).\]
Note that $\gamma_1(b)$ lies in the direction of $\vec u_1$ from $\gamma_1(a)$.

Lets us show that 
\[\measuredangle(\gamma'_1(s),\vec u_1)\ge \measuredangle(\gamma'_2(s),\vec u_2)
\eqlbl{<gamma',u}
\]
for any $s$.
We will prove it for $s\le s_0$; the case $s\ge s_0$ is analogous.

Note that
\[\measuredangle(\gamma'_1(s),\vec u_1)=\measuredangle(\tan_1(s),\vec u_1)=\length (\tan_1|_{[s,s_0]}).\eqlbl{<=length}\]
for any $s\le s_0$.
Indeed, by \ref{ex:tangent-support}, the $y$-coordinate of $\gamma_1$ is nonincreasing in the interval $[a,s_0]$.
Therefore, the arc $\tan_1|_{[a,s_0]}$ lies in one of the unit semicircles with endpoints $\vec u_1$ and $-\vec u_1$.
It remains to apply \ref{obs:tantrix}, \ref{prop:convex-monotone}, and \ref{ex:integral-length-0}.

By \ref{obs:S2-length}, we also have 
\[\measuredangle(\gamma'_2(s),\vec u_2)=\measuredangle(\tan_2(s),\vec u_2)\le \length (\tan_2|_{[s,s_0]}).\eqlbl{<=<length}\]


{

\begin{wrapfigure}{r}{39 mm}
\vskip0mm
\centering
\includegraphics{mppics/pic-57}
\vskip0mm
\end{wrapfigure}

Further,
\begin{align*}
\length (\tan_1|_{[s,s_0]})
&=\int_s^{s_0}|\tan_1'(t)|\cdot d t=
\\
&=\int_s^{s_0}\kur_1(t)\cdot d t\ge
\\
&\ge
\int_s^{s_0}\kur_2(t)\cdot d t=
\\
&=\int_s^{s_0}|\tan_2'(t)|\cdot d t= 
\\
&=\length (\tan_2|_{[s,s_0]}).
\end{align*}
This inequality, together with \ref{<=length} and \ref{<=<length}, implies \ref{<gamma',u}.
}

Since $1=|\gamma_1'(s)|=|\gamma_2'(s)|=|\vec u_1|=|\vec u_2|$,
we have 
\[\langle\gamma'_1(s),\vec u_1\rangle=\cos \measuredangle(\gamma'_1(s),\vec u_1)
\quad\text{and}\quad
\langle\gamma'_2(s),\vec u_2\rangle=\cos \measuredangle(\gamma'_2(s),\vec u_2).
\]
The cosine is decreasing in the interval $[0,\pi]$; therefore, \ref{<gamma',u} implies 
\[\langle\gamma'_1(s),\vec u_1\rangle\le \langle\gamma'_2(s),\vec u_2\rangle\eqlbl{<gamma',u>}\]
for any~$s$.

Further, since $\gamma_1(b)$ lies in the direction of $\vec u_1$ from $\gamma_1(a)$, we have that
\[|\gamma_1(b)-\gamma_1(a)|=\langle \vec u_1,\gamma_1(b)-\gamma_1(a)\rangle.\]
Since $\vec u_2$ is a unit vector, we have that
\[|\gamma_2(b)-\gamma_2(a)|\ge\langle \vec u_2,\gamma_2(b)-\gamma_2(a)\rangle.\]

Integrating \ref{<gamma',u>}, we get 
\begin{align*}
|\gamma_1(b)-\gamma_1(a)|&=\langle \vec u_1,\gamma_1(b)-\gamma_1(a)\rangle=
\\
&=
\int_a^b\langle \vec u_1,\gamma'_1(s)\rangle\cdot ds \le 
\\
&\le\int_a^b\langle \vec u_2,\gamma'_2(s)\rangle\cdot ds 
=
\\
&=\langle \vec u_2,\gamma_2(b)-\gamma_2(a)\rangle
\le
\\
&\le |\gamma_2(b)-\gamma_2(a)|.
\end{align*}
\qedsf

\begin{thm}{Exercise}\label{ex:length-dist}
Let $\gamma\:[a,b]\to \mathbb{R}^3$ be a smooth curve and $0\z<\theta\z\le\tfrac\pi2$.
Assume 
\[\tc\gamma\le 2\cdot\theta.\]

\begin{subthm}{ex:length-dist:>} Show that
\[|\gamma(b)-\gamma(a)|> \cos\theta\cdot\length\gamma.\]
\end{subthm}

\begin{subthm}{ex:length-dist:sef-intersection:>pi}
Use part \ref{SHORT.ex:length-dist:>} to give another solution of \ref{ex:sef-intersection:>pi}. 
\end{subthm}

\begin{subthm}{ex:length-dist:=} Show that the inequality in \ref{SHORT.ex:length-dist:>} is optimal; that is, given 
$\theta$ there is a smooth curve $\gamma$ such that $\tc\gamma\le 
2\cdot\theta$, and $\frac{|\gamma(b)-\gamma(a)|}{\length\gamma}$ is arbitrarily 
close to $\cos\theta$.
\end{subthm}

\end{thm}

The statement in the following exercise is generally attributed to Hermann Schwarz \cite{shur}.

\begin{thm}{Exercise}\label{ex:schwartz}
Let $p$ and $q$ be points on a unit circle dividing it in two arcs with lengths $\ell_1<\ell_2$.
Suppose a space curve $\gamma$ connects $p$ to $q$ and has curvature at most $1$.
Show that either
\[\length \gamma\le \ell_1
\quad\text{or}\quad
\length \gamma\ge \ell_2.
\]
\end{thm}

The following exercise generalizes \ref{ex:length>=2pi}.

\begin{thm}{Exercise}\label{ex:loop}
Suppose $\gamma\:[a,b]\to \mathbb{R}^3$ is a smooth loop with curvature at most~1.
Show that 
\[\length\gamma\ge2\cdot\pi.\]

\end{thm}

\begin{thm}{Exercise}\label{ex:bow-upper}
Let $\kur$ be a smooth nonnegative function defined on $[0,\ell]$.
Show that there is a smooth unit-speed curve $\gamma\:[0,\ell]\to\mathbb{R}^3$ with curvature function $\kur$ such that the distance $|\gamma(\ell)-\gamma(0)|$ is arbitrarily close to $\ell$.
\end{thm}

\begin{thm}{Advanced exercise}\label{ex:gromov-twist}
Let $\gamma$ be a closed smooth space curve with curvature at most 2.
Suppose $|\gamma(t)|\le 1$ for any $t$.
Show that if $\gamma(t)\ne 0$, then 
\[|\gamma(t)| \le \sin [\alpha(t)]\]
where $\alpha(t)$ denotes the angle between  $\gamma(t)$ and $\gamma'(t)$.
\end{thm}

