\chapter{Supporting curves}
\label{chap:supporting-curves}

When two plane curves are tangent and do not cross, it is possible to control the signed curvature of one of them in terms of the signed curvature of other one.
In this chapter, we show how to use this idea to control the global behavior of plane curves.

\section{Cooriented tangent curves}

Suppose $\gamma_1$ and $\gamma_2$ are smooth regular plane curves.
Recall that the curves $\gamma_1$ and $\gamma_2$ are tangent at the  time parameters $t_1$ and $t_2$
if $\gamma_1(t_1)=\gamma_2(t_2)$
and they share the tangent line at these time parameters.
In this case, the point $p=\gamma_1(t_1)=\gamma_2(t_2)$ is called a \index{point!of tangency}\emph{point of tangency} of the curves.

Note that if $\gamma_1$ and $\gamma_2$ are tangent at the time parameters $t_1$ and $t_2$, 
then the velocity vectors $\gamma_1'(t_1)$ and $\gamma_2'(t_2)$ are parallel.
\begin{figure}[!ht]
\vskip-0mm
\centering
\includegraphics{mppics/pic-85}
\vskip0mm
\end{figure}
If $\gamma_1'(t_1)$ and $\gamma_2'(t_2)$ have the same direction, we say that the curves are \index{cooriented!curves}\emph{cooriented},
if their directions are opposite, the curves are called \index{controriented!curves}\emph{counteroriented}.

If we reverse the parametrization of one of the curves, then cooriented curves become counteroriented and vice versa; so we can always assume the curves are cooriented at any given point of tangency.

\section{Supporting curves}

Let $\gamma_1$ and $\gamma_2$ be two smooth regular plane curves that share a point 
\[p=\gamma_1(t_1)=\gamma_2(t_2);\] 
we assume that $p$ is not an endpoint for both.
Suppose that there is $\epsilon>0$ such that the arc $\gamma_2|_{[t_2-\epsilon, t_2+\epsilon]}$ lies in a closed plane region $R$ with the arc $\gamma_1|_{[t_1-\epsilon, t_1+\epsilon]}$ in its boundary,
then we say that $\gamma_1$ \index{support}\emph{locally supports} $\gamma_2$ at the time parameters $t_1$ and $t_2$.

\begin{wrapfigure}[8]{o}{43 mm}
\vskip-3mm
\centering
\includegraphics{mppics/pic-86}
\vskip0mm
\end{wrapfigure}

If the curves on the diagram are oriented according to the arrows, then $\gamma_1$ supports $\gamma_2$ from the right at $p$ (as well as $\gamma_2$ supports $\gamma_1$ from the left at~$p$).

Suppose $\gamma_1$ is a simple curve that cuts the plane into two closed regions, one lies on the left and the other on the right from~$\gamma_1$.
Then we say that $\gamma_1$ \index{support}\emph{globally supports} $\gamma_2$ at point $p=\gamma_2(t_2)$ 
if $\gamma_2$ runs in one of these closed regions, and 
$p$ lies on~$\gamma_1$.

Further, suppose $\gamma_2$ is closed simple plane curve.
By Jordan's theorem (\ref{thm:jordan}), $\gamma_2$ cuts from the plane two regions, one is bounded and the other is unbounded.
We say that a point lies {}\emph{inside} (respectively, {}\emph{outside}) $\gamma_2$ if it lies in the bounded region (respectively, unbounded) region.
In this case, we say that $\gamma_1$ supports $\gamma_2$ \index{support}\emph{from inside} (\index{support}\emph{from outside}) if $\gamma_1$ supports $\gamma_2$ and lies in the region inside $\gamma_2$ (respectively outside $\gamma_2$). 

Note that if $\gamma_1$ and $\gamma_2$ share a point $p=\gamma_1(t_1)=\gamma_2(t_2)$ and are not tangent at $t_1$ and $t_2$, then at time $t_2$ the curve $\gamma_2$ crosses $\gamma_1$  moving from one of its sides to the other.
It follows that $\gamma_1$ cannot locally support $\gamma_2$ at the time parameters $t_1$ and $t_2$.
Whence we get the following:

\begin{thm}{Definition-Observation}
Let $\gamma_1$ and $\gamma_2$ be two smooth regular plane curves.
Suppose $\gamma_1$ locally supports $\gamma_2$ at time parameters $t_1$ and $t_2$.
Then $\gamma_1$ is tangent to $\gamma_2$ at $t_1$ and $t_2$.

In such a case, if the curves are cooriented, and the region $R$ in the definition of supporting curves lies on the right (left) from the arc of $\gamma_1$, then we say that 
$\gamma_1$ supports $\gamma_2$ from the left (respectively right).
\end{thm}

We say that a smooth regular plane curve $\gamma$ has a \index{vertex of  curve}\emph{vertex} at $s$
if the signed curvature function is critical at $s$;
that is, if $\skur'(s)_\gamma=0$.
If in addition to that, $\gamma$ is simple, we could say that the point $p=\gamma(s)$ is a vertex of~$\gamma$.

\begin{thm}{Exercise}\label{ex:vertex-support}
Assume the osculating circle $\sigma_s$ of a smooth regular simple plane curve $\gamma$ locally supports $\gamma$ at $p=\gamma(s)$.
Show that $p$ is a vertex of~$\gamma$.
\end{thm}

\section{Supporting test}

The following proposition resembles the second derivative test. 

\begin{thm}{Proposition}\label{prop:supporting-circline}
Let $\gamma_1$ and $\gamma_2$ be two smooth regular plane curves.

Suppose $\gamma_1$ locally supports $\gamma_2$ from the left (right) at the time parameters $t_1$ and $t_2$.
Then 
\[\skur_1(t_1)\ge \skur_2(t_2)\quad(\text{respectively}\quad \skur_1(t_1)\le \skur_2(t_2)),\]
where $\skur_1$ and $\skur_2$ denote the signed curvature of $\gamma_1$ and $\gamma_2$, respectively.

A partial converse also holds.
Namely, if $\gamma_1$ and $\gamma_2$ tangent and cooriented at the time parameters $t_1$ and $t_2$,
then $\gamma_1$ locally supports $\gamma_2$ from the left (right) at the time parameters $t_1$ and $t_2$
if 
\[\skur_1(t_1)> \skur_2(t_2)\quad(\text{respectively}\quad \skur_1(t_1)< \skur_2(t_2)).\]

\end{thm}

\parit{Proof.} Without loss of generality, we can assume that $t_1=t_2=0$, the shared point $\gamma_1(0)=\gamma_2(0)$ is the origin, and the velocity vectors $\gamma'_1(0)$, $\gamma'_2(0)$ point in the direction of $x$-axis.

Note that small arcs of $\gamma_1|_{[-\epsilon,+\epsilon]}$ and  $\gamma_2|_{[-\epsilon,+\epsilon]}$ can be described as a graph 
$y=f_1(x)$ and $y=f_2(x)$ for smooth functions $f_1$ and $f_2$ such that $f_i(0)=0$ and $f_i'(0)=0$.
Note that $f_1''(0)=\skur_1(0)$, and $f_2''(0)=\skur_2(0)$ (see \ref{ex:curvature-graph})

Clearly, $\gamma_1$ supports $\gamma_2$ from the left (right) if 
\[f_1(x)\ge f_2(x)\quad(\text{respectively}\quad f_1(x)\le f_2(x))\]
for all sufficiently small values of~$x$.
Applying the second derivative test, we get the result.
\qeds


\begin{thm}{Advanced exercise}\label{ex:support}
Suppose that two smooth unit-speed simple plane curves $\gamma_0$ and $\gamma_1$ are tangent and cooriented at the point $p\z=\gamma_0(0)\z=\gamma_1(0)$.
Assume $\skur_0(s)\le\skur_1(s)$ for any~$s$.
Show that $\gamma_0$ locally supports $\gamma_1$ from the right at~$p$.

Give an example of two simple curves $\gamma_0$ and $\gamma_1$ satisfying the above condition such that $\gamma_0$ is closed, but does not support $\gamma_1$ at $p$ globally.
\end{thm}

According to \ref{thm:DNA}, for any closed smooth regular curve that runs in a unit disc, the average of its absolute curvature is at least~1; in particular, there is a point where the absolute curvature is at least~1.
The following exercise says that this statement also holds for loops.

\begin{thm}{Exercise}\label{ex:in-circle}
Assume a smooth regular plane loop $\gamma$ runs in a unit disc.
Show that there is a point on $\gamma$ with absolute curvature at least~1.
\end{thm}


\begin{thm}{Exercise}\label{ex:between-parallels-1}
Assume a closed smooth regular plane curve $\gamma$ runs between two parallel lines at distance 2 from each other.
Show that there is a point on $\gamma$ with absolute curvature at least~1.

Try to prove the same for a smooth regular plane loop.
\end{thm}

\begin{thm}{Exercise}\label{ex:in-triangle}
Assume a closed smooth regular plane curve $\gamma$ runs inside a triangle $\triangle$ with inradius~1; that is, the circle tangent to all three sides of $\triangle$ has radius~1. 
Show that there is a point on $\gamma$ with absolute curvature at least~$1$.
\end{thm}

The three exercises above are baby cases of \ref{ex:moon-rad}; try to find a direct solution, without using \ref{thm:moon}.

{

\begin{wrapfigure}{o}{32 mm}
\vskip-4mm
\centering
\includegraphics{mppics/pic-70}
\vskip0mm
\end{wrapfigure}

\begin{thm}{Exercise}\label{ex:lens}
Let $F$ be a plane figure bounded by two circle arcs $\sigma_1$ and $\sigma_2$ of signed curvature 1 that run from $x$ to~$y$.
Suppose $\sigma_1$ is shorter than~$\sigma_2$.
Assume a smooth regular arc $\gamma$ runs in $F$ and has both endpoints on $\sigma_1$.
Show that the absolute curvature of $\gamma$ is at least 1 at some parameter value.

\end{thm}

}

\section{Convex curves}

Recall that a plane curve is convex if it bounds a convex region.

\begin{thm}{Proposition}\label{prop:convex}
Suppose a closed simple plane curve $\gamma$ bounds a figure~$F$.
Then $F$ is convex if and only if the signed curvature of $\gamma$ does not change the sign.
\end{thm}


\begin{thm}{Lens lemma}\label{lem:lens}
Let $\gamma$ be a smooth regular simple plane curve that runs from $x$ to~$y$.
Assume $\gamma$ runs on the right side (left side) of the oriented line $xy$, and only its endpoints $x$ and $y$ lie on the line.
Then $\gamma$ has a point with positive (respectively negative) signed curvature.
\end{thm}

{

\begin{wrapfigure}{r}{35 mm}
\vskip-4mm
\centering
\includegraphics{mppics/pic-22}
\vskip0mm
\end{wrapfigure}

Note that the lemma fails for curves with self-intersections.
For example, the curve $\gamma$ on the diagram always turns right, 
so it has negative curvature everywhere, but it lies on the right side of the line $xy$.

}

\parit{Proof.}
Choose points $p$ and $q$ on the line $xy$
so that the points $p, x, y, q$ appear in that order.
We can assume that $p$ and $q$ lie sufficiently far from $x$ and $y$, so the half-disc with diameter $pq$ contains~$\gamma$.

\begin{wrapfigure}[6]{o}{50 mm}
\vskip-3mm
\centering
\includegraphics{mppics/pic-24}
\end{wrapfigure}

Consider the smallest disc segment with chord $[p,q]$ that contains~$\gamma$.
Note that its arc $\sigma$ supports $\gamma$ at some point $w=\gamma(t_0)$.

Let us parametrize $\sigma$ from $p$ to~$q$.
Note that $\gamma$ and $\sigma$ are tangent and cooriented at~$w$.
If not, then the arc of $\gamma$ from $w$ to $y$ would be trapped in the curvilinear triangle $xwp$ bounded by the line segment $[p,x]$ and the arcs of $\sigma$,~$\gamma$.
But this is impossible since $y$ does not belong to this triangle.

{

\begin{wrapfigure}{o}{50 mm}
\vskip-4mm
\centering
\includegraphics{mppics/pic-23}
\bigskip
\includegraphics{mppics/pic-230}
\end{wrapfigure}

It follows that $\sigma$ supports $\gamma$ at $t_0$ from the right.
By \ref{prop:supporting-circline}, 
\[\skur(w)_\gamma\ge \skur_\sigma >0.\]
\qedsf

\parit{Remark.}
Instead of taking the minimal disc segment, one can take a point $w$ on $\gamma$ that maximizes the distance to the line $xy$.
The same argument shows that the curvature at $w$ is nonnegative, which is slightly weaker than the required positive curvature.

}

\parit{Proof of \ref{prop:convex}; only-if part.}
If $F$ is convex, then every tangent line of $\gamma$ supports~$\gamma$.
If a point moves along $\gamma$, the figure $F$ has to stay on one side from its tangent line;
that is, we can assume that each tangent line supports $\gamma$ on one side, say on the right.
Since a line has vanishing curvature, the supporting test (\ref{prop:supporting-circline}) implies that $\skur\ge 0$ at each point.

\begin{wrapfigure}{r}{35 mm}
\vskip-3mm
\centering
\includegraphics{mppics/pic-68}
\vskip0mm
\end{wrapfigure}

\parit{If part.}
Denote by $K$ the convex hull of~$F$.
If $F$ is not convex, then $F$ is a proper subset of~$K$.
Therefore, the boundary $\partial K$ contains a line segment that is not a part of $\partial F$.
In other words, there is a line that supports $\gamma$ at two points, say $x$ and $y$, that divide $\gamma$ in two arcs $\gamma_1$ and $\gamma_2$, both distinct from the line segment $[x,y]$.

Note that one of the arcs $\gamma_1$ or $\gamma_2$ is parametrized from $x$ to $y$ and the other from $y$ to~$x$.
Passing to a smaller arc if necessary we can ensure that only its endpoints lie on the line. 
Applying the lens lemma, we get that the arcs $\gamma_1$ and $\gamma_2$ contain points with signed curvatures of opposite signs.
\qeds

\begin{thm}{Exercise}\label{ex:convex small}
Suppose $\gamma$ is a smooth regular simple closed plane curve of diameter larger than~2.
Show that $\gamma$ has a point with absolute curvature less than~1.
\end{thm}

\begin{wrapfigure}{r}{45 mm}
\vskip-0mm
\centering
\includegraphics{mppics/pic-713}
\vskip0mm
\end{wrapfigure}

\begin{thm}{Exercise}\label{ex:convex-lens}
Let $\gamma$ be a simple smooth regular plane arc with endpoints $p$ and~$q$.
Suppose that $\gamma$ has nonnegative signed curvature and $|p-q|$ is the maximal distance between points on~$\gamma$.
Show that the arc $\gamma$ and its chord $[p,q]$ bound a convex figure of the plane.
\end{thm}

\begin{thm}{Exercise}\label{ex:diameter-of-simple-curve}
Show that any simple smooth regular plane curve $\gamma$ with curvature at least 1 has diameter at most 2.

Try to prove that $\gamma$ lies in a unit disc.
\end{thm}


\section{Moon in a puddle}

The following theorem is a slight generalization of the theorem proved by Vladimir Ionin and German Pestov \cite{ionin-pestov}.\index{Ionin--Pestov theorem}
For convex curves, this result was known earlier \cite[\S 24]{blaschke}.


\begin{wrapfigure}{r}{18 mm}
\vskip-6mm
\centering
\includegraphics{mppics/pic-67}
\vskip0mm
\end{wrapfigure}

\begin{thm}{Theorem}\label{thm:moon-orginal}
Assume $\gamma$ is a simple smooth regular plane loop with absolute curvature bounded by~1.
Then it surrounds a unit disc.
\end{thm}

This theorem gives a simple but nontrivial example of the so-called \index{local!to global theorem}\emph{local to global theorems} --- based on some local data (in this case, the curvature of a curve) we conclude a global property (in this case, the existence of a large disc surrounded by the curve).

{

\begin{wrapfigure}{r}{33 mm}
\vskip-4mm
\centering
\includegraphics{mppics/pic-62}
\vskip0mm
\end{wrapfigure}

A straightforward approach would be to start with some disc in the region bounded by the curve and blow it up to maximize its radius.
However, as one may see from the spinner-like example on the diagram it does not always lead to a solution --- a closed plane curve of curvature at most 1 may surround a disc of radius smaller than 1 that cannot be enlarged continuously.

}

\begin{thm}{Key lemma}\label{thm:moon}
Assume $\gamma$ is a simple smooth regular plane loop.
Then at one point of $\gamma$ (distinct from its base), its osculating circle $\sigma$ globally supports $\gamma$ from the inside.
\end{thm}

First, let us show that the theorem follows from the lemma.

\parit{Proof of \ref{thm:moon-orginal} modulo \ref{thm:moon}.}
Since $\gamma$ has absolute curvature at most 1, each osculating circle has radius at least~1.
According to the key lemma, one of the osculating circles $\sigma$ globally supports $\gamma$ from the inside.
In particular, $\sigma$ lies inside $\gamma$, whence the result.
\qeds

\parit{Proof of \ref{thm:moon}.}
Denote by $F$ the closed region surrounded by $\gamma$.
We can assume that $F$ lies on the left from~$\gamma$.
Arguing by contradiction,
assume that the osculating circle at each point $p\in \gamma$ does not lie in~$F$.

\begin{figure}[!ht]
\vskip-0mm
\centering
\includegraphics{mppics/pic-32}
\vskip0mm
\end{figure}

Given a point $p\in\gamma$ let us consider the maximal circle $\sigma$ that lies completely in $F$ and tangent to $\gamma$ at~$p$.
The circle $\sigma$ will be called the {}\emph{incircle} of $F$ at~$p$.

Note that the curvature $\skur_\sigma$ of the incircle $\sigma$ has to be larger than $\skur(p)_\gamma$.
Indeed, since $\sigma$ supports $\gamma$ from the left, by \ref{prop:supporting-circline} we have $\skur_\sigma\ge \skur(p)_\gamma$; in the case of equality, $\sigma$ is the osculating circle at~$p$.
The latter is impossible by our assumption.

It follows that $\sigma$ has to touch $\gamma$ at another point.
Otherwise, we can increase $\sigma$ slightly while keeping it inside~$F$.

Indeed, since $\skur_\sigma> \skur(p)_\gamma$, 
by \ref{prop:supporting-circline} we can choose a neighborhood $U$ of $p$ such that after a slight increase of $\sigma$, the intersection $U\cap \sigma$ is still in~$F$.
On the other hand, if $\sigma$ does not touch $\gamma$ at another point, then after some (maybe smaller) increase of $\sigma$ the complement $\sigma\backslash U$ is still in~$F$.
That is, a slightly increased $\sigma$ is still in $F$ --- a contradiction.



Choose a point $p_1$ on $\gamma$ that is distinct from its base point. 
Let $\sigma_1$ be the incircle at $p_1$.
Denote by $\gamma_1$ an arc of $\gamma$ from $p_1$ to a first point $q_1$ on $\sigma_1$.
Denote by $\hat\sigma_1$ and $\check\sigma_1$ two arcs of $\sigma_1$ from $p_1$ to $q_1$ such that the cyclic concatenation of $\hat\sigma_1$ and $\gamma_1$ surrounds~$\check\sigma_1$. 

\begin{wrapfigure}{r}{32 mm}
\vskip-2mm
\centering
\includegraphics{mppics/pic-64}
\caption*{Two ovals pretend to be circles.}
\vskip0mm
\end{wrapfigure}

Let $p_2$ be the midpoint of $\gamma_1$.
Denote by $\sigma_2$ the incircle at $p_2$.

Note that $\sigma_2$ cannot intersect $\hat\sigma_1$.
Otherwise, if $\sigma_2$ intersects $\hat\sigma_1$ at some point $s$, then $\sigma_2$ has to have two more common points with $\check\sigma_1$, say $x$ and $y$ --- one for each arc of $\sigma_2$ from $p_2$ to~$s$.
Therefore, $\sigma_1\z=\sigma_2$ since these two circles have three common points: $s$, $x$, and~$y$. 
On the other hand, by construction, $p_2\in \sigma_2$ and $p_2\notin \sigma_1$ --- a contradiction.


Recall that $\sigma_2$ has to touch $\gamma$ at another point.
From above, it follows that $\sigma_2$ can only touch $\gamma_1$. 
Therefore we can choose an arc $\gamma_2\subset \gamma_1$ that runs from $p_2$ to a first point $q_2$ on $\sigma_2$.
Since $p_2$ is the midpoint of $\gamma_1$, we have that
\[\length \gamma_2< \tfrac12\cdot\length\gamma_1.\eqlbl{eq:length<length/2}\]

Repeating this construction recursively,
we get an infinite sequence of arcs $\gamma_1\supset \gamma_2\supset\dots$;
by \ref{eq:length<length/2}, we also get that 
\[\length\gamma_n\to0\quad\text{as}\quad n\to\infty.\] 
Therefore, the intersection $\gamma_1\cap\gamma_2\cap\dots$
contains a single point; denote it by $p_\infty$.

Let $\sigma_\infty$ be the incircle at $p_\infty$; it has to touch $\gamma$ at another point, say $q_\infty$.
The same argument as above shows that $q_\infty\in\gamma_n$ for any~$n$.
It follows that $q_\infty =p_\infty$ --- a contradiction.
\qeds

\begin{thm}{Exercise}\label{ex:moon-rad}
Assume a closed smooth regular curve $\gamma$ lies in a figure $F$ bounded by a closed simple plane curve.
Suppose $R$ is the maximal radius of discs that lie in~$F$.
Show that the absolute curvature of $\gamma$ is at least $\tfrac1R$ at some parameter value.
\end{thm}




\section{Four-vertex theorem}
\index{Four-vertex theorem}
{

\begin{wrapfigure}{r}{20 mm}
\vskip-8mm
\centering
\includegraphics{mppics/pic-26}
\vskip0mm
\end{wrapfigure}

Recall that a vertex of a smooth regular curve is defined as a critical point of its signed curvature;
in particular, any local minimum (or maximum) of the signed curvature is a vertex.
For example, every point of a circle is a vertex.

\begin{thm}{Four-vertex theorem}\label{thm:4-vert}
Any smooth regular simple closed plane curve has at least four vertices.
\end{thm}

}

Evidently, any closed smooth regular curve has at least two vertices --- where the minimum and the maximum of the curvature are attained.
On the diagram the vertices are marked;
the first curve has one self-intersection and exactly two vertices;
the second curve has exactly four vertices and no self-intersections.

The four-vertex theorem was first proved by Syamadas Mukhopadhyaya \cite{mukhopadhyaya} for convex curves.
Many proofs and generalizations are known.
One of our favorite proofs was given by Robert Osserman \cite{osserman}.
Our proof of the following stronger statement based on the key lemma in the previous section.


\begin{thm}{Theorem}\label{thm:4-vert-supporting}
Any smooth regular simple closed plane curve is globally supported by its osculating circle at least at 4 distinct points;
two from inside and two from outside.
\end{thm}

{

\begin{wrapfigure}[8]{r}{33 mm}
\vskip-0mm
\centering
\includegraphics{mppics/pic-63}
\vskip0mm
\end{wrapfigure}

\parit{Proof of \ref{thm:4-vert} modulo \ref{thm:4-vert-supporting}.}
First, note that if an osculating circline $\sigma$ at a point $p$ supports $\gamma$ locally, then $p$ is a vertex.
Indeed, if not, then a small arc around $p$ has monotone curvature.


Applying the spiral lemma (\ref{lem:spiral}) we get that the osculating circles at this arc are nested.
In particular, the curve $\gamma$ crosses $\sigma$ at~$p$. 
Therefore $\sigma$ does not support $\gamma$ locally at~$p$.
\qeds


}


\parit{Proof of \ref{thm:4-vert-supporting}.}
According to the key lemma (\ref{thm:moon}), there is a point $p\in\gamma$ such that its osculating circle supports $\gamma$ from inside.
The curve $\gamma$ can be considered as a loop with the base at~$p$.
Therefore, the key lemma implies the existence of another point $q\in\gamma$ with the same property.

It shows the existence of two osculating circles that support $\gamma$ from inside;
it remains to show the existence of two osculating circles that support $\gamma$ from outside.

In order to get the osculating circles supporting $\gamma$ from outside, one can repeat the proof of the key lemma taking instead of incircle the circline of maximal signed curvature that supports the curve from outside, assuming that $\gamma$ is oriented so that the region on the left from it is bounded.

(Alternatively, if one applies to $\gamma$ an inversion with respect to a circle whose center lies inside~$\gamma$, then the obtained curve $\gamma_1$ also has  two osculating circles that support $\gamma_1$ from inside.
According to \ref{ex:inverse}, these osculating circlines are inverses of the osculating circlines of~$\gamma$.
Note that the region lying inside $\gamma$ is mapped to the region outside $\gamma_1$, and the other way around.
Therefore, these two circlines correspond to the osculating circlines supporting $\gamma$ from outside.)
\qeds

{

\begin{wrapfigure}{r}{20 mm}
\vskip-8mm
\centering
\includegraphics{mppics/pic-725}
\vskip0mm
\end{wrapfigure}

\begin{thm}{Exercise}\label{ex:2-squares}
Assume that a smooth regular simple closed plane curve \(\gamma\) is contained in a square with side 2 and surrounds a square with diagonal~2.
Show that \(\gamma\) contains a point with curvature~1.
\end{thm}

}

\begin{thm}{Exercise}\label{ex:moon-area}
Suppose a smooth regular simple plane loop $\gamma$ bounds area $a$.
Show that $\gamma$ has curvature $\sqrt{\pi/a}$ at some point.
\end{thm}

\begin{wrapfigure}[5]{r}{25 mm}
\vskip-7mm
\centering
\includegraphics{mppics/pic-65}
\vskip0mm
\end{wrapfigure}

\begin{thm}{Advanced exercise}\label{ex:curve-crosses-circle}
Suppose $\gamma$ is a closed simple smooth regular plane curve, and $\sigma$ is a circle.
Assume $\gamma$ crosses $\sigma$ at the points $p_1,\dots,p_{2{\cdot} n}$, and these points appear in the same cycle order on $\gamma$ and on $\sigma$.
Show that $\gamma$ has at least $2\cdot n$ vertices.

Construct an example of a closed simple smooth regular plane curve $\gamma$ with only 4 vertices that crosses a given circle at arbitrarily many points. 
\end{thm}

%The following exercise is a version of four-vertex theorem for space curves without parallel tangents;
%they exist by \ref{ex:no-parallel-tangents}.

%\begin{thm}{Advanced exercise}\label{ex:4x0-torsion}
%Let $\gamma$ be a closed smooth regular space curve that has no pair of points with parallel tangent lines.
%Assume the curvature of $\gamma$ does not vanish.
%Show that $\gamma$ has four points with zero torsion.
%\end{thm}

%\begin{thm}{Advanced exercise}\label{ex:order} Show that the points $a,b,c,d$ guaranteed by \ref{thm:4-vert-supporting} can be chosen so that they appear in the same order on the curve and the osculating circles at $a$ and $c$ support the curve from inside and $b$ and $d$ support the curve from outside. \end{thm}
