\chapter{Afterword}

After our book, if you continue with Riemannian geometry,
then half of the material in this subject will look familiar.
But first, you need to read a text on tensor calculus;
the book by Richard Bishop and Samuel Goldberg \cite{bishop-goldberg} is one of our favorites.

Let us list a few introductory texts we know and love that range from very detailed to very condensed and challenging:
\begin{itemize}
\item ``Riemannian manifolds''  \cite{lee2006riemannian} by John Lee.
\item ``Riemannian geometry'' \cite{carmo1992riemannian} by Manfredo do Carmo.
\item ``Riemannsche Geometrie im Großen'' \cite{gromoll-klingenberg-meyer} by 
Detlef Gromoll,
Wilhelm Klingenberg, 
and  Werner Meyer;
it is in German, plus there is a Russian translation.
In Russian, there is a more elementary textbook by Yurii Burago and Viktor Zalgaller \cite{burago-zalgaller}.
\item ``Comparison geometry'' \cite{cheeger-ebin} by Jeff Cheeger and David Ebin. 
\item ``Sign and geometric meaning of curvature'' \cite{gromov-1991} by Mikhael Gromov.
\end{itemize}
Good luck.

\begin{flushright}
Anton Petrunin and Sergio Zamora Barrera,\\
State College, Pennsylvania, May 11, 2021.
\end{flushright}
