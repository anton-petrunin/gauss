\chapter*{Want more?}
\addcontentsline{toc}{chapter}{Want more?}



For further study, you need to read a text on tensor calculus;
the book of Richard Bishop and Samuel Goldberg \cite{bishop-goldberg} is one of our favorites.
Once it is done, you are ready to do Riemannian geometry.
Let us list few books we know and love.

Introductory books are ``Riemannian geometry'' \cite{carmo1992riemannian} by Manfredo DoCarmo and ``Riemannian manifolds: an introduction to curvature''  \cite{lee2006riemannian} by John M. Lee. The former covers the basics in a minimalist way getting into more advanced theory quickly. The latter goes more in depth allowing the reader to move in distinct directions once such theory is learned.



You may go with the slightly more advanced ``Comparison geometry'' \cite{cheeger-ebin} --- the good old classic from Jeff Cheeger and David Ebin. 
In this case, you might be surprised to see that half of the material looks familiar.
For example, proofs of the comparison theorems require only cosmetic modifications.


A safer option would be another classical book ``Riemannsche Geometrie im Großen'' \cite{gromoll-klingenberg-meyer} by 
Detlef Gromoll,
Wilhelm Klingenberg, 
and  Werner Meyer (it is available in German and Russian).

Mikhael Gromov's ``Sign and geometric meaning of curvature'' \cite{gromov-1991} is more challenging but worth trying.

Good luck.

\begin{flushright}
Anton Petrunin and Sergio Zamora Barrera,\\
State College, Pennsylvania, April 19, 2021.
\end{flushright}
