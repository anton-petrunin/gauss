\chapter{Geodesics}
\label{chap:geodesics}


\section{Definition}

A smooth curve $\gamma$ in a smooth surface is called a \index{geodesic}\emph{geodesic} if, for any~$t$, the acceleration $\gamma''(t)$ is perpendicular to the tangent plane $\T_{\gamma(t)}$.

\begin{thm}{Exercise}\label{ex:helix-geodesic}
Show that the helix $\gamma(t)\z\df(\cos t,\sin t, t)$ is a geodesic on the cylindrical surface $x^2+y^2=1$.
\end{thm}


Geodesics can be understood as the trajectories of particles that slide on $\Sigma$ without friction and external forces.
Indeed, since there is no friction, the force that keeps the particle on $\Sigma$ must be perpendicular to~$\Sigma$.
Therefore, by Newton's second law of motion,
we get that the acceleration $\gamma''$ is perpendicular to $\T_{\gamma(t)}$.

From a physics point of view, the following lemma is a corollary of the law of conservation of energy.


\begin{thm}{Lemma}\label{lem:constant-speed}
Let $\gamma$ be a geodesic in a smooth surface~$\Sigma$. 
Then $|\gamma'|$ is constant.
Moreover, for any $\lambda\in\mathbb{R}$, the curve 
$\gamma_{\lambda}(t)\df \gamma (\lambda\cdot t)$ is a geodesic as well. 
\end{thm}

In other words, any geodesic has a constant speed, and multiplying its parameter by a constant yields another geodesic.

\parbf{Proof.} 
Since $\gamma'(t)$ is a tangent vector at $\gamma(t)$,
we have that $\gamma''(t)\z\perp\gamma'(t)$, or equivalently $\langle\gamma'',\gamma'\rangle=0$ for any~$t$.
Hence 
$\langle\gamma',\gamma'\rangle'=2\cdot \langle\gamma'',\gamma'\rangle=0$.
That is, $|\gamma'|^2=\langle\gamma',\gamma'\rangle$ is constant.

The second part follows since 
$\gamma_{\lambda}''(t) =\lambda^2\cdot \gamma''(\lambda t)$.
\qeds


The statement in the following exercise is called \index{Clairaut's relation}\emph{Clairaut's relation};
it can be obtained from the conservation of angular momentum.

\begin{thm}{Exercise}\label{ex:clairaut}
Let $\gamma$ be a geodesic on a smooth surface of revolution.
Suppose $r(t)$ denotes the distance from $\gamma(t)$ to the axis of rotation
and $\theta(t)$ --- the angle between $\gamma'(t)$ and the latitudinal circle thru $\gamma(t)$. 

Show that the value $r(t)\cdot \cos\theta(t)$ is constant. 
\end{thm}


Recall that an {}\emph{asymptotic line} is a curve in the surface with vanishing normal curvature.

\begin{thm}{Exercise}\label{ex:asymptotic-geodesic}
Assume a curve $\gamma$ is a geodesic and, at the same time, an asymptotic line of a smooth surface.
Show that $\gamma$ is a straight-line segment.
\end{thm}


\begin{thm}{Exercise}\label{ex:reflection-geodesic}
Assume a smooth surface $\Sigma$ is mirror-symmetric with respect to a plane~$\Pi$.
Suppose $\Sigma$ and $\Pi$ intersect along a smooth curve~$\gamma$.
Show that $\gamma$, parametrized by arc-length, is a geodesic on~$\Sigma$.
\end{thm}



\section{Existence and uniqueness}

The following proposition says that the motion without friction and external forces depends smoothly on its initial data. 
The formulation uses the smoothness of the map $w$, which spits a point in $\mathbb{R}^3$ for a point $p$ on the surface $\Sigma$, a tangent vector $\vec{v}$ at $p$, and a real parameter $t$.
In a chart $s$ on $\Sigma$, the point $p$ is described by a pair of coordinates $(u,v)$, and the vector $\vec{v}$ can be expressed as the sum $a\cdot s_u+b\cdot s_v$.
Therefore, in local coordinates, $w$ is defined by the mapping $(u,v,a,b,t)\mapsto w(p,\vec v,t)$ from a subset of $\mathbb{R}^5$ to $\mathbb{R}^3$, to which the usual definition of smoothness applies.
Now, the map $(p,\vec v, t)\z\mapsto w(p,\vec v, t)$ is considered to be \index{smooth map}\emph{smooth} if it is described by a smooth mapping $(u,v,a,b,t)\mapsto w(p,\vec v,t)$ in any local coordinates on $\Sigma$.


\begin{thm}{Proposition}\label{prop:geod-existence} 
Let $\Sigma$ be a smooth surface without boundary.
Given a tangent vector ${\vec v}$ to $\Sigma$ at a point $p$,
there is a unique geodesic $\gamma\:\mathbb{I}\to \Sigma$ defined on a maximal open interval $\mathbb{I}\ni 0$ that starts at $p$ with velocity vector ${\vec v}$;
that is, $\gamma(0)=p$ and $\gamma'(0)={\vec v}$.
Moreover, we have the following.
\begin{subthm}{prop:geod-existence:smooth}
The map $w\:(p,{\vec v},t)\mapsto \gamma(t)$ is smooth, and it has an open domain of definition%
\footnote{That is, if $w$ is defined for a triple $p\in \Sigma$, ${\vec v}\in \T_p$, and $t\in \mathbb{R}$,
then it is also defined for any triple $q\in \Sigma$, $\vec u\in \T_p$, and $s\in \mathbb{R}$, where $q$, $\vec u$, and $s$ are sufficiently close to $p$, ${\vec v}$, and $t$, respectively.}%
.
\end{subthm}

\begin{subthm}{prop:geod-existence:whole}
If $\Sigma$ is proper, then $\mathbb{I}=\mathbb{R}$; that is, $\gamma$ is defined on the entire real line.
\end{subthm}

\end{thm}

A surface that satisfies the conclusion of \ref{SHORT.prop:geod-existence:whole} for any tangent vector ${\vec v}$ is said to be \index{geodesically complete}\emph{geodesically complete}.
So part \ref{SHORT.prop:geod-existence:whole} says that \textit{any proper surface without boundary is geodesically complete}.
This is a part of the \index{Hopf--Rinow theorem}\emph{Hopf--Rinow theorem} \cite{hopf-rinow}.

In the proof, we will rewrite the definition of a geodesic using a differential equation and then apply Theorems \ref{thm:ODE-nth-order} and \ref{thm:ODE}.

\begin{thm}{Lemma}\label{lem:geodesic=2nd-order}
Let $f$ be a smooth function defined on an open domain in $\mathbb{R}^2$.
A smooth curve $t\mapsto \gamma(t)=(x(t),y(t),z(t))$ is a geodesic in the graph $z=f(x,y)$ if and only if $z(t)=f(x(t),y(t))$ for any $t$ and the functions $t\mapsto x(t)$ and $t\mapsto y(t)$
satisfy a system of differential equations
\[
\begin{cases}
x''=g(x,y,x',y'),
\\
y''=h(x,y,x',y'),
\end{cases}
\]
where $g$ and $h$ are smooth functions of four variables that uniquely determined by~$f$.
\end{thm}

\parbf{Proof.} The first condition $z(t)=f(x(t),y(t))$ simply means that $\gamma(t)$ lies on the graph $z=f(x,y)$.

In the following calculations, we often omit the arguments;
in other words, we may use shortcuts $x=x(t)$, $f=f(x,y)=f(x(t),y(t))$, and so on.

First, let us express $z''$ in terms of $f$, $x$, and $y$.
\[
\begin{aligned}
z''&=f(x,y)''=
\\
&=\left(f_x\cdot x'+ f_y\cdot y'\right)'=
\\
&=
f_{xx}\cdot (x')^2
+
f_x\cdot x''
+ 2\cdot f_{xy}\cdot x'\cdot y'
+
f_{yy}\cdot (y')^2
+
f_y\cdot y''.
\end{aligned}
\eqlbl{eq:def-geod}
\]

The condition
\[\gamma''(t)\perp\T_{\gamma(t)}\] 
means that 
$\gamma''$ is perpendicular to the two basis vectors in $\T_{\gamma(t)}$; that is,
\[
\begin{cases}
\langle \gamma'',s_x\rangle=0,
\\
\langle\gamma'',s_y\rangle=0,
\end{cases}
\]
where $s(x,y)\df (x,y,f(x,y))$, $x=x(t)$, and $y=y(t)$.
Observe that 
$s_x=(1,0, f_x)$ 
and 
$s_y=(0,1, f_y)$.
Since $\gamma''\z=(x'',y'',z'')$, we can rewrite our system as
\[
\begin{cases}
x''+ f_x\cdot z''=0,
\\
y''+ f_y\cdot z''=0.
\end{cases}
\]
It remains to plug in \ref{eq:def-geod} for $z''$, combine the similar terms, and simplify.
\qeds


\parbf{Proof of \ref{prop:geod-existence}.}
Let $z=f(x,y)$ be a description of $\Sigma$ in tangent-normal coordinates at~$p$.
By Lemma \ref{lem:geodesic=2nd-order}, the condition $\gamma''(t)\perp\T_{\gamma(t)}$ can be written as a system of second-order differential equations.
From \ref{thm:ODE-nth-order} and \ref{thm:ODE} we get the existence and uniqueness of a geodesic $\gamma$ in an interval $(-\epsilon,\epsilon)$ for some $\epsilon>0$.

Let us extend the geodesic $\gamma$ to a maximal open interval $\mathbb{I}$.
Suppose there is another geodesic $\gamma_1$ with the same initial data that is defined on a maximal open interval $\mathbb{I}_1$.
Suppose $\gamma_1$ splits from $\gamma$ at some $t_0>0$;
that is, $\gamma_1$ and $\gamma$ coincides on the interval $[0,t_0)$, but they are different on any interval $[0,t_0+\delta)$ with $\delta>0$.
By continuity, $\gamma_1(t_0)=\gamma(t_0)$, and $\gamma_1'(t_0)=\gamma'(t_0)$.
Applying \ref{thm:ODE-nth-order} and \ref{thm:ODE} again, we get that $\gamma_1$ coincides with $\gamma$ in a small neighborhood of $t_0$ --- a contradiction.

The same argument shows that $\gamma_1$ cannot split from $\gamma$ at $t_0<0$.
It follows that $\gamma_1=\gamma$;
in particular, $\mathbb{I}_1=\mathbb{I}$.

If $\Sigma$ is the graph of a smooth function, then part \ref{SHORT.prop:geod-existence:smooth} follows from \ref{thm:ODE-nth-order}, \ref{thm:ODE}, and the lemma.
In this case, the mapping
\[\vec{w}(p,\vec v,t)\df\tfrac{\partial}{\partial t}w(p,\vec v,t)\]
is also smooth.
Note that $\vec{w}(p,\vec v,t_0) \in \T_{\gamma(t_0)}$ is the velocity vector of the geodesic $\gamma\:t\mapsto w(p,\vec v,t)$ at time $t_0$.

In the general case, suppose $w(p,{\vec v},b)$ is defined for $b\ge0$; that is, the geodesic $\gamma\:t\mapsto w(p,{\vec v},t)$ is defined on the interval $[0,b]$.
Then there exists a partition $0=t_0<t_1<\dots<t_n=b$ of the interval $[0,b]$ such that each geodesic segment $\gamma|_{[t_{i-1},t_i]}$ is covered by a chart defined by some tangent-normal coordinates.
Set $p_i=\gamma(t_i)$ and $\vec v_i=\gamma'(t_i)$, so $p_0=p$ and $\vec v_0=\vec v$.
Since $\gamma|_{[t_{i-1},t_i]}$ lies in a graph, by the previous reasoning, we conclude that for each $i$, the mappings
$w$ and $\vec w$ are defined and smooth in a neighborhood of the triples $(p_{i-1},\vec v_{i-1}, t_i-t_{i-1})$.
Note that $p_i=w(p_{i-1},\vec v_{i-1},t_i-t_{i-1})$ and $\vec v_i=\vec w(p_{i-1},\vec v_{i-1},t_i-t_{i-1})$ for each $i$.
Since the composition of smooth mappings is smooth, the mapping $w$ is smoothly defined in a neighborhood of the triple $(p,\vec v, b)$.

The case $b\le 0$ is similar; so, we have obtained the general case in \ref{SHORT.prop:geod-existence:smooth}.


Assume \ref{SHORT.prop:geod-existence:whole} does not hold;
that is, the maximal interval $\mathbb{I}$ is a proper subset of $\mathbb{R}$.
Without loss of generality, we may assume that $b=\sup\mathbb{I}<\infty$.
(If not, switch the direction of~$\gamma$.)

By \ref{lem:constant-speed}, $|\gamma'|$ is constant; in particular, $t\mapsto \gamma(t)$ is a uniformly continuous function.
Therefore, the limit point
$q=\lim_{t\to b}\gamma(t)$
is defined.
Since $\Sigma$ is a proper surface, $q\in \Sigma$. 

Applying the argument above in a tangent-normal coordinate chart at $q$, we conclude that $\gamma$ can be extended as a geodesic beyond~$q$.
Therefore, $\mathbb{I}$ is not maximal --- a contradiction.
\qeds

\begin{thm}{Exercise}\label{ex:round-torus}
Let $\Sigma$ be a smooth torus of revolution; that is,
a surface of revolution whose generatrix is a smooth closed curve.
Show that any closed geodesic on $\Sigma$ is noncontractible.

(In other words, if $s\:\mathbb{R}^2\to \Sigma$ is the natural bi-periodic parametrization of $\Sigma$, then
there is no closed curve $\gamma$ in $\mathbb{R}^2$ such that $s\circ\gamma$ is a geodesic.)
\end{thm}


\section{Exponential map}\label{sec:exp}

Let $\Sigma$ be a smooth surface and $p\in \Sigma$.
Given a tangent vector ${\vec v}\in \T_p$, consider a geodesic $\gamma_{\vec v}$ in $\Sigma$ that starts at $p$ with initial velocity~$\vec v$; 
that is, $\gamma(0)=p$ and $\gamma'(0)={\vec v}$.

Let us define the \index{exponential map}\emph{exponential map}%
\footnote{Explaining the reason for this term, would take us far away from the subject.}
at $p$ by
\[\exp_p\:\vec v\mapsto \gamma_{\vec v}(1).\]
By \ref{prop:geod-existence}, this map is smooth, and it is defined in a neighborhood of zero in the tangent plane $\T_p$;
moreover, if $\Sigma$ is proper, 
then $\exp_p$ is defined on the entire plane $\T_p$.

The exponential map
is defined on the tangent plane (or an open subset of it), which is a smooth surface,
and its target is another smooth surface.
We can identify the plane $\T_p$
with its tangent plane $\T_0\T_p$, so the differential $d_0(\exp_p)\:\vec v\mapsto D_{\vec v}\exp_p$ maps $\T_p$ to itself.
Furthermore, by Lemma \ref{lem:constant-speed}, this differential is the identity map; that is, $d_0\exp_p(\vec v)=
\vec v$ for any $\vec v\in \T_p$.

Summarizing, we get the following statement:

\begin{thm}{Observation}\label{obs:d(exp)=1}
Let $\Sigma$ be a smooth surface and $p\in \Sigma$.
Then:

\begin{subthm}{}
The exponential map $\exp_p$ is smooth, and its domain $\Dom(\exp_p)$ contains a neighborhood of the origin in $\T_p$.
Moreover, if $\Sigma$ is proper, then $\Dom(\exp_p)=\T_p$
\end{subthm}

\begin{subthm}{}
The differential $d_0(\exp_p)\:\T_p\to \T_p$ is the identity map.
\end{subthm}

\end{thm}

It is easy to check that $\Dom(\exp_p)$ is \index{star-shaped}\emph{star-shaped} in $\T_p$;
the latter means that if $\vec v\in \Dom(\exp_p)$, then $\lambda\cdot\vec v\in \Dom(\exp_p)$ for any $0\le \lambda\le 1$.

\section{Injectivity radius}

The \index{injectivity radius}\emph{injectivity radius} of $\Sigma$ at $p$ (briefly $\inj(p)$) is defined as least upper bound on radii $r_p\ge0$ such that the exponential map $\exp_p$ is defined on the open ball $B_p\df B(0,r_p)_{\T_p}$,
and the restriction $\exp_p|_{B_p}$ is a smooth regular parametrization of a neighborhood of $p$ in~$\Sigma$.


\begin{thm}{Proposition}\label{prop:exp}
Injectivity radius is positive at any point on a smooth surface $\Sigma$ (without boundary).
Moreover, it is {}\emph{locally bounded away from zero};
that is, for any $p\in\Sigma$ there is $\epsilon>0$ such that if $\dist{p}{q}\Sigma<\epsilon$ for some $q\in \Sigma$, then $\inj(q)\ge\epsilon$.
\end{thm}

In fact, it is true that \textit{the function $\inj\:\Sigma\to (0,\infty]$ is continuous} \cite[5.4]{gromoll-klingenberg-meyer}.
The proof of the proposition uses \ref{obs:d(exp)=1} and the inverse function theorem (\ref{thm:inverse}).

\parbf{Proof.}
Let $z=f(x,y)$ be a local graph representation of $\Sigma$ in tangent-normal coordinates at~$p$.
In this case, the $(x,y)$-plane coincides with the tangent plane $\T_p$.

Denote by $h$ the composition of $\exp_p$ with the projection $(x,y,z)\z\mapsto (x,y)$.
By \ref{obs:d(exp)=1}, the differential $d_0h$ is the identity;
in other words, the Jacobian matrix of $h$ at $0$ is the identity.
Applying the inverse function theorem (\ref{thm:inverse}) we get the first part of the proposition.

The proof of second part is similar, but more technical.

Denote by $h_q$ the composition of $\exp_q$ with the orthogonal projection $(x,y,z)\mapsto (x,y)$.
Consider the chart $s\:(u,v)\z\mapsto (u,v,f(u,v))$.
Let
\[m\:(u,v,a,b)\mapsto h_q(\vec v),\]
where $q=s(u,v)$ and $\vec v=a\cdot s_u+b\cdot s_v$.
By \ref{prop:geod-existence}, $m$ is a smooth map defined in a neighborhood of $0$.
Passing to a smaller neighborhood of $0$, we can assume that first and second partial derivatives of $m$ are bounded.
Form above, the Jacobian matrix of $(a,b)\z\mapsto m(0,0,a,b)$ at $0$ is the identity.
It follows that for small fixed $u$ and $v$
the Jacobian matrix of $(a,b)\z\mapsto m(u,v,a,b)$ at $0$ is close to identity.
In particular, we can apply the second part of the inverse function theorem (\ref{thm:inverse}) to
the map $(a,b)\z\mapsto m(u,v,a,b)$  for small fixed $u$ and $v$, hence the result.
\qeds

The proof of the following statement will be indicated in \ref{ex:inj-rad}.

\begin{thm}{Proposition}\label{prop:inj-rad}
Let $p$ be a point on a smooth surface $\Sigma$ (without boundary).
If $\exp_p$ is injective in $B_p=B(0,r)_{\T_p}$, then the restriction $\exp_p|_{B_p}$ is a diffeomorphism between $B_p$ and its image in~$\Sigma$.

\end{thm}

In other words, $\inj(p)$ can be defined as the least upper bound on $r$ such that $\exp_p$ is injective in the ball $B(0,r)_{\T_p}$.

\section{Shortest paths are geodesics}

\begin{thm}{Proposition}\label{prop:gamma''}
Let $\Sigma$ be a smooth surface.
Then any shortest path $\gamma$ in $\Sigma$ parametrized proportional to its arc-length is a geodesic in~$\Sigma$.
In particular, $\gamma$ is a smooth curve.

A local converse also holds: any point $p$ in $\Sigma$ has a neighborhood $U$ such that any geodesic that lies entirely in $U$ is a shortest path.
\end{thm}

In particular, a sufficiently short arc of any geodesic is a shortest path.
A geodesic that is also a shortest path is called \index{minimizing geodesic}\emph{minimizing}.
As one can see from the following exercise, geodesics might fail to be minimizing.

\begin{thm}{Exercise}\label{ex:helix=geodesic}
Let $\Sigma$ be the cylindrical surface described by the equation $x^2\z+y^2=1$.
Show that the helix $\gamma\:[0,2\cdot\pi]\to \Sigma$ defined by $\gamma(t)\z\df(\cos t,\sin t, t)$
is a geodesic, but not a shortest path on~$\Sigma$.
\end{thm}

A formal proof of the proposition will be given in Section~\ref{sec:proof-of-gamma''}.
The following informal physical explanation might be sufficiently convincing.
In fact, if one assumes that $\gamma$ is smooth, then it is easy to convert this explanation into a rigorous proof.

\parbf{Informal explanation.}
Let us think about a shortest path $\gamma$ as a stable position of a stretched elastic thread that is forced to lie on a frictionless surface.
Since it is frictionless, the force density $\vec n=\vec n(t)$ that keeps $\gamma$ in the surface must be proportional to the normal vector to the surface at~$\gamma(t)$.

Denote the tension by $\tau$ the tension in the thread.
It has to be the same at all points;
otherwise, the thread would move back or forth, and it would not be stable.


We can assume that $\gamma$ has unit speed;
in this case, the net force from tension along the arc $\gamma_{[t_0,t_1]}$ is $\tau\cdot(\gamma'(t_1)-\gamma'(t_0))$.
Hence, the density of the net force from tension at $t_0$ is 
\begin{align*}
\vec f(t_0)&=\lim_{t_1\to t_0}\tau\cdot\frac{\gamma'(t_1)-\gamma'(t_0)}{t_1-t_0}=
\\
&=\tau\cdot\gamma''(t_0).
\end{align*}
By Newton's second law of motion,  
$\vec f+\vec n=0$; hence $\gamma''(t)\perp\T_{\gamma(t)}\Sigma$.
\qeds

\begin{thm}{Corollary}
Let $p$ be a point on a smooth surface $\Sigma$, and $r\z\le \inj(p)$.
Then the exponential map $\exp_p$ defines a diffeomorphism from $B(0,r)_{\T_p}$ to $B(p,r)_\Sigma$.
\end{thm}

\parbf{Proof.}
By \ref{prop:inj-rad}, the restriction of $\exp_p$ to $B_p={B(0,r)_{\T_p}}$ is a diffeomorphism to its image $\exp_p(B_p)\subset \Sigma$.

Evidently, $B(p,r)_\Sigma\supset\exp_p(B_p)$.
By \ref{prop:gamma''}, $B(p,r)_\Sigma\subset\exp_p(B_p)$, hence the result.
\qeds

By the corollary, the restriction $\exp_p|_{B(0,r)_{\T_p}}$ admits an inverse map called \index{logarithmic map}\emph{logarithmic};
it is denoted by \[\log_p\:B(p,r)_\Sigma\to B(0,r)_{\T_p}.\]

According to the proposition above, any shortest path parametrized by its arc-length is a smooth curve.
This observation helps to solve the following two exercises:

\begin{thm}{Exercise}\label{ex:two-min-geod}
Show that if two shortest paths have two distinct common points $p$ and $q$, then either $p$ and $q$ are the endpoints of both shortest paths, or both shortest paths share a common arc from $p$ to~$q$.

Show by an example that nonoverlapping geodesics can cross each other an arbitrary number of times.
\end{thm}

\begin{thm}{Exercise}\label{ex:min-geod+plane}
Assume a smooth surface $\Sigma$ is mirror-symmetric with respect to a plane $\Pi$.
Show that no shortest path in $\Sigma$ can {}\emph{cross} $\Pi$ more than once.

In other words, if you travel along а shortest path, then you change sides of $\Pi$ at most once. 
\end{thm}

{

\begin{wrapfigure}{r}{40 mm}
\vskip-8mm
\centering
\includegraphics{mppics/pic-250}
\vskip-0mm
\end{wrapfigure}

\begin{thm}{Advanced exercise}\label{ex:milka}
Let $\Sigma$ be a smooth closed strictly convex surface 
in $\mathbb{R}^3$ 
and $\gamma\:[0,\ell]\z\to \Sigma$ be a unit-speed minimizing geodesic.
Set $p\z=\gamma(0)$, $q=\gamma(\ell)$, and 
\[p^s=\gamma(s)-s\cdot\gamma'(s).\] 


Show that for any $s\in (0,\ell)$,
one cannot see $q$ from $p^s$;
that is, the line segment $[p^s,q]$ intersects $\Sigma$ at a point distinct from~$q$.

Show that the statement does not hold without assuming that $\gamma$ is minimizing.
\end{thm}

}

\begin{thm}{Exercise}\label{ex:round-sphere}
Let $\Sigma$ be a smooth closed surface.
Suppose that for any $p,q\z\in \Sigma$ the distance $\dist{p}{q}\Sigma$ depends only on the distance $\dist{p}{q}{\mathbb{R}^3}$.
Show that $\Sigma$ is a round sphere.
\end{thm}



\section{Liberman's lemma}

A version of the following lemma was used by Joseph Liberman \cite{liberman}.

\begin{thm}{Lemma}
\label{lem:liberman}
\index{Liberman's lemma}
Let $f$ be a smooth locally convex function defined on an open subset of the plane.
Suppose $t\mapsto \gamma(t)\z=(x(t),y(t),z(t))$ is a unit-speed geodesic on the graph $z=f(x,y)$.
Then $t\mapsto z(t)$ is a convex function; that is, $z''(t)\ge 0$ for any~$t$.
\end{thm}

\parbf{Proof.}
Choose the orientation on the graph so that the unit normal vector $\Norm$ always points up;
that is, $\Norm$ has a positive $z$-coordinate at each point.
Let us use the shortcut $\Norm(t)$ for $\Norm(\gamma(t))$.

Since $\gamma$ is a geodesic, we have $\gamma''(t)\perp\T_{\gamma(t)}$,
or equivalently the acceleration $\gamma''(t)$ is proportional to $\Norm(t)$ for any~$t$.
Furthermore,
\[\gamma''=k\cdot\Norm,\]
where $k=k(t)$ is the normal curvature of the graph at $\gamma(t)$ in the direction of $\gamma'(t)$.

Therefore,
\[z''=k\cdot\cos\theta,
\eqlbl{eq:z''}\]
where $\theta=\theta(t)$ denotes the angle between $\Norm(t)$ and the $z$-axis.

Since $\Norm$ points up, we have $\theta(t)<\tfrac\pi2$, or equivalently $\cos\theta>0$.

Since $f$ is convex, the tangent plane supports the graph from below at any point;
in particular, $k(t)\ge 0$ for any~$t$.
It follows that the right-hand side in \ref{eq:z''} is nonnegative;
whence the statement follows.
\qeds

\begin{thm}{Exercise}\label{ex:closed-liberman}
Let $\Sigma$ be the graph of a locally convex function defined on an open subset of the plane.
Show that $\Sigma$ has no closed geodesics.
\end{thm}


\begin{thm}{Exercise}\label{ex:rho''}
Assume $\gamma$ is a unit-speed geodesic on a smooth convex surface $\Sigma$, and a point $p$ lies in the interior of the convex set bounded by~$\Sigma$.
Set $\rho(t)=|p-\gamma(t)|^2$.
Show that $\rho''(t)\le 2$ for any~$t$.
\end{thm}



\section{Total curvature of geodesics}

Recall that $\tc\gamma$ denotes the total curvature of the curve~$\gamma$;
see~\ref{sec:Total curvature}.

\begin{thm}{Exercise}\label{ex:tc-spherical-image}
Let $\gamma$ be a geodesic on an oriented smooth surface $\Sigma$
with unit normal field $\Norm$.
Show that $\length(\Norm\circ\gamma)\ge \tc\gamma$.
\end{thm}


\begin{thm}{Theorem}\label{thm:usov}
Assume $\Sigma$ is the graph of a convex $\ell$-Lipschitz function $f$ defined on an open set in the $(x,y)$-plane.
Then the total curvature of any geodesic in $\Sigma$ is at most $2\cdot \ell$.
\end{thm}

This theorem was first proved by Vladimir Usov \cite{usov}.

\parbf{Proof.}
Let $t\mapsto\gamma(t)=(x(t),y(t),z(t))$ be a unit-speed geodesic on~$\Sigma$.
According to Liberman's lemma (\ref{lem:liberman}), the function
$t\mapsto z(t)$ is convex.

Since the slope of $f$ is at most $\ell$, we have
\[|z'(t)|\le \frac{\ell}{\sqrt{1+\ell^2}}\]
for any $t$.
We can assume that $\gamma$ is defined on the interval $[a,b]$.
Then
\[
\begin{aligned}
\int_a^b z''(t) dt&=z'(b)-z'(a)\le 
 2\cdot \frac{\ell}{\sqrt{1+\ell^2}}.
\end{aligned}
\eqlbl{eq:intz''}
\]

Also, note that $z''$ is the projection of $\gamma''$ to the $z$-axis.
The slope of the tangent plane $\T_{\gamma (t)} \Sigma$ cannot be greater than $\ell$ for any~$t$.
Because $\gamma ''$ is perpendicular to that plane, we have that
\[|\gamma'' (t)| \le z''(t)\cdot\sqrt{1+ \ell ^2}.\]

By \ref{eq:intz''}, we get that
\begin{align*}
\tc\gamma&=\int_a^b|\gamma'' (t)|\cdot dt\le 
\\
&\le \sqrt{1+ \ell ^2}\cdot \int_a^b z''(t)\cdot dt\le 
\\
&\le 2\cdot \ell.
\end{align*}
\qedsf

By the following exercise, the estimate in the theorem is optimal.

\begin{thm}{Exercise}\label{ex:usov-exact}
Let $\Sigma$ be the graph $z=\ell\cdot\sqrt{x^2+y^2}$ with the origin removed.
Show that any both-side-infinite geodesic $\gamma$ in $\Sigma$ has total curvature exactly $2\cdot \ell$.
\end{thm}

\begin{thm}{Exercise}\label{ex:ruf-bound-mountain}
Assume $f$ is a smooth convex $\tfrac32$-Lipschitz function defined on the $(x,y)$-plane.
Show that any geodesic $\gamma$ on the graph $z\z=f(x,y)$ is simple;
that is, it has no self-intersections.

Construct a convex $2$-Lipschitz function defined on the plane
with a nonsimple geodesic $\gamma$ in its graph.
\end{thm}


\begin{thm}{Theorem}\label{thm:tc-of-mingeod}
Suppose a smooth surface $\Sigma$ bounds a convex set $K$.
Assume $B(0,\epsilon)\subset K\subset B(0,1)$.
Then the total curvatures of any shortest path in $\Sigma$ can be bounded in terms of~$\epsilon$. 
\end{thm}

\begin{wrapfigure}{r}{47 mm}
\vskip-0mm
\centering
\includegraphics{mppics/pic-83}
\vskip-0mm
\end{wrapfigure} 

\begin{thm}{Proof-guided exercise}\label{ex:bound-tc}
Let $\Sigma$ be as in the theorem, and $\gamma$ be a unit-speed shortest path in~$\Sigma$.
Denote by $\Norm(t)$ the unit normal vector at $\gamma(t)$ that points outside $\Sigma$;
denote by $\theta(t)$ the angle between $\Norm(t)$ and the direction from the origin to $\gamma(t)$.
Set $\rho(t)\z=|\gamma(t)|^2$; denote by $k(t)$ the curvature of $\gamma$ at~$t$.

\begin{subthm}{ex:bound-tc:a}
Show that $\cos[\theta(t)]\ge \epsilon$ for any~$t$.
\end{subthm}

\begin{subthm}{ex:bound-tc:b}
 Show that $|\rho'(t)|\le 2$ for any~$t$.
\end{subthm}

\begin{subthm}{ex:bound-tc:c}
 Show that 
\[\rho''(t)=2-2\cdot k(t)\cdot \cos \theta(t)\cdot |\gamma(t)|\]
for any~$t$.
\end{subthm}

\begin{subthm}{ex:bound-tc:d}
 Use the nearest-point projection from the unit sphere to $\Sigma$ to show that 
\[\length \gamma\le \pi.\]
\end{subthm}

\begin{subthm}{ex:bound-tc:e}
Conclude that $\tc\gamma\le 100/\epsilon^2$.
\end{subthm}

\end{thm}

\parit{Remark.}
The obtained estimate tends to infinity as $\epsilon \to 0$,
but there is also an estimate that does not depend on $\epsilon$;
this is a result by Nina Lebedeva and the first author \cite{lebedeva-petrunin}.
Aleksei Pogorelov proposed the hypothesis that there exists a bound on the length of the spherical image of a shortest path \cite{pogorelov}.
According to \ref{ex:tc-spherical-image}, this hypothesis is stronger,
but counterexamples have been found for all of its possible variations \cite{zalgaller,milka,usov,pach}.
