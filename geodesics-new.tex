\chapter{Geodesics}
\label{chap:geodesics}


\section{Definition}

A smooth curve $\gamma$ in a smooth surface $\Sigma$ is called a \index{geodesic}\emph{geodesic} if, for any~$t$, the acceleration $\gamma''(t)$ is perpendicular to the tangent plane $\T_{\gamma(t)}$.

\begin{thm}{Exercise}\label{ex:helix-geodesic}
Show that the helix $\gamma(t)\z\df(\cos t,\sin t, t)$ is a geodesic on the cylindrical surface defined by the equation $x^2+y^2=1$.
\end{thm}


Physically, geodesics can be understood as the trajectories of particles that slide on $\Sigma$ without friction.
Indeed, since there is no friction, the force that keeps the particle on $\Sigma$ must be perpendicular to~$\Sigma$.
Therefore, by Newton's second law of motion,
we get that the acceleration $\gamma''$ is perpendicular to $\T_{\gamma(t)}$.

From a physics point of view, the following lemma is a corollary of the law of conservation of energy.


\begin{thm}{Lemma}\label{lem:constant-speed}
Let $\gamma$ be a geodesic in a smooth surface~$\Sigma$. 
Then $|\gamma'|$ is constant.

Moreover, for any $\lambda\in\mathbb{R}$, the curve 
$\gamma_{\lambda}\df \gamma (\lambda\cdot t)$ is a geodesic as well. 
In other words, any geodesic has a constant speed, and multiplying its parameter by a constant yields another geodesic.
\end{thm}


\parit{Proof.} 
Since $\gamma'(t)$ is a tangent vector at $\gamma(t)$,
we have that $\gamma''(t)\z\perp\gamma'(t)$, or equivalently $\langle\gamma'',\gamma'\rangle=0$ for any~$t$.
Whence 
\begin{align*}
\langle\gamma',\gamma'\rangle'&=2\cdot \langle\gamma'',\gamma'\rangle=0.
\end{align*}
That is, $|\gamma'|^2=\langle\gamma',\gamma'\rangle$ is constant.

The second part follows since 
$\gamma_{\lambda}''(t) =\lambda^2\cdot \gamma''(\lambda t)$.
\qeds


The statement in the following exercise is called \index{Clairaut's relation}\emph{Clairaut's relation};
it can be obtained from the lemma above and the conservation of angular momentum.

\begin{thm}{Exercise}\label{ex:clairaut}
Let $\gamma$ be a geodesic on a smooth surface of revolution.
Suppose $r(t)$ denotes the distance from $\gamma(t)$ to the axis of rotation
and $\theta(t)$ --- the angle between $\gamma'(t)$ and the latitudinal circle thru $\gamma(t)$. 

Show that the value $r(t)\cdot \cos\theta(t)$ is constant. 
\end{thm}


Recall that an {}\emph{asymptotic line} is a curve in the surface with vanishing normal curvature.

\begin{thm}{Exercise}\label{ex:asymptotic-geodesic}
Assume a curve $\gamma$ is a geodesic and, at the same time, an asymptotic line of a smooth surface~$\Sigma$.
Show that $\gamma$ is a straight-line segment.
\end{thm}


\begin{thm}{Exercise}\label{ex:reflection-geodesic}
Assume a smooth surface $\Sigma$ is mirror-symmetric with respect to a plane~$\Pi$.
Suppose $\Sigma$ and $\Pi$ intersect along a smooth curve~$\gamma$.
Show that $\gamma$, parametrized by arc-length, is a geodesic on~$\Sigma$.
\end{thm}



\section{Existence and uniqueness}

The following proposition says that the position of a particle that travels without friction depends smoothly on its initial position, initial velocity, and time. 

\begin{thm}{Proposition}\label{prop:geod-existence} 
Let $\Sigma$ be a smooth surface without boundary.
Given a tangent vector ${\vec v}$ to $\Sigma$ at a point $p$,
there is a unique geodesic $\gamma\:\mathbb{I}\to \Sigma$ defined on a maximal open interval $\mathbb{I}\ni 0$ that starts at $p$ with velocity vector ${\vec v}$;
that is, $\gamma(0)=p$ and $\gamma'(0)={\vec v}$.

Moreover,
\begin{subthm}{prop:geod-existence:smooth} the map $(p,{\vec v},t)\mapsto \gamma(t)$ is smooth in its domain of definition.
\end{subthm}

\begin{subthm}{prop:geod-existence:whole} if $\Sigma$ is proper, then $\mathbb{I}=\mathbb{R}$; that is, the maximal interval is the entire real line.
\end{subthm}

\end{thm}

A surface that satisfies the conclusion of \ref{SHORT.prop:geod-existence:whole} for any tangent vector ${\vec v}$ is said to be \index{geodesically complete}\emph{geodesically complete}.
So part \ref{SHORT.prop:geod-existence:whole} says that any proper surface is geodesically complete.
The latter statement is a part of the \index{Hopf--Rinow theorem}\emph{Hopf--Rinow theorem} \cite{hopf-rinow}.

The proof of this proposition uses the theorem on the existence of solutions for an initial value problem (\ref{thm:ODE}).
The following lemma shows that the condition of being a geodesic can be stated as a second-order differential equation.

\begin{thm}{Lemma}\label{lem:geodesic=2nd-order}
Let $f$ be a smooth function defined on an open domain in $\mathbb{R}^2$.
A smooth curve $t\mapsto \gamma(t)=(x(t),y(t),z(t))$ is a geodesic in the graph $z=f(x,y)$ if and only if $z(t)=f(x(t),y(t))$ for any $t$ and the functions $t\mapsto x(t)$ and $t\mapsto y(t)$
satisfy the following differential equation
\[
\begin{cases}
x''=g(x,y,x',y'),
\\
y''=h(x,y,x',y'),
\end{cases}
\]
where $g$ and $h$ are smooth functions of four variables determined by~$f$.
\end{thm}

The proof of the lemma is done by direct calculations.

\parit{Proof.} In the following calculations, we often omit the arguments;
in other words, we may use shortcuts $x=x(t)$, $f=f(x,y)=f(x(t),y(t))$, and so on.

First, let us express $z''(t)$ in terms of $f$, $x(t)$, and $y(t)$.
\[
\begin{aligned}
z''&=f(x,y)''=
\\
&=\left(f_x\cdot x'+ f_y\cdot y'\right)'=
\\
&=
f_{xx}\cdot (x')^2
+
f_x\cdot x''
+ 2\cdot f_{xy}\cdot x'\cdot y'
+
f_{yy}\cdot (y')^2
+
f_y\cdot y''.
\end{aligned}
\eqlbl{eq:def-geod}
\]

Observe that the equation 
\[\gamma''(t)\perp\T_{\gamma(t)} \eqlbl{eq:def-def-geod} \] 
means that 
$\gamma''$ is perpendicular to the two basis vectors in $\T_{\gamma(t)}$.
Therefore, the vector equation \ref{eq:def-def-geod} can be rewritten as the following system of two real equations
\[
\begin{cases}
\langle \gamma'',s_x\rangle=0,
\\
\langle\gamma'',s_y\rangle=0,
\end{cases}
\]
where $s(x,y)\df (x,y,f(x,y))$, $x=x(t)$, and $y=y(t)$.
Observe that 
$s_x=(1,0, f_x)$ 
and 
$s_y=(0,1, f_y)$.
Since $\gamma''\z=(x'',y'',z'')$, we can rewrite the system in the following way:
\[
\begin{cases}
x''+ f_x\cdot z''=0,
\\
y''+ f_y\cdot z''=0.
\end{cases}
\]
It remains to plug in \ref{eq:def-geod} for $z''$, combine the similar terms, and simplify.
\qeds


\parit{Proof of \ref{prop:geod-existence}.}
Let $z=f(x,y)$ be a description of $\Sigma$ in tangent-normal coordinates at~$p$.
By Lemma \ref{lem:geodesic=2nd-order}, the condition $\gamma''(t)\perp\T_{\gamma(t)}$ can be written as a second-order differential equation.
Applying the existence and uniqueness of the initial value problem (\ref{thm:ODE}) we get the existence and uniqueness of a geodesic $\gamma$ in an interval $(-\epsilon,\epsilon)$ for some $\epsilon>0$.

Let us extend $\gamma$ to a maximal open interval $\mathbb{I}$.
Suppose there is another geodesic $\gamma_1$ with the same initial data that is defined on a maximal open interval $\mathbb{I}_1$.
Suppose $\gamma_1$ splits from $\gamma$ at some $t_0>0$;
that is, $\gamma_1$ coincides with $\gamma$ on the interval $[0,t_0)$, but they are different on any interval $[0,t_0+\delta)$ for $\delta>0$.
By continuity, $\gamma_1(t_0)=\gamma(t_0)$, and $\gamma'(t_0)=\gamma'(t_0)$.
Applying uniqueness of the initial value problem (\ref{thm:ODE}) again, we get that $\gamma_1$ coincides with $\gamma$ in a small neighborhood of $t_0$ --- a contradiction.

The same argument shows that $\gamma_1$ cannot split from $\gamma$ at $t_0<0$.
It follows that $\gamma_1=\gamma$;
in particular, $\mathbb{I}_1=\mathbb{I}$.

Part \ref{SHORT.prop:geod-existence:smooth} follows since the solution of the initial value problem depends smoothly on the initial data (\ref{thm:ODE}).

Assume \ref{SHORT.prop:geod-existence:whole} does not hold;
that is, the maximal interval $\mathbb{I}$ is a proper subset of the real line $\mathbb{R}$.
Without loss of generality, we may assume that $b=\sup\mathbb{I}<\infty$.
(If not, switch the direction of~$\gamma$.)

By \ref{lem:constant-speed} $|\gamma'|$ is constant; in particular, $t\mapsto \gamma(t)$ is a uniformly continuous function.
Therefore, the limit point
$q=\lim_{t\to b}\gamma(t)$
is defined.
Since $\Sigma$ is a proper surface, $q\in \Sigma$. 

Applying the argument above in a tangent-normal coordinate chart at $q$, we conclude that $\gamma$ can be extended as a geodesic beyond~$q$.
Therefore, $\mathbb{I}$ is not maximal --- a contradiction.
\qeds

\begin{thm}{Exercise}\label{ex:round-torus}
Let $\Sigma$ be a smooth torus of revolution; that is,
a smooth surface of revolution with a closed generatrix.
Show that any closed geodesic on $\Sigma$ is noncontractible.

(In other words, if $s\:\mathbb{R}^2\to \Sigma$ is the natural bi-periodic parametrization of $\Sigma$, then
there is no closed curve $\gamma$ in $\mathbb{R}^2$ such that $s\circ\gamma$ is a geodesic.)
\end{thm}


\section{Exponential map}\label{sec:exp}

Let $\Sigma$ be a smooth surface and $p\in \Sigma$.
Given a tangent vector ${\vec v}\in \T_p$, consider a geodesic $\gamma_{\vec v}$ in $\Sigma$ that starts at $p$ with initial velocity~$\vec v$; 
that is, $\gamma(0)=p$ and $\gamma'(0)={\vec v}$.

The map 
\[\exp_p\:\vec v\mapsto \gamma_{\vec v}(1)\]
is called \index{exponential map}\emph{exponential}.%
\footnote{There is a good reason to call this map \textit{exponential}, but it is far from the subject.}
By \ref{prop:geod-existence}, the map $\exp_p\:\T_p\to \Sigma$ is smooth, and it is defined in a neighborhood of zero in the tangent plane $\T_p$.
Moreover, if $\Sigma$ is proper, then $\exp_p$ is defined on the entire plane $\T_p$.

The exponential map
is defined on the tangent plane, which is a smooth surface,
and its target is another smooth surface.
Observe that one can identify the plane $\T_p$
with its tangent plane $\T_0\T_p$ so the differential $d_0(\exp_p)\:\vec v\mapsto D_{\vec v}\exp_p$ maps $\T_p$ to itself.
Furthermore, note that by Lemma \ref{lem:constant-speed}, this differential is the identity map; that is, $d_0\exp_p(\vec v)=
\vec v$ for any $\vec v\in \T_p$.

Summarizing, we get the following statement:

\begin{thm}{Observation}\label{obs:d(exp)=1}
Let $\Sigma$ be a smooth surface and $p\in \Sigma$.
Then:

\begin{subthm}{}
The exponential map $\exp_p$ is smooth, and its domain $\Dom(\exp_p)$ contains a neighborhood of the origin in $\T_p$.
Moreover, if $\Sigma$ is proper, then $\Dom(\exp_p)=\T_p$
\end{subthm}

\begin{subthm}{}
The differential $d_0(\exp_p)\:\T_p\to \T_p$ is the identity map.
\end{subthm}

\end{thm}

In fact, it is easy to see that $\Dom(\exp_p)$ --- the domain of definition of $\exp_p$ --- is an open \index{star-shaped}\emph{star-shaped} region of $\T_p$;
the latter means that if $\vec v\in \Dom(\exp_p)$, then $\lambda\cdot\vec v\in \Dom(\exp_p)$ for any $0\le \lambda\le 1$.


\begin{thm}{Proposition}\label{prop:exp}
Let $p$ be a point on a smooth surface $\Sigma$ (without boundary).
Then there is $r_p>0$ such that
the exponential map $\exp_p$ is defined on the open ball $B_p\df B(0,r_p)_{\T_p}$,
and the restriction $\exp_p|_{B_p}$ is a smooth regular parametrization of a neighborhood of $p$ in~$\Sigma$.

Moreover, we have {}\emph{local control} on $r_p$;
that is, for any $q\in \Sigma$ there is $\epsilon>0$ such that if $\dist{p}{q}\Sigma<\epsilon$, then $r_p\ge\epsilon$.
\end{thm}

The proof of the proposition uses the observation and the inverse function theorem (\ref{thm:inverse}).

\parit{Proof.}
Let $z=f(x,y)$ be a local graph representation of $\Sigma$ in tangent-normal coordinates at~$p$.
In this case, the $(x,y)$-plane coincides with the tangent plane $\T_p$.

Denote by $s$ the composition of the exponential map $\exp_p$ with the orthogonal projection $(x,y,z)\mapsto (x,y)$.
By \ref{obs:d(exp)=1}, the differential $d_0s$ is the identity;
in other words, the Jacobian matrix of this map at $0$ is the identity.
Applying the inverse function theorem (\ref{thm:inverse}) we get the first part of the proposition.

The second part of the inverse function theorem (\ref{thm:inverse}) 
 guarantees that for a given point $q\in \Sigma$, the radii of the balls $B_p$
 for all points $p$ sufficiently close to $q$
 can be taken uniformly 
 bounded below by a positive number $\epsilon_q >0$.
\qeds

Given $p\in \Sigma$, the least upper bound on $r_p$ that satisfies \ref{prop:exp} is called the \index{injectivity radius}\emph{injectivity radius} of $\Sigma$ at $p$;
it is denoted by $\inj(p)$.
The proposition states that the \textit{injectivity radius is positive and locally bounded away from zero}.
In fact, the function \textit{$\inj\:\Sigma\to (0,\infty]$ is continuous};
the latter was proved by Wilhelm Klingenberg \cite[5.4]{gromoll-klingenberg-meyer}. 

The proof of the following statement will be indicated in \ref{ex:inj-rad}.

\begin{thm}{Proposition}\label{prop:inj-rad}
Let $p$ be a point on a smooth surface $\Sigma$ (without boundary).
If $\exp_p$ is injective in $B_p=B(0,r)_{\T_p}$, then the restriction $\exp_p|_{B_p}$ is a diffeomorphism between $B_p$ and its image in~$\Sigma$.

In other words, the injectivity radius at $p$ can be defined as the least upper bound on the $r$ such that $\exp_p$ is injective in the ball $B(0,r)_{\T_p}$.
\end{thm}

\section{Shortest paths are geodesics}

\begin{thm}{Proposition}\label{prop:gamma''}
Let $\Sigma$ be a smooth surface.
Then any shortest path $\gamma$ in $\Sigma$ parametrized proportional to its arc-length is a geodesic in~$\Sigma$.
In particular, $\gamma$ is a smooth curve.

A partial converse to the first statement also holds: a sufficiently short arc of any geodesic is a shortest path.
More precisely, any point $p$ in $\Sigma$ has a neighborhood $U$ such that any geodesic that lies entirely in $U$ is a shortest path.
\end{thm}

As one can see from the following exercise, geodesics might fail to be shortest.
A geodesic that is also a shortest path is called a \index{minimizing geodesic}\emph{minimizing geodesic}.

\begin{thm}{Exercise}\label{ex:helix=geodesic}
Let $\Sigma$ be the cylindrical surface described by the equation $x^2\z+y^2=1$.
Show that the helix $\gamma\:[0,2\cdot\pi]\to \Sigma$ defined by $\gamma(t)\z\df(\cos t,\sin t, t)$
is a geodesic, but not a shortest path on~$\Sigma$.
\end{thm}

A formal proof of the proposition will be given in Section~\ref{sec:proof-of-gamma''}.

The following informal physical explanation might be sufficiently convincing.
In fact, if one assumes that $\gamma$ is smooth, then it is easy to convert this explanation into a rigorous proof.

\parit{Informal explanation.}
Let us think about a shortest path $\gamma$ as a stable position of a stretched elastic thread that is forced to lie on a frictionless surface.
Since it is frictionless, the force density $\vec n=\vec n(t)$ that keeps $\gamma$ in the surface must be proportional to the normal vector to the surface at~$\gamma(t)$.

The tension in the thread has to be the same at all points. (Otherwise, the thread would move back or forth, and it would not be stable.)
Denote the tension by $\tau$.

We can assume that $\gamma$ has unit speed;
in this case, the net force from tension along the arc $\gamma_{[t_0,t_1]}$ is $\tau\cdot(\gamma'(t_1)-\gamma'(t_0))$.
Hence, the density of the net force from tension at $t_0$ is 
\begin{align*}
\vec f(t_0)&=\lim_{t_1\to t_0}\tau\cdot\frac{\gamma'(t_1)-\gamma'(t_0)}{t_1-t_0}=
\\
&=\tau\cdot\gamma''(t_0).
\end{align*}
According to Newton's second law of motion, we have 
$\vec f+\vec n=0$.
The latter implies that $\gamma''(t)\perp\T_{\gamma(t)}\Sigma$.
\qeds

\begin{thm}{Corollary}
Let $\Sigma$ be a smooth surface, $p\in\Sigma$ and $r\z\le \inj(p)$.
Then the exponential map $\exp_p$ is a diffeomorphism from $B(0,r)_{\T_p}$ to $B(p,r)_\Sigma$.
\end{thm}

\parit{Proof.}
By \ref{prop:inj-rad}, the restriction of $\exp_p$ to $B_p={B(0,r)_{\T_p}}$ is a diffeomorphism to its image $\exp_p(B_p)\subset \Sigma$.

Evidently, $B(p,r)_\Sigma\supset\exp_p(B_p)$.
By \ref{prop:gamma''}, $B(p,r)_\Sigma\subset\exp_p(B_p)$, hence the result.
\qeds

According to the corollary, the restriction $\exp_p|_{\T_p}$ admits an inverse map called the \index{logarithmic map}\emph{logarithmic map at $p$};
it is denoted by \[\log_p\:B(p,r)_\Sigma\to B(0,r)_{\T_p}.\]

Note that according to the proposition above, any shortest path parametrized by its arc-length is a smooth curve.
This observation helps to solve the following two exercises:

\begin{thm}{Exercise}\label{ex:two-min-geod}
Show that two distinct shortest paths can cross each other at most once.
More precisely, if two shortest paths have two distinct common points $p$ and $q$, then either $p$ and $q$ are the endpoints of both shortest paths, or both shortest paths share a common arc from $p$ to~$q$.

Show by an example that nonoverlapping geodesics can cross each other an arbitrary number of times.
\end{thm}

\begin{thm}{Exercise}\label{ex:min-geod+plane}
Assume a smooth surface $\Sigma$ is mirror-symmetric with respect to a plane $\Pi$.
Show that no shortest path $\alpha$ in $\Sigma$ can {}\emph{cross} $\Pi$ more than once.


In other words, if you travel along $\alpha$, then you change sides of $\Pi$ at most once. 
\end{thm}

{

\begin{wrapfigure}{r}{40 mm}
\vskip-10mm
\centering
\includegraphics{mppics/pic-250}
\vskip-0mm
\end{wrapfigure}

\begin{thm}{Advanced exercise}\label{ex:milka}
Let $\Sigma$ be a smooth closed strictly convex surface 
in $\mathbb{R}^3$ 
and $\gamma\:[0,\ell]\z\to \Sigma$ be a unit-speed minimizing geodesic.
Set $p\z=\gamma(0)$, $q=\gamma(\ell)$, and 
$$p^s=\gamma(s)-s\cdot\gamma'(s),$$ 
where $\gamma'(s)$ denotes the velocity vector of $\gamma$ at~$s$.

Show that for any $s\in (0,\ell)$,
one cannot see $q$ from $p^s$;
that is, the line segment $[p^s,q]$ intersects $\Sigma$ at a point distinct from~$q$.

Show that the statement does not hold without assuming that $\gamma$ is minimizing.
\end{thm}

}

\begin{thm}{Exercise}\label{ex:round-sphere}
Let $\Sigma$ be a smooth surface.
Suppose that for any $p,q\z\in \Sigma$ the distance $\dist{p}{q}\Sigma$ depends only on the distance $\dist{p}{q}{\mathbb{R}^3}$.
Show that $\Sigma$ is a round sphere.
\end{thm}


\section{Liberman's lemma}

The following lemma is a smooth analog of a lemma proved by Joseph Liberman \cite{liberman}.

\begin{thm}{Lemma}
\label{lem:liberman}
\index{Liberman's lemma}
Let $f$ be a smooth locally convex function defined on an open subset of the plane.
Suppose $t\mapsto \gamma(t)\z=(x(t),y(t),z(t))$ is a unit-speed geodesic on the graph $z=f(x,y)$.
Then $t\mapsto z(t)$ is a convex function; that is, $z''(t)\ge 0$ for any~$t$.
\end{thm}

\parit{Proof.}
Choose the orientation on the graph so that the unit normal vector $\Norm$ always points up;
that is, $\Norm$ has a positive $z$-coordinate at each point.
Let us use the shortcut $\Norm(t)$ for $\Norm(\gamma(t))$.

Since $\gamma$ is a geodesic, we have $\gamma''(t)\perp\T_{\gamma(t)}$,
or equivalently $\gamma''(t)$ is proportional to $\Norm(t)$ for any~$t$.
Furthermore,
\[\gamma''=k\cdot\Norm,\]
where $k=k(t)$ is the normal curvature at $\gamma(t)$ in the direction of $\gamma'(t)$.

Therefore,
\[z''=k\cdot\cos\theta,
\eqlbl{eq:z''}\]
where $\theta=\theta(t)$ denotes the angle between $\Norm(t)$ and the $z$-axis.

Since $\Norm$ points up, we have $\theta(t)<\tfrac\pi2$, or equivalently
\[\cos\theta>0.\]

Since $f$ is convex, we have that the tangent plane supports the graph from below at any point;
in particular, $k(t)\ge 0$ for any~$t$.
It follows that the right-hand side in \ref{eq:z''} is nonnegative;
whence the statement follows.
\qeds

\begin{thm}{Exercise}\label{ex:closed-liberman}
Let $\Sigma$ be the graph of a locally convex function defined on an open subset of the plane.
Show that $\Sigma$ has no closed geodesics.
\end{thm}


\begin{thm}{Exercise}\label{ex:rho''}
Assume $\gamma$ is a unit-speed geodesic on a smooth convex surface $\Sigma$, and a point $p$ lies in the interior of the convex set bounded by~$\Sigma$.
Set $\rho(t)=|p-\gamma(t)|^2$.
Show that $\rho''(t)\le 2$ for any~$t$.
\end{thm}



\section{Total curvature of geodesics}

Recall that $\tc\gamma$ denotes the total curvature of curve~$\gamma$.

\begin{thm}{Exercise}\label{ex:tc-spherical-image}
Let $\gamma$ be a geodesic on an oriented smooth surface $\Sigma$
with unit normal field $\Norm$.
Show that 
\[\length(\Norm\circ\gamma)\ge \tc\gamma.\]
\end{thm}


\begin{thm}{Theorem}\label{thm:usov}
Assume $\Sigma$ is the graph $z=f(x,y)$ of a convex $\ell$-Lipschitz function $f$ defined on an open set in the $(x,y)$-plane.
Then the total curvature of any geodesic in $\Sigma$ is at most $2\cdot \ell$.
\end{thm}

This theorem was first proved by Vladimir Usov \cite{usov};
an amusing generalization was found by David Berg \cite{berg}.

\parit{Proof.}
Let $t\mapsto\gamma(t)=(x(t),y(t),z(t))$ be a unit-speed geodesic on~$\Sigma$.
According to Liberman's lemma (\ref{lem:liberman}), the function
$t\mapsto z(t)$ is convex.

Since the slope of $f$ is at most $\ell$, we have
\[|z'(t)|\le \frac{\ell}{\sqrt{1+\ell^2}}.\]
If $\gamma$ is defined on the interval $[a,b]$, then
\[
\begin{aligned}
\int_a^b z''(t) dt&=z'(b)-z'(a)\le 
\\
&\le 2\cdot \frac{\ell}{\sqrt{1+\ell^2}}.
\end{aligned}
\eqlbl{eq:intz''}
\]

Also, note that $z''$ is the projection of $\gamma''$ to the $z$-axis.
Since $f$ is $\ell$-Lipschitz, the tangent plane $\T_{\gamma (t)} \Sigma$ cannot have a slope greater than $\ell$ for any~$t$.
Because $\gamma ''$ is perpendicular to that plane, we have that
\[|\gamma'' (t)| \le z''(t)\cdot\sqrt{1+ \ell ^2}.\]

By \ref{eq:intz''}, we get that
\begin{align*}
\tc\gamma&=\int_a^b|\gamma'' (t)|\cdot dt\le 
\\
&\le \sqrt{1+ \ell ^2}\cdot \int_a^b z''(t)\cdot dt\le 
\\
&\le 2\cdot \ell.
\end{align*}
\qedsf

\begin{thm}{Exercise}\label{ex:usov-exact}
Note that the graph $z=\ell\cdot\sqrt{x^2+y^2}$ with the origin removed is a smooth surface; denote it by~$\Sigma$.
Show that any both-side-infinite geodesic $\gamma$ in $\Sigma$ has total curvature exactly $2\cdot \ell$.
\end{thm}

The exercise implies that the estimate in Usov's theorem is optimal.
To see this, mollify the function $f(x,y)=\ell\cdot\sqrt{x^2+y^2}$ in a small neighborhood of the origin while keeping it convex and $\ell$-Lipschitz. 
Note that we can assume that the geodesic $\gamma$ does not enter the smoothed part of the graph.


\begin{thm}{Exercise}\label{ex:ruf-bound-mountain}
Assume $f$ is a smooth convex $\tfrac32$-Lipschitz function defined on the $(x,y)$-plane.
Show that any geodesic $\gamma$ on the graph $z\z=f(x,y)$ is simple;
that is, it has no self-intersections.

Construct a convex $2$-Lipschitz function defined on the $(x,y)$-plane
with a nonsimple geodesic $\gamma$ in its graph $z=f(x,y)$.
\end{thm}


\begin{thm}{Theorem}\label{thm:tc-of-mingeod}
Suppose a smooth surface $\Sigma$ bounds a convex set $K$ in the Euclidean space.
Assume $B(0,\epsilon)\subset K\subset B(0,1)$.
Then the total curvatures of any shortest path in $\Sigma$ can be bounded in terms of~$\epsilon$. 
\end{thm}

The following exercise will guide you thru the proof of the theorem. 

\begin{wrapfigure}{r}{48 mm}
\vskip-2mm
\centering
\includegraphics{mppics/pic-83}
\vskip-0mm
\end{wrapfigure}

\begin{thm}{Exercise}\label{ex:bound-tc}
Let $\Sigma$ be as in the theorem, and $\gamma$ be a unit-speed shortest path in~$\Sigma$.
Denote by $\Norm(t)$ the unit normal vector at $\gamma(t)$ that points outside $\Sigma$;
denote by $\theta(t)$ the angle between $\Norm(t)$ and the direction from the origin to $\gamma(t)$.
Set $\rho(t)\z=|\gamma(t)|^2$; denote by $k(t)$ the curvature of $\gamma$ at~$t$.

\begin{subthm}{ex:bound-tc:a}
Show that $\cos[\theta(t)]\ge \epsilon$ for any~$t$.
\end{subthm}

\begin{subthm}{ex:bound-tc:b}
 Show that $|\rho'(t)|\le 2$ for any~$t$.
\end{subthm}

\begin{subthm}{ex:bound-tc:c}
 Show that 
\[\rho''(t)=2-2\cdot k(t)\cdot \cos \theta(t)\cdot |\gamma(t)|\]
for any~$t$.
\end{subthm}

\begin{subthm}{ex:bound-tc:d}
 Use the closest-point projection from the unit sphere to $\Sigma$ to show that 
\[\length \gamma\le \pi.\]
\end{subthm}

\begin{subthm}{ex:bound-tc:e}
Conclude that $\tc\gamma\le 100/\epsilon^2$.
\end{subthm}

\end{thm}

\parit{Remark.}
The obtained bound of the total curvature goes to infinity as $\epsilon\to 0$,
but there is a bound that is independent of $\epsilon$;
it is a result of Nina Lebedeva and the first author \cite{lebedeva-petrunin}.
According to Exercise~\ref{ex:tc-spherical-image}, this statement would also follow if the length of the spherical image of $\gamma$ could be bounded above; 
that is, if $\length(\Norm\circ\gamma)\le C$ for a universal constant~$C$.
The latter was conjectured by Aleksei Pogorelov \cite{pogorelov};
counterexamples to the different forms of this conjecture were found 
by Viktor Zalgaller \cite{zalgaller},
Anatoliy Milka \cite{milka},
and Vladimir Usov \cite{usov};
these results were partly rediscovered later 
by J\'{a}nos Pach \cite{pach}.
