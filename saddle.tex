\chapter{Supporting surfaces}
\label{chap:surface-support}

\section{Definitions}

Assume two surfaces $\Sigma_1$ and $\Sigma_2$ have a common point~$p$.
If there is a neighborhood $U$ of $p$ such that $\Sigma_1\cap U$ lies on one side from $\Sigma_2$ in $U$, then we say that $\Sigma_2$ \index{local!support}\emph{locally supports} $\Sigma_1$ at~$p$.

Let us describe $\Sigma_2$ locally at $p$ as a graph $z=f_2(x,y)$ in tangent-normal coordinates at~$p$.
If $\Sigma_2$ locally supports $\Sigma_1$ at $p$, then  all points of $\Sigma_1$ near $p$ lie either above or below the graph $z=f_2(x,y)$.

In both cases, the surfaces $\Sigma_1$ and $\Sigma_2$ have a common tangent plane at~$p$.
So we can write both as graphs $z=f_1(x,y)$ and $z=f_2(x,y)$ in the common tangent-normal coordinates at~$p$.
Note that $\Sigma_2$ locally supports $\Sigma_1$ at $p$ if and only if 
\[\text{either}\quad f_1(x,y)\ge f_2(x,y)
\quad\text{or}\quad
f_1(x,y)\le f_2(x,y)\]
holds for all $(x,y)$ sufficiently close to the origin.

If the surfaces are orientable, we can assume that they are \index{cooriented!surfaces}\emph{cooriented} at $p$;
that is, they have a common unit normal vector at $p$ in the direction of the $z$-axis.
If the normal vectors are opposite, we say that $\Sigma_1$ and $\Sigma_2$ are \index{controriented!surfaces}\emph{counteroriented} at $p$;
in this case, reversing the orientation of one of the surfaces makes them cooriented.

If $\Sigma_2$ locally supports $\Sigma_1$, and they are cooriented at $p$,
then we can say that $\Sigma_1$ supports $\Sigma_2$ \index{support}\emph{from inside} or \index{support}\emph{from outside},
assuming that the normal vector points {}\emph{inside} the domain bounded by the surface $\Sigma_2$ in~$U$.
Using the above notations, $\Sigma_1$ locally supports $\Sigma_2$ from inside (from outside)  if $f_1(x,y)\ge f_2(x,y)$ (respectively $f_1(x,y)\le f_2(x,y)$) for $(x,y)$ in a sufficiently small neighborhood of the origin.

\begin{thm}{Proposition}\label{prop:surf-support}
Let $\Sigma_1$ and $\Sigma_2$ be oriented surfaces.
Assume $\Sigma_1$ locally supports $\Sigma_2$ from inside at the point $p$ (equivalently $\Sigma_2$ locally supports $\Sigma_1$ from outside).
Then 
\[k_1(p)_{\Sigma_1}\ge k_1(p)_{\Sigma_2}\quad\text{and}\quad k_2(p)_{\Sigma_1}\z\ge k_2(p)_{\Sigma_2}.\]
\end{thm}

\begin{thm}{Exercise}\label{ex:surf-support}
Give an example of two surfaces $\Sigma_1$ and $\Sigma_2$ that have a  common point $p$ with a common unit normal vector $\Norm_p$ such that 
$k_1(p)_{\Sigma_1}\z> k_1(p)_{\Sigma_2}$ and $k_2(p)_{\Sigma_1}\z> k_2(p)_{\Sigma_2}$, but $\Sigma_1$ does not support $\Sigma_2$ locally at~$p$.
\end{thm}


\parit{Proof.}
We can assume that $\Sigma_1$ and $\Sigma_2$ are graphs $z=f_1(x,y)$  and $z=f_2(x,y)$ in common tangent-normal coordinates at $p$, so we have $f_1\ge f_2$.

\begin{wrapfigure}{o}{40 mm}
\vskip-4mm
\centering
\includegraphics{mppics/pic-80}
\vskip-0mm
\end{wrapfigure}

Fix a unit vector ${\vec w}\in \T_p\Sigma_1=\T_p\Sigma_2$.
Consider the plane $\Pi$ passing thru $p$ and spanned by the normal vector $\Norm_p$ and ${\vec w}$.
Let $\gamma_1$ and $\gamma_2$ be the curves of intersection of $\Sigma_1$ and $\Sigma_2$ with $\Pi$.

Let us orient $\Pi$ so that the common normal vector $\Norm_p$ for both surfaces at $p$ points to the left from ${\vec w}$.
Further, let us parametrize both curves so that they are running in the direction of ${\vec w}$ at $p$ and are therefore cooriented.
Note that in this case, the curve $\gamma_1$ supports the curve $\gamma_2$ from the right.


By \ref{prop:supporting-circline}, we have the following inequality for the normal curvatures of $\Sigma_1$ and $\Sigma_2$ at $p$ in the direction of ${\vec w}$:
\[k_{\vec w}(p)_{\Sigma_1}\ge k_{\vec w}(p)_{\Sigma_2}.\eqlbl{kw>=kw}\]

According to \ref{obs:k1-k2},
\[k_1(p)_{\Sigma_i}=\min\set{k_{\vec w}(p)_{\Sigma_i}}{{\vec w}\in\T_p, |{\vec w}|=1}\]
for $i=1,2$.
Choose ${\vec w}$ so that $k_1(p)_{\Sigma_1}=k_{\vec w}(p)_{\Sigma_1}$.
Then by \ref{kw>=kw}, we have that
\begin{align*}
k_1(p)_{\Sigma_1}&=k_{\vec w}(p)_{\Sigma_1}\ge
\\
&\ge k_{\vec w}(p)_{\Sigma_2}\ge
\\
&\ge\min_{\vec v}\set{k_{\vec v}(p)_{\Sigma_2}}{}=
\\
&=k_1(p)_{\Sigma_2};
\end{align*}
here we assume $\vec v\in\T_p$ and $|\vec v|=1$.

Similarly, by \ref{obs:k1-k2}, we have that
\[k_2(p)_{\Sigma_i}=\max_{\vec w}\set{k_{\vec w}(p)_{\Sigma_i}}{}.\]
Let us fix ${\vec w}$ so that $k_2(p)_{\Sigma_2}=k_{\vec w}(p)_{\Sigma_2}$.
Then 
\begin{align*}
k_2(p)_{\Sigma_2}&=k_{\vec w}(p)_{\Sigma_2}\le
\\
&\le k_{\vec w}(p)_{\Sigma_1}\le
\\
&\le\max_{\vec v}\set{k_{\vec v}(p)_{\Sigma_1}}{}=
\\
&=k_2(p)_{\Sigma_1}.
\end{align*}
\qedsf

\begin{thm}{Corollary}\label{cor:surf-support}
Let $\Sigma_1$ and $\Sigma_2$ be oriented surfaces.
Assume $\Sigma_1$ locally supports $\Sigma_2$ from inside at the point~$p$.
Then:

\begin{subthm}{cor:surf-support:mean}$H(p)_{\Sigma_1}\ge H(p)_{\Sigma_2}$;
\end{subthm}

\begin{subthm}{cor:surf-support:gauss} If $k_1(p)_{\Sigma_2}\ge 0$, then $K(p)_{\Sigma_1}\ge K(p)_{\Sigma_2}$.
\end{subthm}
 
\end{thm}

\parit{Proof; \ref{SHORT.cor:surf-support:mean}.}
The statement follows from  \ref{prop:surf-support} since 
\[H(p)_{\Sigma_i}=k_1(p)_{\Sigma_i}+k_2(p)_{\Sigma_i}.\]


\parit{\ref{SHORT.cor:surf-support:gauss}.} Since $k_2(p)_{\Sigma_i}\ge k_1(p)_{\Sigma_i}$, and $k_1(p)_{\Sigma_2}\ge 0$, we get that all the principal curvatures 
$k_1(p)_{\Sigma_1}$, 
$k_1(p)_{\Sigma_2}$, 
$k_2(p)_{\Sigma_1}$, and 
$k_2(p)_{\Sigma_2}$ are nonnegative.
By \ref{prop:surf-support}, it implies that
\begin{align*}
K(p)_{\Sigma_1}&=k_1(p)_{\Sigma_1}\cdot k_2(p)_{\Sigma_1}\ge 
\\
&\ge k_1(p)_{\Sigma_2}\cdot k_2(p)_{\Sigma_2}=
\\
&=K(p)_{\Sigma_2}.
\end{align*}
\qedsf

\begin{thm}{Exercise}\label{ex:positive-gauss-0}
Show that any closed surface in a unit ball has a point with Gauss curvature at least~1.
Conclude that any closed surface has a point with strictly positive Gauss curvature.
\end{thm}

\begin{thm}{Exercise}\label{ex:positive-gauss}
Show that any closed surface that lies at distance at most 1 from a straight line has a point with Gauss curvature at least~1.
\end{thm}

\section{Convex surfaces}

A surface that bounds a convex region is called \index{convex!surface}\emph{convex}.

\begin{thm}{Exercise}\label{ex:convex-surf}
Show that the Gauss curvature of any convex smooth surface is nonnegative at each point.
\end{thm}

\begin{thm}{Exercise}\label{ex:convex-lagunov}
Assume $R$ is a convex body in $\mathbb{R}^3$ bounded by a surface with principal curvatures at most~1.
Show that $R$ contains a unit ball.
\end{thm}

Recall that a region $R$ in the Euclidean space is called  {}\emph{strictly convex} if, for any two points $x,y\in R$, any point $z$ between $x$ and $y$ lies in the interior of~$R$.

Evidently, a closed convex region is strictly convex if and only if its boundary does not contain a line segment.
Note that any open convex set is strictly convex;
the cube (as well as any convex polyhedron) gives an example of a non-strictly convex set.


\begin{thm}{Lemma}\label{lem:gauss+=>convexity}
Let $z=f(x,y)$ be the local description of a smooth surface $\Sigma$ in tangent-normal coordinates at some point $p\in\Sigma$.
Assume both principal curvatures of $\Sigma$ are positive at~$p$.
Then the function $f$ is strictly convex in a neighborhood of the origin and has a local minimum at the origin.

In particular, the tangent plane $\T_p$ locally supports $\Sigma$ from outside at~$p$.
\end{thm}

\parit{Proof.}
Since both principal curvatures are positive, by \ref{cor:Shape(ij)}, we have 
\[D^2_{\vec w}f(0,0)=\langle \Shape_p({\vec w}),{\vec w}\rangle\ge k_1(p)>0\] 
for any unit tangent vector ${\vec w}\in\T_p\Sigma$ (which is the $(x,y)$-plane).

By continuity of the function $(x,y,{\vec w})\mapsto D^2_{\vec w}f(x,y)$,
we have that $D^2_{\vec w}f(x,y)>0$ if $\vec w\ne 0$ and $(x,y)$ lies in a sufficiently small neighborhood of the origin.
This property implies that $f$ is a strictly convex function in a neighborhood of the origin in the $(x,y)$-plane (see Section~\ref{sec:analysis}).

Finally, since $\nabla f(0,0)=0$, and $f$ is strictly convex in a neighborhood of the origin, $f$ has a strict local minimum at the origin.
\qeds

\begin{thm}{Exercise}\label{ex:section-of-convex}
Let $\Sigma$ be a smooth surface (without boundary) with positive Gauss curvature.
Show that any connected component of the intersection of $\Sigma$ with a plane $\Pi$ is either a single point or a smooth regular plane curve whose signed curvature has a constant sign.
\end{thm}

The following theorem gives a global description of surfaces with positive Gauss curvature.

\begin{thm}{Theorem}\label{thm:convex-embedded}
Suppose $\Sigma$ is a proper smooth surface with positive Gauss curvature.
Then $\Sigma$ bounds a strictly convex region.
\end{thm}

In fact the theorem holds for surfaces with possible self-intersections. 
This is the so-called {}\emph{Hadamard theorem};
it was proved by James Stoker \cite{stoker} and attributed it to Jacques Hadamard who proved a closely relevant statement \cite[item 23]{hadamard}.

Note that in the proof we have to use that the surface is a connected set;
otherwise, a pair of disjoint spheres would give a counterexample.

\parit{Proof.}
Since the Gauss curvature is positive, we can choose a unit normal field $\Norm$ on $\Sigma$ so that the principal curvatures are positive at any point.
Denote by $R$ the region bounded by $\Sigma$ that lies on the side of $\Norm$;
that is, $\Norm$ points inside $R$ at any point of~$\Sigma$.
(The region $R$ exists by \ref{clm:proper-divides}.)

Let us show that $R$ is {}\emph{locally strictly convex};
that is, for any point $p\in R$, the intersection of $R$ with a small ball centered at $p$ is strictly convex.

Indeed, suppose $z=f(x,y)$ is a local description of $\Sigma$ in the tangent-normal coordinates at~$p$.
By \ref{lem:gauss+=>convexity}, $f$ is strictly convex in a neighborhood of the origin.
In particular, the intersection of a small ball centered at $p$ with the epigraph $z\ge f(x,y)$ is strictly convex.

Since $\Sigma$ is connected, so is $R$;
moreover, any two points in the interior of $R$ can be connected by a polygonal line in the interior of~$R$.

\begin{wrapfigure}{o}{43 mm}
\vskip-0mm
\centering
\includegraphics{mppics/pic-37}
\vskip-0mm
\end{wrapfigure}

Assume the interior of $R$ is not convex;
that is, there are points $x,y\in R$, and a point $z$ between $x$ and $y$ that does not lie in the interior of~$R$.
Consider a polygonal  line $\beta$ from $x$ to $y$ in the interior of~$R$.
Let $y_0$ be the first point on $\beta$ such that the chord $[x,y_0]$ touches $\Sigma$ at some point, say~$z_0$.

Since $R$ is locally strictly convex, $R\z\cap B(z_0,\epsilon)$ is strictly convex for all sufficiently small $\epsilon>0$.
On the other hand, $z_0$ lies between two points in the intersection $[x,y_0]\cap B(z_0,\epsilon)$.
Since $[x,y_0]\subset R$, we arrived to a contradiction.

Therefore, the interior of $R$ is a convex set.
Note that the region $R$ is the closure of its interior, therefore $R$ is convex as well.

Since $R$ is locally strictly convex, its boundary $\Sigma$ contains no line segments.
Therefore, $R$ is strictly convex.
\qeds

\parit{Remark.}
We proved a more general statement.
Namely, {}\emph{any closed connected locally convex region in the Euclidean space is convex}.

\begin{thm}{Exercise}\label{ex:surrounds-disc}
Assume a closed convex surface $\Sigma$ surrounds a unit circle.
Show that there is a point  $p \in \Sigma$ with $K(p)\le 1$. 
\end{thm} 

\begin{thm}{Exercise}\label{ex:small-gauss}
Let $\Sigma$ be a closed convex smooth surface of diameter at least $\pi$;
that is, there is a pair of points $p,q\in\Sigma$ such that $|p-q|\ge \pi$.
Show that $\Sigma$ has a point with Gauss curvature at most~1.
\end{thm}

\begin{thm}{Theorem}\label{thm:convex-closed}
Suppose $\Sigma$ is a closed smooth convex surface.
Then it is a smooth sphere; that is, $\Sigma$ admits a smooth regular parametrization by $\mathbb{S}^2$.\end{thm}

The following exercise will guide you thru the proof of the theorem.

{

\begin{wrapfigure}{r}{33 mm}
\vskip-0mm
\centering
\includegraphics{mppics/pic-78}
\end{wrapfigure}

\begin{thm}{Exercise}\label{ex:convex-proper-sphere}
Assume a convex compact region $R$ contains the origin in its interior and is bounded by a smooth surface~$\Sigma$.

\begin{subthm}{ex:convex-proper-sphere:single}
Show that any half-line that starts at the origin intersects $\Sigma$ at a single point;
that is, there is a positive function $\rho\:\mathbb{S}^2\z\to\mathbb{R}$ such that $\Sigma$ consists of the points $q\z=\rho(\xi)\cdot \xi$ for $\xi\in \mathbb{S}^2$.
\end{subthm}

\begin{subthm}{ex:convex-proper-sphere:smooth}
Show that $\rho\:\mathbb{S}^2\to\mathbb{R}$ is a smooth function.
Conclude that $\xi\mapsto \rho(\xi)\cdot \xi$ is a smooth regular parametrization $\mathbb{S}^2\z\to \Sigma$.
\end{subthm}

\end{thm}

\begin{thm}{Theorem}\label{thm:convex-open}
Suppose $\Sigma$ is an open smooth strictly convex surface.
Then there is a coordinate system in which $\Sigma$ is the graph $z\z=f(x,y)$ of a convex function $f$ defined on a convex open region $\Omega$ of the $(x,y)$-plane.

Moreover, $f(x_n,y_n)\to\infty$ as $(x_n,y_n)\to(x_\infty,y_\infty)\z\in \partial\Omega$.

\end{thm}

\begin{wrapfigure}{r}{33 mm}
\vskip-8mm
\centering
\includegraphics{mppics/pic-1181}
\end{wrapfigure}


\begin{thm}{Exercise}\label{ex:convex-proper-plane}
Assume a strictly convex closed noncompact region $R$ contains the origin in its interior and is bounded by a smooth surface~$\Sigma$.

\begin{subthm}{ex:convex-proper-plane:a}
Show that $R$ contains a half-line $\ell$.
\end{subthm}

\begin{subthm}{ex:convex-proper-plane:b}
Show that any line $m$ parallel to $\ell$ intersects $\Sigma$ at most at one point.
\end{subthm}


\begin{subthm}{ex:convex-proper-plane:c}
Consider an $(x,y,z)$-coordinate system such that the $z$-axis points in the direction of $\ell$.
Show that the projection of $\Sigma$ to the $(x,y)$ plane is an open convex set; denote it by $\Omega$.
\end{subthm}

\begin{subthm}{ex:convex-proper-plane:d}
Conclude that $\Sigma$ is a graph $z=f(x,y)$ of a convex function $f$ defined on $\Omega$.
(It proves the main statement in \ref{thm:convex-open}.)
\end{subthm}

\begin{subthm}{ex:convex-proper-plane:e}
Prove the last statement in \ref{thm:convex-open}.
\end{subthm}


\end{thm}



}

\begin{thm}{Exercise}\label{ex:open+convex=plane}
Show that any open surface $\Sigma$ with positive Gauss curvature is a topological plane;
that is, there is an embedding $\mathbb{R}^2\to\mathbb{R}^3$ with image~$\Sigma$.

Try to show that $\Sigma$ is a smooth plane;
that is, the embedding $f$ can be made smooth and regular.
\end{thm}

\begin{thm}{Exercise}\label{ex:circular-cone}
Show that any open smooth surface $\Sigma$ with positive Gauss curvature
lies inside an infinite circular cone. In other words, there is an $(x,y,z)$-coordinate system in which $\Sigma$ lies in the region $z \z\ge m \cdot\sqrt{x^2 + y^2}$ for some $m > 0$.
\end{thm} 

\begin{thm}{Exercise}\label{ex:intK}
Assume $\Sigma$ is a smooth convex surface with positive Gauss curvature.

\begin{subthm}{ex:intK:4pi}
Show that if $\Sigma$ is closed, then the spherical map $\Norm\:\Sigma\to \mathbb{S}^2$ is a bijection. Conclude that 
\[\iint_\Sigma K=4\cdot\pi.\]
\end{subthm}

\begin{subthm}{ex:intK:2pi}
Show that if $\Sigma$ is open, then  the spherical map $\Norm\:\Sigma\to \mathbb{S}^2$
maps $\Sigma$ bijectively into a subset of a hemisphere. Conclude that 
\[\iint_\Sigma K\le 2\cdot\pi.\]
\end{subthm}

\end{thm}



\section{Saddle surfaces}\label{sec:saddle}

\begin{wrapfigure}{r}{43 mm}
\vskip-8mm
\centering
\includegraphics{asy/saddle}
\vskip0mm
\end{wrapfigure}

A surface is called \index{saddle surface}\emph{saddle} if its Gauss curvature at each point is nonpositive;
in other words, the principal curvatures at each point have opposite signs or at least one of them is zero.

If the Gauss curvature is negative at each point,
then the surface is said to be {}\emph{strictly saddle};
equivalently, this means that the principal curvatures have opposite signs at each point.
Note that in this case, the tangent plane cannot support the surface even locally --- moving along the surface in the principal directions at a given point, one goes above and below the tangent plane.  


\begin{thm}{Exercise}\label{ex:convex-revolution}
Let $f\:\mathbb{R}\to\mathbb{R}$ be a smooth positive function.
Show that the surface of revolution of the graph $y\z=f(x)$ around the $x$-axis
 is saddle if and only if $f$ is convex; that is, if $f''(x)\ge0$ for any~$x$.
\end{thm}

A surface $\Sigma$ is called \index{ruled surface}\emph{ruled} if, for every point $p\in \Sigma$, there is a line segment $\ell_p\subset \Sigma$ passing thru $p$ that is infinite or its endpoint(s) lie on the boundary line of~$\Sigma$.

\begin{thm}{Exercise}\label{ex:ruled=>saddle}
Show that any ruled surface is saddle.
\end{thm}

\begin{thm}{Exercise}\label{ex:saddle-convex}
Let $\Sigma$ be an open strictly saddle surface and $f\:\mathbb{R}^3\z\to\mathbb{R}$ be a smooth convex function.
Show that the restriction of $f$ to $\Sigma$ does not have a point of local maximum.
\end{thm}

A tangent direction on a smooth surface with vanishing normal curvature is called \index{asymptotic direction and line}\emph{asymptotic}.
A smooth regular curve that always runs in an asymptotic direction is called an
{}\emph{asymptotic line}.\label{page:asymptotic line}

Recall that a set $R$ in the plane is called \index{star-shaped}\emph{star-shaped} if there is a point $p\in R$ such that for any $x\in R$ the line segment $[p,x]$ belongs to~$R$.

The statement in the following exercise is due to Dmitri Panov \cite{panov-curves}.

\begin{thm}{Advanced exercise}\label{ex:panov}
Let $\gamma$ be a closed smooth asymptotic line
in the graph $z\z=f(x,y)$ of a smooth function~$f$. 
Assume the graph is strictly saddle in a neighborhood of~$\gamma$.
Show that the region in the $(x,y)$-plane bounded by the projection $\bar \gamma$ of $\gamma$ cannot be star-shaped. 
\end{thm}

\begin{thm}{Advanced exercise}\label{ex:crosss}
Let $\Sigma$ be a smooth surface and $p\in \Sigma$.
Assume $K(p)<0$.
Show that there is a neighborhood $\Omega$ of $p$ in $\Sigma$
such that the intersection of $\Omega$ with the tangent plane $\T_p$ is a union of two smooth curves  \index{transversality}\emph{intersecting transversally} at~$p$.
\end{thm}


\section{Hats}

Note that a \emph{closed surface cannot be saddle}.
Indeed, suppose $\Sigma$ is a closed surface.
Consider the smallest sphere that surrounds~$\Sigma$.
The sphere supports $\Sigma$ at some point $p$, and at this point, the principal curvatures must have the same sign.
The following more general statement is proved using the same idea.

{

\begin{wrapfigure}[6]{r}{45 mm}
\vskip-6mm
\centering
\includegraphics{mppics/pic-73}
\vskip0mm
\end{wrapfigure}

\begin{thm}{Lemma}\label{lem:convex-saddle}
Assume $\Sigma$ is a compact saddle surface, and its boundary line lies in a closed convex  region~$R$.
Then the entire surface $\Sigma$ lies in~$R$.
\end{thm}

\parit{Remark.}
In the case of strictly saddle surface, the lemma can be deduced from \ref{ex:saddle-convex}.

\parit{Proof.}
Arguing by contradiction,
assume there is a point $p\in \Sigma$ that does not lie in~$R$.
Let $\Pi$ be a plane that separates $p$ from $R$; it exists by \ref{lem:separation}.
Denote by $\Sigma'$ the part of $\Sigma$ that lies with $p$ on the same side of~$\Pi$.

}

Since $\Sigma$ is compact, it is surrounded by a sphere;
let $\sigma$ be the circle of intersection of this sphere and $\Pi$.
Consider the smallest spherical dome $\Sigma_0$ with boundary $\sigma$ that surrounds~$\Sigma'$.

Note that $\Sigma_0$ supports $\Sigma$ at some point~$q$.
Without loss of generality, we may assume that $\Sigma_0$ and $\Sigma$ are cooriented at $q$, and $\Sigma_0$ has positive principal curvatures.
In this case, $\Sigma_0$ supports $\Delta$ from outside.
By \ref{cor:surf-support}, we have $K(q)_\Sigma\z\ge K(q)_{\Sigma_0}>0$, a contradiction.
\qeds

\begin{thm}{Exercise}\label{ex:length-of-bry}
Let $\Delta$ be a compact smooth regular saddle surface with boundary and $p\in \Delta$.
Assume the boundary line of $\Delta$ lies in the unit sphere centered at~$p$.
Show that if $\Delta$ is a disc, then $\length(\partial\Delta)\ge 2\cdot\pi$.

Show that the statement does not hold without assuming that $\Delta$ is a disc.
\end{thm}

\parit{Remark.} In fact $\area \Delta\ge \pi$;
that is, the unit plane disc has minimal possible area.
The proof of this statement can be obtained by applying the so-called \index{coarea formula}\emph{coarea formula} together with the inequality in the exercise. 

\begin{thm}{Exercise}\label{ex:circular-cone-saddle}
Show that an open saddle surface
cannot lie inside an infinite circular cone. 
\end{thm}

A disc $\Delta$ in a surface $\Sigma$ is called a \index{hat}\emph{hat} of $\Sigma$
if its boundary line $\partial\Delta$ lies in a plane $\Pi$ and $\Delta \backslash \partial \Delta$ lies on one side of $\Pi$.

\begin{thm}{Proposition}\label{prop:hat}
A smooth surface $\Sigma$ is saddle if and only if it has no hats.
\end{thm}

Note that a saddle surface can contain a closed plane curve.
For example, the hyperboloid $x^2+y^2-z^2=1$ contains the unit circle in the $(x,y)$-plane.
However, according to the proposition (as well as the lemma), a plane curve cannot bound a disc (as well as any compact set) in a saddle surface.

\parit{Proof.}
Since a plane is convex, the only-if part follows from \ref{lem:convex-saddle};
it remains to prove the if part.

Assume $\Sigma$ is not saddle; that is, it has a point $p$ with strictly positive Gauss curvature;
or equivalently, the principal curvatures $k_1(p)$ and $k_2(p)$ have the same sign.


Let $z=f(x,y)$ be a graph representation of $\Sigma$ in tangent-normal coordinates at~$p$.
Consider the set $F_\epsilon$ in the $(x,y)$-plane defined by the inequality $f(x,y)\le \epsilon$.
By \ref{lem:gauss+=>convexity}, $f$ is convex in a small neighborhood of $(0,0)$.
Therefore, $F_\epsilon$ is convex, for sufficiently small $\epsilon>0$.
In particular, $F_\epsilon$ is a topological disc.

Note that $(x,y)\mapsto (x,y,f(x,y))$ is a homeomorphism from $F_\epsilon$
to
\[\Delta_\epsilon=\set{(x,y,f(x,y))\in \mathbb{R}^3}{f(x,y)\le \epsilon};\]
so $\Delta_\epsilon$ is a topological disc for any small $\epsilon>0$.
Note that the boundary line of $\Delta_\epsilon$ lies on the plane $z=\epsilon$, and the entire disc lies below it;
that is, $\Delta_\epsilon$ is a hat of~$\Sigma$.
\qeds

The following exercise shows that $\Delta_\epsilon$ is in fact a smooth disc.
This can be used to prove a slightly stronger version of \ref{prop:hat};
namely in the definition of hats one can assume that the disc is smooth.

\begin{thm}{Exercise}\label{ex:disc-hat}
Let $f\:\mathbb{R}^2\to\mathbb{R}$ be a smooth strictly convex function with a minimum at the origin.
Show that the set $F_\epsilon$ in the graph $z=f(x,y)$ defined by the inequality $f(x,y)\le \epsilon$ is a smooth disc for any $\epsilon>0$;
that is, there is a diffeomorphism 
$F_\epsilon$ to the unit disc $\Delta\z=\set{(x,y)\in\mathbb{R}^2}{x^2+y^2\le 1}$.
\end{thm}

\begin{thm}{Exercise}\label{ex:saddle-linear}
Let $L\:\mathbb{R}^3\to\mathbb{R}^3$ be an affine transformation; that is, $L$ is an invertible map $\mathbb{R}^3\to\mathbb{R}^3$ that sends any plane to a plane. 
Show that for any saddle surface $\Sigma$ the image $L(\Sigma)$ is also a saddle surface.
\end{thm}


\section{Saddle graphs}

The following theorem was proved by Sergei Bernstein \cite{bernstein}.
\index{Bernstein's theorem}

\begin{thm}{Theorem}\label{thm:bernshtein}
Let $f\:\mathbb{R}^2\to\mathbb{R}$ be a smooth function.
Assume its graph $z=f(x,y)$ is a strictly saddle surface in $\mathbb{R}^3$.
Then $f$ is not bounded;
that is, there is no constant $C$ such that 
$|f(x,y)|\le C$ for any $(x,y)\in\mathbb{R}^2$.
\end{thm}

The theorem states that a saddle graph cannot lie between two parallel horizontal planes;
applying \ref{ex:saddle-linear} we get that saddle graphs cannot lie between two parallel planes,
not necessarily horizontal.
The following exercise shows that the theorem does not hold for saddle surfaces that are not graphs.


\begin{thm}{Exercise}\label{ex:between-parallels}
Construct an open strictly saddle surface that lies between two parallel planes.
\end{thm}

Since $\exp(x-y^2)>0$,
the following exercise shows that there are strictly saddle graphs with functions bounded on one side;
that is, both (upper and lower) bounds are needed in the proof of Bernshtein's theorem.

\begin{thm}{Exercise}\label{ex:one-side-bernshtein}
Show that the graph
$z=\exp(x-y^2)$
is strictly saddle.
\end{thm}

The following exercise gives a condition that guarantees that a saddle surface is a graph;
it can be used in combination with Bernshtein's theorem.

\begin{thm}{Advanced exercise}\label{ex:saddle-graph}
Let $\Sigma$ be an open smooth strictly saddle surface in $\mathbb{R}^3$.
Assume there is a compact subset $K\subset \Sigma$ such that the complement $\Sigma\backslash K$ is the graph $z=f(x,y)$ of a smooth function defined in an open domain of the $(x,y)$-plane.
Show that the surface $\Sigma$ is a graph.
\end{thm}

The following lemma gives an analogous statement for an infinte parallelepiped.

\begin{thm}{Lemma}\label{lem:region}
There is no proper strictly saddle smooth surface that has its boundary line in a plane $\Pi$ 
and lies at a bounded distance from a line contained in $\Pi$.
\end{thm}


\parit{Proof.}
Note that by \ref{ex:saddle-linear}, the statement can be reformulated in the following way:
\emph{There is no proper strictly saddle smooth surface 
with  boundary line in the $(x,y)$-plane
and contained in a region of the form:}
\[R=\set{(x,y,z)\in\mathbb{R}^3}{0\le z\le r, 0\le y\le r}.\]
Let us prove this statement.

Assume the contrary, let $\Sigma$ be such a surface.
Consider the projection $\hat \Sigma$ of $\Sigma$ to the $(x,z)$-plane.
It lies in the upper half-plane and below the line $z=r$.

Consider the open upper half-plane $H=\set{(x,z)\in \mathbb{R}^2}{z> 0}$. 
Let $\Theta$ be the connected component of the complement $H\backslash \hat \Sigma$ that contains all the points above the line $z=r$.

Note that $\Theta$ is convex.
If not, then there is a line segment $[p,q]$ for some $p,q\in \Theta$ that cuts from $\hat\Sigma$ a compact piece.
\begin{figure}[!ht]
\vskip-0mm
\centering
\includegraphics{mppics/pic-74}
\vskip0mm
\end{figure}
Consider the plane $\Pi$ thru $[p,q]$ that is perpendicular to the $(x,z)$-plane.
Note that $\Pi$ cuts from $\Sigma$ a compact region $\Delta$.
By a general position argument (see \ref{lem:reg-section}),
we can assume that $\Delta$ is a compact surface with its boundary line in $\Pi$, and the remaining part of $\Delta$ lies on one side from $\Pi$.
Since the plane $\Pi$ is convex, this statement contradicts \ref{lem:convex-saddle}.

Summarizing, $\Theta$ is an open convex set of $H$ that contains all points above $z=r$.
By convexity, together with any point $w$, the set $\Theta$ contains all points on the half-lines that start as $w$ and {}\emph{point up}; that is, in directions with positive $z$-coordinate. 
In other words, with any point $w$,
the set $\Theta$ contains all points with larger $z$-coordinates.
\begin{figure}[!ht]
\vskip-0mm
\centering
\includegraphics{mppics/pic-75}
\vskip0mm
\end{figure}
Since $\Theta$ is open it can be described by an inequality $z>r_0$.
It follows that the plane $z=r_0$ supports $\Sigma$ at some point (in fact at many points).
By \ref{prop:surf-support}, the latter is impossible --- a contradiction.
\qeds

\parit{Proof of \ref{thm:bernshtein}.}
Denote by $\Sigma$ the graph $z=f(x,y)$.
Assume the contrary; that is, $\Sigma$ lies between two planes $z=\pm C$.

Note that the function $f$ cannot be constant.
It follows that the tangent plane $\T_p$ at some point $p\in\Sigma$ is not horizontal.

Denote by $\Sigma^+$ the part of $\Sigma$ that lies above $\T_p$.
Note that it has at least two connected components which are approaching $p$ from both sides 
in the principal direction with positive principal curvature.
Indeed, if there was a curve that runs in $\Sigma^+$ and approaches $p$ from both sides, then it would cut a disc from $\Sigma$ with the boundary line above $\T_p$ and some points below it;
the latter would contradict \ref{lem:convex-saddle}.

\begin{figure}[!ht]
\vskip-0mm
\centering
\includegraphics{mppics/pic-76}
\caption*{The surface $\Sigma$ seeing from above.}
\vskip0mm
\end{figure}

Summarizing, $\Sigma^+$ has at least two connected components, denote them by $\Sigma^+_0$ and $\Sigma^+_1$.
Let $z=h(x,y)=a\cdot x+b\cdot y+c$ be the equation of $\T_p$.
Note that $\Sigma^+$ contains all points in the region
\[R_-=\set{(x,y,f(x,y))\in\Sigma}{h(x,y)< -C}\] 
which is a connected set and no points in 
\[R_+=\set{(x,y,f(x,y))\in\Sigma}{h(x,y)> C}\]
Whence one of the connected components, say $\Sigma^+_0$, lies in the strip
\[R_0=\set{(x,y,f(x,y))\in\Sigma}{|h(x,y)|\le  C}.\]
This set lies at a bounded distance from the line of intersection of $\T_p$ with the $(x,y)$-plane.

Let us move $\T_p$ slightly upward and cut from $\Sigma^+_0$ the piece above the obtained plane, say $\bar\Sigma^+_0$.
By the general position argument (\ref{lem:reg-section}),
we can assume that $\bar\Sigma^+_0$ is a surface with smooth boundary line;
by construction the boundary line lies in the plane.
Note that the obtained surface $\bar\Sigma^+_0$ still lies on a bounded distance to a line.
The latter is impossible by \ref{lem:region}.
\qeds

\section*{Remarks}

Note that Bernstein's theorem and the lemma in its proof do not hold for nonstrictly saddle surfaces;
counterexamples can be found among infinite cylinders over smooth regular curves.
In fact it can be shown that these are the only counterexamples;
a proof is based on the same idea, but it is more technical.

By \ref{prop:hat}, saddle surfaces can be defined as smooth surfaces without hats.
This definition can be used for arbitrary surfaces, not necessarily smooth.
Some results, for example, Bernshtein's characterization of saddle graphs, can be extended to generalized saddle surfaces, but this class of surfaces is far from being understood; see \cite[Chapter 4]{alexander-kapovitch-petrunin2019} and the references therein.
