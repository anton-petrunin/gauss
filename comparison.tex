\chapter{Comparison theorems}
\label{chap:comparison}

The comparison theorems provide a powerful means of applying Euclidean intuition in differential geometry.

\section{Triangles and hinges}

Recall that a shortest path between points $x$ and $y$ in a surface $\Sigma$ is denoted by $[x,y]$ or $[x,y]_\Sigma$, and
$\dist{x}{y}\Sigma$ denotes the \index{intrinsic!distance}\emph{intrinsic distance} from $x$ to $y$ in~$\Sigma$.

A \index{geodesic!triangle}\emph{geodesic triangle} in a surface $\Sigma$ is defined as a triple of points $x,y,z\z\in \Sigma$ with a choice of minimizing geodesics $[x,y]_\Sigma$, $[y,z]_\Sigma$, and $[z,x]_\Sigma$.
The points $x,y,z$ are called the {}\emph{vertices} of the triangle,
the minimizing geodesics $[x,y]$, $[y,z]$, and $[z,x]$ are called its {}\emph{sides};
the triangle itself is denoted by $[xyz]$, or $[xyz]_\Sigma$;
the latter notation is used if we need to emphasize that the triangle lies on the surface~$\Sigma$.\index{10aad@$[xyz]$, $[xyz]_\Sigma$ (geodesic triangle)}

A triangle $[\tilde x\tilde y\tilde z]$ in the plane $\mathbb{R}^2$ is called a \index{model!angle and triangle}\emph{model triangle} of the triangle $[xyz]$
if its corresponding sides are equal;
that is,
\[\dist{\tilde x}{\tilde y}{\mathbb{R}^2}=\dist{x}{y}\Sigma,
\quad
\dist{\tilde y}{\tilde z}{\mathbb{R}^2}=\dist{y}{z}\Sigma,
\quad
\dist{\tilde z}{\tilde x}{\mathbb{R}^2}=\dist{z}{x}\Sigma.
\]
In this case, we write $[\tilde x\tilde y\tilde z]=\tilde\triangle xyz$.
\index{10aae@$\tilde\triangle$ (model triangle)}

A pair of minimizing geodesics $[x,y]$ and $[x,z]$ starting from one point $x$ is called a \index{hinge}\emph{hinge} and is denoted by $\hinge xyz$.\index{10aac@$\hinge yxz$ (hinge)}
The angle between these geodesics at $x$ is denoted by $\measuredangle\hinge xyz$.
The corresponding angle $\measuredangle\hinge {\tilde x}{\tilde y}{\tilde z}$ in a model triangle $[\tilde x\tilde y\tilde z]=\tilde\triangle xyz$ is denoted by $\modangle xyz$;
\index{10aab@$\modangle yxz$ (model angle)}
it is called {}\emph{model angle} of the triangle $[xyz]$ at~$x$.

By side-side-side congruence condition,
the model triangle $[\tilde x\tilde y\tilde z]$ is uniquely defined up to congruence.
Therefore, the model angle $\tilde\theta=\modangle xyz$ is uniquely defined as well.
By the cosine rule, we have 
\[\cos \tilde\theta=\frac{a^2+b^2-c^2}{2\cdot a \cdot b},\]
where $a=\dist{x}{y}{\Sigma}$, $b=\dist{x}{z}{\Sigma}$, and $c=\dist{y}{z}{\Sigma}$.

\begin{thm}{Exercise}\label{ex:wide-hinges}
Let $[x_ny_nz_n]$ be a sequence of triangles in a smooth surface~$\Sigma$.
Set $a_n=\dist{x_n}{y_n}{\Sigma}$,
$b_n=\dist{x_n}{z_n}{\Sigma}$,
and $c_n=\dist{y_n}{z_n}{\Sigma}$, and $\tilde\theta_n=\modangle {x_n}{y_n}{z_n}$.
Suppose the sequences $a_n$ and $b_n$ are bounded away from zero;
that is, $a_n>\epsilon$ and $b_n>\epsilon$ for a fixed $\epsilon>0$ and any~$n$.
Show that
\[(a_n+b_n-c_n)\to 0\qquad\iff\qquad \tilde\theta_n\to \pi\]
as $n\to\infty$
\end{thm}

\section{Formulations}

Part \ref{SHORT.thm:comp:cat} of the following theorem is called the {}\emph{Cartan--Hadamard theorem};
it was proved by 
Hans von Mangoldt \cite{mangoldt} and generalized by 
Elie Cartan \cite{cartan} and
Jacques Hadamard \cite{hadamard}.
Part \ref{SHORT.thm:comp:toponogov} is called the {}\emph{Toponogov comparison theorem} and sometimes the {}\emph{Alexandrov comparison theorem};
it was proved by Paolo Pizzetti \cite{pizzetti}, rediscovered by Alexandr Alexandrov \cite{aleksandrov}, and 
generalized by Victor Toponogov \cite{toponogov1957}.%, Mikhael Gromov, Yuri Burago and Grigory Perelman~\cite{burago-gromov-perelman}.

Recall that a surface $\Sigma$ is called \index{simply-connected surface}\emph{simply-connected} if any simple closed curve in $\Sigma$ bounds a disc.

\begin{thm}{Comparison theorems}
\label{thm:comp}
\index{comparison theorem}
\ 

\begin{subthm}{thm:comp:cat}
If $\Sigma$ is an open simply-connected  smooth surface with nonpositive Gauss curvature, then 
\[\measuredangle\hinge {x}{y}{z}\le\modangle xyz\]
for any geodesic triangle $[xyz]$ in $\Sigma$.
\end{subthm}

\begin{subthm}{thm:comp:toponogov}
If $\Sigma$ is a closed (or open) smooth surface with nonnegative Gauss curvature, then 
 \[\measuredangle\hinge {x}{y}{z}\ge\modangle xyz\]
for any geodesic triangle $[xyz]$.
\end{subthm}

\end{thm}

The proofs of parts \ref{SHORT.thm:comp:cat} and \ref{SHORT.thm:comp:toponogov} will be given in sections~\ref{sec:nonpos-comp} and~\ref{sec:nonneg-comp}, respectively.

\begin{thm}{Exercise}\label{ex:thm:comp:cat:nsc}
Show that the conclusion in \ref{thm:comp:cat} does not hold the hyperboloid $\set{(x,y,z)\in\mathbb{R}^3}{x^2+y^2-z^2=1}$.
In particular, \ref{thm:comp:cat} does not hold without assuming that $\Sigma$ is simply-connected.
\end{thm}

Let us compare the Gauss--Bonnet formula with the comparison theorems.
Suppose a disc $\Delta$ is bounded by a geodesic triangle $[xyz]$ with internal angles $\alpha$, $\beta$, and~$\gamma$.
Then the Gauss--Bonnet implies that 
\[\alpha+\beta+\gamma-\pi=\iint_\Delta K;\]
in particular, both sides of the equation have the same sign.
It follows that
\begin{itemize}
\item if $K_\Sigma\ge 0$, then $\alpha+\beta+\gamma\ge\pi$, and
\item if $K_\Sigma\le 0$, then $\alpha+\beta+\gamma\le\pi$.
\end{itemize}

\begin{wrapfigure}{r}{35mm}
\centering
\vskip-10mm
\includegraphics{mppics/pic-2307}
\end{wrapfigure}

Now set 
$\hat\alpha\z=\measuredangle\hinge {x}{y}{z}$,
$\hat\beta\z=\measuredangle\hinge {y}{z}{x}$,
and $\hat\gamma\z=\measuredangle\hinge {z}{x}{y}$.
Note that $\hat\alpha,\hat\beta,\hat\gamma\in[0,\pi]$.
Since the angles of any plane triangle add up $\pi$, from the comparison theorems, we get that
\begin{itemize}
\item if $K_\Sigma\ge 0$, then $\hat\alpha+\hat\beta+\hat\gamma\ge\pi$, and
\item if $K_\Sigma\le 0$, then $\hat\alpha+\hat\beta+\hat\gamma\le\pi$.
\end{itemize}

Now we see that despite the Gauss--Bonnet formula and the comparison theorems are closely related,
this relationship is not straightforward.

For example, suppose $K\ge 0$.
Then the Gauss--Bonnet formula does not forbid the internal angles $\alpha$, $\beta$, and $\gamma$ to be simultaneously close to $2\cdot\pi$.
But $\hat \alpha=\alpha$ if $\alpha\le \pi$, and otherwise $\hat \alpha=\pi-\alpha$;
that is,
\begin{align*}
\hat \alpha&=\min\{\,\alpha,2\cdot\pi-\alpha\,\},
&
\hat\beta &=\min\{\,\beta,2\cdot\pi-\beta\,\},
&
\hat\gamma&=\min\{\,\gamma,2\cdot\pi-\gamma\,\}.
\end{align*}
Therefore,
if $\alpha$, $\beta$, and $\gamma$ are each close to $2\cdot\pi$, then $\hat\alpha$, $\hat\beta$, and $\hat\gamma$ are close to $0$.
The latter is impossible by the comparison theorem.


\begin{thm}{Exercise}\label{ex:diam-angle}
Let $p$ and $q$ be points on a closed convex surface $\Sigma$ that lie at a maximal intrinsic distance from each other;
that is, for any $x,y\in \Sigma$ we have $\dist{p}{q}\Sigma\z\ge\dist{x}{y}\Sigma$.
Show that $\measuredangle\hinge xpq\ge \tfrac\pi3$
for any point $x\in \Sigma\setminus\{p,q\}$.
\end{thm}

\begin{thm}{Exercise}\label{ex:sum=<2pi}
Let $\Sigma$ be a closed (or open) surface with nonnegative Gauss curvature.
Show that 
\[\modangle pxy+\modangle pyz+\modangle pzx\le2\cdot \pi.\]
for any four distinct points $p,x,y,z$ in~$\Sigma$.
\end{thm}

\section{Local comparisons}\label{sec:loc-comp}

The following local theorem is the first step in the proof of the comparison theorems (\ref{thm:comp});
it will be deduced follows from the Rauch comparison (\ref{prop:rauch}).

\begin{thm}{Theorem}\label{thm:loc-comp}
The comparison theorem (\ref{thm:comp}) holds in a small neighborhood of any point.

Moreover, let $\Sigma$ be a smooth surface without boundary.
Then for any $p\in \Sigma$ there is $r>0$ such that if $\dist{p}{x}\Sigma<r$, then $\inj(x)_\Sigma>r$, and the following statements hold:

\begin{subthm}{thm:loc-comp:cba}
If $\Sigma$ has nonpositive Gauss curvature, then 
\[\measuredangle\hinge {x}{y}{z}\le\modangle xyz\]
for any geodesic triangle $[xyz]$ in $B(p,\tfrac r4)_\Sigma$.
\end{subthm}

\begin{subthm}{thm:loc-comp:cbb}
If $\Sigma$ has nonnegative Gauss curvature, then 
\[\measuredangle\hinge {x}{y}{z}\ge\modangle xyz\]
for any geodesic triangle $[xyz]$ in $B(p,\tfrac r4)_\Sigma$.
\end{subthm}

\end{thm}

\parbf{Proof.}
The existence of $r>0$ follows from \ref{prop:exp}.
Let $[xyz]$ be a geodesic triangle in $B(p,\tfrac{r}4)$.

Since $r<\inj(x)_\Sigma$, we can choose $\vec v,\vec w\in\T_x$ such that 
\begin{align*}
y&=\exp_x\vec v,
& 
z&=\exp_x\vec w,
\\
|\vec v|_{\T_x}&=\dist{x}{y}\Sigma,&
|\vec w|_{\T_x}&=\dist{x}{z}\Sigma,
&\measuredangle\hinge 0{\vec v}{\vec w}_{\T_x}&=\measuredangle\hinge xyz_\Sigma;
\end{align*}
in particular, $|\vec v|, |\vec w|< \tfrac r2$.

\parit{\ref{SHORT.thm:loc-comp:cba}.}
Consider a minimizing geodesic $\gamma$ joining $y$ to $z$.
Since $\dist{x}{y}{\Sigma},\dist{x}{z}{\Sigma}\z<\tfrac r2$, the triangle inequality implies that $\gamma$ lies in the $r$-neighborhood of~$x$.
In particular, $\log_x\circ\gamma$ is defined, and the curve
$\tilde \gamma\df\log_x\circ\gamma$ lies in an $r$-neighborhood of zero in $\T_x$.
Note that $\tilde\gamma$ connects $\vec v$ to $\vec w$ in $\T_x$.

By Rauch comparison (\ref{prop:rauch:K=<0}), we have
\[\length \tilde \gamma\le \length\gamma.\]
Since $\dist{\vec v}{\vec w}{\T_x}\le\length\tilde \gamma$ and $\length\gamma=\dist{y}{z}{\Sigma}$, we get 
\[\dist{\vec v}{\vec w}{\T_x}\le \dist{y}{z}\Sigma.\]
By angle monotonicity (\ref{lem:angle-monotonicity}), we get
\[\measuredangle\hinge 0{\vec v}{\vec w}_{\T_x}\le \modangle xzy,\]
hence the result.

\parit{\ref{SHORT.thm:loc-comp:cbb}.}
Consider the line segment $\tilde \gamma$ joining $\vec v$ to $\vec w$ in the tangent plane $\T_x$, and set $\gamma\df\exp_x\circ\tilde \gamma$.
By Rauch comparison (\ref{prop:rauch:K>=0}), we have
\[\length \tilde \gamma\ge\length\gamma.\]
Since $\dist{\vec v}{\vec w}{\T_x}=\length\tilde \gamma$ and $\length\gamma\ge\dist{y}{z}\Sigma$, we get 
\[\dist{\vec v}{\vec w}{\T_x}\ge \dist{y}{z}\Sigma.\]
By the angle monotonicity (\ref{lem:angle-monotonicity}), we obtain
\[\measuredangle\hinge 0{\vec v}{\vec w}_{\T_x}\ge\modangle xzy,\]
hence the result.
\qeds

\section{Nonpositive curvature}\label{sec:nonpos-comp}

\parbf{Proof of \ref{thm:comp:cat}.}
Sine $\Sigma$ is simply-connected, \ref{ex:unique-geod} implies that 
\[\inj(p)_\Sigma=\infty\]
for any $p\in\Sigma$.
Therefore, \ref{thm:loc-comp:cba} implies \ref{thm:comp:cat}.
\qeds

\section{Nonnegative curvature}\label{sec:nonneg-comp}

We will prove \ref{thm:comp:toponogov}, first assuming that $\Sigma$ is compact.
The general case requires only minor modifications; they are indicated in Exercise \ref{ex:open-comparison} at the end of the section.
The proof is taken from \cite{alexander-kapovitch-petrunin2027}, and it is close to the proof found independently by Urs Lang and Viktor Schroeder \cite{lang-schroeder}.

\parbf{Proof of \ref{thm:comp:toponogov} in the compact case.}\label{proof(thm:comp:toponogov)}
Assume $\Sigma$ is compact. 
From the local theorem (\ref{thm:loc-comp}), we get that there is $\epsilon>0$ such that the inequality 
\[\measuredangle\hinge {x}{p}{q}\ge\modangle xpq.\]
holds for any hinge $\hinge{x}{p}{q}$ with 
$\dist{x}{p}\Sigma+\dist{x}{q}\Sigma<\epsilon$.
The following lemma states that in this case, the same holds true for any hinge $\hinge{x}{p}{q}$ such that $\dist{x}{p}\Sigma+\dist{x}{q}\Sigma<\tfrac32\cdot\epsilon$.
Applying the key lemma (\ref{key-lem:globalization}) a few times we get that the comparison holds for an arbitrary hinge, which proves \mbox{\ref{thm:comp:toponogov}}.
\qeds

\begin{thm}{Key lemma}\label{key-lem:globalization} 
Let $\Sigma$ be an open or closed smooth surface.
Assume the comparison
\[\measuredangle\hinge x y z
\ge\modangle x y z\eqlbl{eq:key-lem:globalization}\]
holds for any hinge $\hinge x y z$ with 
$\dist{x}{y}\Sigma+\dist{x}{z}\Sigma
<
\frac{2}{3}\cdot\ell$.
Then the comparison \ref{eq:key-lem:globalization}
holds for any hinge $\hinge x y z$ with $\dist{x}{y}\Sigma+\dist{x}{z}\Sigma<\ell$.
\end{thm}

{

\begin{wrapfigure}{r}{35mm}
\centering
\vskip-3mm
\includegraphics{mppics/pic-2308}
\end{wrapfigure}

Given a hinge $\hinge x p q$ consider a triangle in the plane
with angle $\measuredangle\hinge x p q$ and two adjacent sides $\dist{x}{p}\Sigma$ and $\dist{x}{q}\Sigma$.
Let us denote by $\side \hinge x p q$ the third side of this triangle;
it will be called the \index{model!side}\emph{model side} of the hinge.

The next computational exercise plays a role in the following proof.

}

\begin{thm}{Exercise}\label{ex:s-r}
Suppose hinges $\hinge xpq$ and $\hinge xpy$ have a common side $[x,p]$ and $[x,y]\subset [x,q]$.
Show that 
\[\frac{\dist{x}{p}{}+\dist{x}{q}{}-\side\hinge xpq}{\dist{x}{q}{}}
\le
\frac{\dist{x}{p}{}+\dist{x}{y}{}-\side\hinge xpy}{\dist{x}{y}{}}.\]
\end{thm}

\parbf{Proof.} 
By the angle monotonicity (\ref{lem:angle-monotonicity}), we have
\[\measuredangle\hinge x p q\ge \modangle x p q\quad\iff\quad\side \hinge x p q
\ge\dist{p}{q}\Sigma.\]
Therefore, it is sufficient to prove that
\[\side \hinge x p q
\ge\dist{p}{q}\Sigma,
\eqlbl{eq:thm:=def-loc}\]
assuming that $\dist{x}{p}\Sigma+\dist{x}{q}\Sigma<\ell$.

Let us produce a new hinge $\hinge{x'}p q$ for a given hinge $\hinge x p q$ such that 
\[\tfrac{2}{3}\cdot\ell \le\dist{p}{x}\Sigma\z+\dist{x}{q}\Sigma< \ell.\]

\begin{wrapfigure}{r}{32mm}
\vskip0mm
\centering
\includegraphics{mppics/pic-2310}
\end{wrapfigure}

Assume $\dist{x}{q}\Sigma\ge\dist{x}{p}\Sigma$; otherwise, switch the roles of $p$ and $q$ in the construction.
Take $x'\in [x, q]$ such that 
\[\dist{p}{x}\Sigma+3\cdot\dist{x}{x'}\Sigma
=\tfrac{2}{3}\cdot\ell
\eqlbl{3|xx'|}\]
Choose a geodesic $[x', p]$, and consider the hinge $\hinge{x'}p q$ formed by $[x',p]$ and $[x',q]\subset [x,q]$.

By the triangle inequality, we have 
\[
\dist{p}{x}\Sigma+\dist{x}{q}\Sigma\ge\dist{p}{x'}\Sigma+\dist{x'}{q}\Sigma.
\eqlbl{eq:thm:=def-loc-fourstar}\]
Let us show that
\[\side \hinge x p q
\ge\side \hinge{x'}p q
\eqlbl{eq:thm:=def-loc-fivestar}\]

By \ref{3|xx'|}, we have that 
\[
\begin{aligned}
\dist{p}{x}{\Sigma}\z+\dist{x}{x'}{\Sigma}&<\tfrac{2}{3}\cdot\ell,
\\
\dist{p}{x'}{\Sigma}\z+\dist{x'}{x}{\Sigma}&<\tfrac{2}{3}\cdot\ell.
\end{aligned}
\]
Therefore, the assumption imply that 
\[\begin{aligned}
\measuredangle\hinge x p{x'}
\ge\modangle x p{x'}
\quad\text{and}\quad
\measuredangle\hinge {x'}p x
\ge\modangle {x'}p x.
  \end{aligned}
\eqlbl{eq:thm:=def-loc-threestar}
\]

Consider the model triangle
$[\tilde x\tilde x'\tilde p]\z=\modtrig xx'p$.
Take $\tilde q$ on the extension of $[\tilde x,\tilde x']$ beyond $x'$ such that $\dist{\tilde x}{\tilde q}\Sigma=\dist{x}{q}\Sigma$, and therefore $\dist{\tilde x'}{\tilde q}\Sigma\z=\dist{x'}{q}\Sigma$.

From \ref{eq:thm:=def-loc-threestar}, we get
\[\measuredangle\hinge x p q
=\measuredangle\hinge x p{x'}\ge\modangle x p{x'}.\]
Therefore,
\[
\side\hinge x q p
\ge
\dist{\tilde p}{\tilde q}{\mathbb{R}^2}.
\]
Further, since $\measuredangle\hinge{x'}p x+\measuredangle\hinge{x'}p q= \pi$,
the inequalities in \ref{eq:thm:=def-loc-threestar} imply
\[
\pi
-\modangle{x'}p x
\ge
\pi-\measuredangle\hinge{x'}p x
\ge
\measuredangle\hinge{x'}p q.
\]
Therefore,
\[\dist{\tilde p}{\tilde q}{\mathbb{R}^2}\ge\side \hinge{x'}q p \]
and \ref{eq:thm:=def-loc-fivestar} follows.

Set $x_0=x$; apply inductively the above construction to get a sequence of hinges $\hinge{x_n}p q$ with $x_{n+1}=x_n'$.
By \ref{eq:thm:=def-loc-fivestar} and \ref{eq:thm:=def-loc-fourstar}, both sequences
\[s_n=\side \hinge{x_n}pq\quad\text{and}\quad r_n=\dist{p}{x_n}\Sigma+\dist{x_n}{q}\Sigma\]
are nonincreasing.

The sequence might terminate at $x_n$ only if $r_n< \tfrac{2}{3}\cdot\ell $.
In this case, by the assumptions of the lemma, we have
\[s_n=\side \hinge{x_n}p q\ge\dist{p}{q}\Sigma.\]
Since the sequence $s_n$ is nonincreasing, we get
\[\side \hinge{x}p q=s_0\ge s_n\ge\dist{p}{q}\Sigma;\]
whence \ref{eq:thm:=def-loc} follows.

\begin{figure}[!ht]
\centering
\includegraphics{mppics/pic-2315}
\end{figure}

It remains to prove \ref{eq:thm:=def-loc} if the sequence $x_n$ does not terminate.
By \ref{3|xx'|}, we have 
\[
\dist{x_n}{x_{n-1}}\Sigma
\ge 
\tfrac1{100}\cdot \ell.
\eqlbl{eq:|x-x|><l}
\]
By \ref{eq:thm:=def-loc-fourstar}, $\dist{x_n}{p}{},\dist{x_n}{q}{}<
\ell$ for any~$n$.
In case $x_{n+1}\in [x_n,q]$, apply \ref{ex:s-r} for the hinges $\hinge{x_n}pq$ and $\hinge{x_n}p{x_{n+1}}$.
By \ref{eq:thm:=def-loc-threestar}, $\dist{p}{x_{n+1}}{}\le \side \hinge{x_n}{x_{n+1}}{p}$.
Therefore,
\[r_n-s_n\le 100\cdot (r_n-r_{n+1})\eqlbl{eq:r-s<100(r-r)}\]
In the case $x_{n+1}\in [x_n,p]$, inequality \ref{eq:r-s<100(r-r)} follows if one applies \ref{ex:s-r} for the hinges $\hinge{x_n}pq$ and $\hinge{x_n}{x_{n+1}}q$.

The sequences $r_n$ and $s_n$ are nonincreasing and nonnegative;
so, they have to converge.
In particular, $(r_n-r_{n+1})\to0$ as $n\to \infty$.
Therefore, \ref{eq:r-s<100(r-r)} implies that
\[\lim_{n\to\infty}s_n=\lim_{n\to\infty}r_n.\]
By the triangle inequality, $r_n\ge \dist{p}{q}\Sigma$ for any~$n$.
Since $s_n$ is nonincreasing, we get
\[\side \hinge{x}p q=s_0\ge \lim_{n\to\infty}s_n=\lim_{n\to\infty}r_n\ge \dist{p}{q}\Sigma\]
which finishes the proof of \ref{eq:thm:=def-loc}.
\qeds

\begin{thm}{Exercise}\label{ex:open-comparison}
Let $\Sigma$ be an open surface with nonnegative Gauss curvature.
Given $p\in\Sigma$, denote by $R_p$ 
(the {}\emph{comparison radius} at $p$) 
the maximal value (possibly $\infty$) such that the comparison 
\[\measuredangle\hinge x p y
\ge\modangle x p y\]
holds for any hinge $\hinge x p y$ with $\dist{p}{x}\Sigma+\dist{x}{y}\Sigma<R_p$.

\begin{subthm}{ex:open-comparison:positive}
Show that for any compact subset $K\subset \Sigma$, there is $\epsilon>0$ such that $R_p>\epsilon$ for any $p\in K$.
\end{subthm}

\begin{subthm}{ex:open-comparison:almost-min}
Use part \ref{SHORT.ex:open-comparison:positive} to show that 
there is a point $p\in\Sigma$ such that 
\[R_q>(1-\tfrac1{100})\cdot R_p,\]
for any $q\in B(p,100\cdot R_p)_\Sigma$.
\end{subthm}

\begin{subthm}{ex:open-comparison:proof}
Explain how to use \ref{SHORT.ex:open-comparison:almost-min} to extend the proof of \ref{thm:comp:toponogov} (page \pageref{proof(thm:comp:toponogov)}) to open surfaces. 
(That is, to show that $R_p=\infty$ for any $p\in\Sigma$.) 
\end{subthm}

\end{thm}

\section{Alexandrov's lemma}
\index{Alexandrov's lemma}

A reformulation of the following lemma (\ref{lem:alex-reformulation}) will be used in the next section to produce a few equivalent reformulations of the comparison theorems.

\begin{thm}{Lemma}
\label{lem:alex}
Consider two quadrangles in the Euclidean plane $pxyz$ and $p'x'y'z'$ with equal corresponding sides.
Assume the sides $[x',y']$ and $[y',z']$ extend each other; that is, $y'$ lies on the line segment $[x',z']$.
Then the following expressions have the same signs:
\begin{enumerate}[(i)]
 \item $|p-y|-|p'-y'|$;
 \item $\measuredangle\hinge xpy-\measuredangle\hinge {x'}{p'}{y'}$;
 \item $\pi-\measuredangle\hinge ypx-\measuredangle\hinge ypz$.
\end{enumerate}
\end{thm}

\parbf{Proof.} 
Take 
a point $\bar z$ on the extension of 
$[x,y]$ beyond $y$ so that $\dist{y}{\bar z}{}=\dist{y}{z}{}$ (and therefore $\dist{x}{\bar z}{}=\dist{x'}{z'}{}$). 
 
\begin{figure}[!ht]
\vskip-0mm
\centering
\includegraphics{mppics/pic-50}
\vskip-0mm
\end{figure}

By the angle monotonicity (\ref{lem:angle-monotonicity}), 
the following expressions have the same sign:
\begin{enumerate}[(i)]
\item $|p-y|-|p'-y'|$;
\item $\measuredangle\hinge{x}{y}{p}-\measuredangle\hinge{x'}{y'}{p'}=\measuredangle\hinge{x}{\bar z}{p}-\measuredangle\hinge{x'}{z'}{p'}$;
\item $|p-\bar z|-|p'-z'| = | p - \bar z | - | p-z | $;
\item $\measuredangle\hinge{y}{\bar z}{p}-\measuredangle\hinge{y}{z}{p}$.
\end{enumerate}
The statement follows since
\[\measuredangle\hinge{y'}{z'}{p'}+\measuredangle\hinge{y'}{x'}{p'}=\pi
\quad\text{and}\quad
\measuredangle\hinge{y}{\bar z}{p}+\measuredangle\hinge{y}{x}{p}=\pi.\]
\qedsf

\section{Reformulations}

For any triangle $[xyz]$ in a surface $\Sigma$, and its model triangle $[\tilde x \tilde y \tilde z]$, there is a natural map $p\mapsto \tilde p$ that isometrically sends the geodesics $[x,y]$, $[y,z]$, $[z,x]$ to the line segments $[\tilde x,\tilde y ]$, $[\tilde y , \tilde z]$, $[ \tilde z , \tilde x ]$, respectively.
The triangle $[xyz]$ is called \index{fat triangle}\emph{fat} (\index{thin triangle}\emph{thin})
if the inequality
\[\dist{p}{q}{\Sigma}\ge |\tilde p- \tilde q|_{\mathbb{R}^2}\qquad \text{(or, respectively,}\quad \dist{p}{q}{\Sigma}\le |\tilde p- \tilde q|_{\mathbb{R}^2})\]
holds for any two points $p$ and $q$ on the sides of $[xyz]$.

\begin{thm}{Proposition}\label{prop:comp-reformulations}
Let $\Sigma$ be an open or closed smooth surface.
Then the following three conditions are equivalent:

\begin{subthm}{mang>angk}
For any geodesic triangle $[xyz]$ in $\Sigma$ we have
 \[\measuredangle\hinge {x}{y}{z}\ge\modangle xyz.\]
\end{subthm}

\begin{subthm}{angk>angk} For any geodesic triangle $[pxz]$ in $\Sigma$ and $y$ on the side $[x,z]$ we have
 \[\modangle xpy \ge \modangle xpz.\]
 
\end{subthm}

\begin{subthm}{fat}
 Any geodesic triangle in $\Sigma$ is fat.
\end{subthm}

\medskip

Similarly, the following three conditions are equivalent:

\begin{subthmA}{mang<angk}
For any geodesic triangle $[xyz]$ in $\Sigma$ we have
 \[\measuredangle\hinge {x}{y}{z}\le\modangle xyz.\]
\end{subthmA}

\begin{subthmA}{angk<angk} For any geodesic triangle $[pxz]$ in $\Sigma$ and $y$ on the side $[x,z]$ we have
 \[\modangle xpy \le \modangle xpz.\]
\end{subthmA}

\begin{subthmA}{thin}
Any geodesic triangle in $\Sigma$ is thin.
\end{subthmA}

\end{thm}

Let us rewrite the Alexandrov lemma (\ref{lem:alex}) in the language of comparison triangles and angles.

\begin{thm}{Reformulation of Alexandrov lemma}\label{lem:alex-reformulation}
Let $[pxz]$ be a triangle in a surface $\Sigma$
and $y$ a point on the side $[x,z]$.
Consider its model triangle $[\tilde p\tilde x\tilde z]\z=\tilde\triangle pxz$, and let $\tilde y$ be the corresponding point on the side $[\tilde x,\tilde z]$.

\begin{wrapfigure}{r}{25mm}
\vskip-2mm
\centering
\includegraphics{mppics/pic-2305}
\end{wrapfigure}

Then the following expressions have the same signs:
\begin{enumerate}[(i)]
 \item $\dist{p}{y}\Sigma-\dist{\tilde p}{\tilde y}{\mathbb{R}^2}$;
 \item $\modangle xpy-\modangle {x}{p}{z}$;
 \item $\pi-\modangle ypx-\modangle ypz$;
\end{enumerate}
\end{thm}

\parbf{Proof of \ref{prop:comp-reformulations}.}
We will prove the implications \ref{SHORT.mang>angk}$\Rightarrow$\ref{SHORT.angk>angk}$\Rightarrow$\ref{SHORT.fat}$\Rightarrow$\ref{SHORT.mang>angk}.
The implications \ref{SHORT.mang<angk}$\Rightarrow$\ref{SHORT.angk<angk}$\Rightarrow$\ref{SHORT.thin}$\Rightarrow$\ref{SHORT.mang<angk} can be proved in the same way with all inequalities reversed.

\parit{\ref{SHORT.mang>angk}$\Rightarrow$\ref{SHORT.angk>angk}.}
Note that $\measuredangle\hinge ypx+\measuredangle\hinge ypz=\pi$.
By \ref{SHORT.mang>angk}, 
\[\modangle ypx+\modangle ypz\le \pi.\]
It remains to apply Alexandrov's lemma (\ref{lem:alex-reformulation}).

\parit{\ref{SHORT.angk>angk}$\Rightarrow$\ref{SHORT.fat}.}
Applying \ref{SHORT.angk>angk} twice, first for $y\in [x,z]$ and then for $w\in [p,x]$, we get that
\[\modangle xwy \ge \modangle xpy \ge \modangle xpz,\]
and therefore
\[\dist{w}{y}\Sigma\ge \dist{\tilde w}{\tilde y}{\mathbb{R}^2},\]
where $\tilde w$ and $\tilde y$ are the points corresponding to $w$ and $y$ points on the sides of the model triangle. 

\parit{\ref{SHORT.fat}$\Rightarrow$\ref{SHORT.mang>angk}.}
Since the triangle is fat, we have 
\[\modangle xwy \ge \modangle xpz\]
for any $w\in [x,p]\setminus \{x\}$ and $y\in [x,z]\setminus \{x\}$.
Note that $\modangle xwy\to \measuredangle\hinge xpz$ as $w,y\to x$.
Whence the implication follows.
\qeds

\begin{thm}{Exercise}\label{ex:geod-convexity}
Let $\Sigma$ be an open smooth surface, 
 $\gamma$ a unit-speed geodesic in $\Sigma$, and $p\in\Sigma$.

Consider the function
\[h(t)=\dist{p}{\gamma(t)}\Sigma^2-t^2.\]

\begin{subthm}{}
Show that if $\Sigma$ is simply-connected and the Gauss curvature of $\Sigma$ is nonpositive, then the function $h$ is convex.
\end{subthm}

\begin{subthm}{} Show that if the Gauss curvature of $\Sigma$ is nonnegative, then the function $h$ is concave.
\end{subthm}

\end{thm}

\begin{thm}{Exercise}\label{ex:midpoints}
Let $\bar x$ and $\bar y$ be the midpoints of minimizing geodesics $[p,x]$ and $[p,y]$ in an open smooth surface~$\Sigma$.

\begin{subthm}{}
Show that if $\Sigma$ is simply-connected and has nonpositive Gauss curvature, then 
\[2\cdot \dist{\bar x}{\bar y}\Sigma\le \dist{x}{y}\Sigma.\]
\end{subthm}
 
\begin{subthm}{} Show that if the Gauss curvature of $\Sigma$ is nonnegative, then 
 \[2\cdot \dist{\bar x}{\bar y}\Sigma\ge \dist{x}{y}\Sigma.\]
\end{subthm}

\end{thm}

\begin{thm}{Exercise}\label{ex:convex-dist}
Let $\gamma_1$ and $\gamma_2$ be two geodesics in an open smooth simply-connected surface $\Sigma$ with nonpositive Gauss curvature.
Show that the function $h(t)\df\dist{\gamma_1(t)}{\gamma_2(t)}\Sigma$
is convex.
\end{thm}

\begin{thm}{Exercise}\label{ex:disc+}
Let $\Sigma$ be an open or closed smooth surface $\Sigma$ with nonnegative Gauss curvature.
Show that area of any $R$-disc in the intrinsic metric of $\Sigma$ is at most $\pi\cdot R^2$.

\end{thm}

\begin{thm}{Exercise}\label{ex:disc-}
Let $\Delta$ be an $R$-disc in the intrinsic metric of an open simply-connected smooth surface $\Sigma$  with nonpositive Gauss curvature.

\begin{subthm}{ex:disc-:kg}
Show that the boundary of $\Delta$ is a smooth curve with geodesic curvature at least $\tfrac1R$.
\end{subthm}

\begin{subthm}{ex:disc-:area}
Show that the area of $\Delta$ is at least $\pi\cdot R^2$.
\end{subthm}

\end{thm}

The following exercise generalizes the Moon in a puddle theorem (\ref{thm:moon-orginal}).

\begin{thm}{Advanced exercise}\label{ex:moon-}
Let $\Delta$ be a disc in a smooth surface $\Sigma$ with nonpositive Gauss curvature.
Assume $\Delta$ is bounded by a smooth curve $\gamma$ with geodesic curvature at most 1 by absolute value.
Show that $\Delta$ contains a unit disc in the intrinsic metric of $\Sigma$.

Conclude that the area of $\Delta$ is at least $\pi$.
\end{thm}

